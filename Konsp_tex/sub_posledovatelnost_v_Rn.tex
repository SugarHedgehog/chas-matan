\begin{opr}
Последовательностью в \Rn называется отображение $f:\N\to\R^n$.
\end{opr}
Это означает, что $\forall(k\in\N)\exists(x_k\in\R^n)[f(k)=x_k]$.

\begin{primer}
\begin{multline*}
\left\{x_k=\left( \frac{1}{k};k^2+1;2^k; \frac{k}{3k+1}\right)\right\}
\\
x_1=\left(1;2;2;\frac{1}{4}\right)
\\
x_2=\left(\frac{1}{2};5;4;\frac{2}{7}\right)
\end{multline*}
и т. д.
\end{primer}

\begin{opr}
Пусть $\{x_k\}\subset\R^n$ - последовательность.
Если $$\exists(x_0\in\R^n)[\{\|x_k-x_0\|\}\to0]$$ (здесь $\{\|x_k-x_0\|\}$ -- числовая последовательность), то говорят, что $\{x_k\}$ сходится к $x_0$ и пишут:
$$
\{x_k\}\to x_0
$$
или
$$
\lim x_k =  x_0
$$
или
$$
\lim_{k\to\infty} x_k =  x_0
$$
\end{opr}
Иначе говоря,
\begin{equation*}
\{x_k\}\to x_0 \Leftrightarrow \forall(\varepsilon>0)\exists(k_0\in\N)\forall(k>k_0)[\|x_k-x_0\|<\varepsilon]
\end{equation*}

Легко доказать, что если две нормы эквивалентны, то сходимость по первой из этих норм равносильна сходимости по второй.

\begin{teorema}
Сходимость по норме эквивалентна покоординатной сходимости, т. е.
$$
\{x_k\}\to x_0 \Leftrightarrow \forall(i\in\Z\cap[1;n])[x_k^i\to x_0^i]
$$
\end{teorema}
\dokvo
Так как все нормы эквивалентны, то докажем утверждение только для евклидовой нормы (\ref{evklidova_norma}):
$$|x_k-x_0|=\sqrt{\sum_{i=1}^{n}(x_k^i-x_0^i)^2} \to 0
\Rightarrow \forall(i\in\Z\cap[1;n])[x_k^i- x_0^i\to0]
$$
\dokno

\begin{sledstvie}
\begin{multline*}
\forall(\{x_k\}\to x_0:\{x_k\}\subset \R^n,\{y_k\}\to y_0:\{y_k\}\subset \R^n,\{\lambda_k\}\to \lambda_0:\{\lambda_k\}\subset \R)
\\
[\{x_k+y_k\}\to x_0+y_0 ~\cap~ \{\lambda_k x_k\}\to \lambda_0 x_0]
\end{multline*}
\end{sledstvie}

\begin{sledstvie}
Некоторое множество $G\subset\R^n$ ограничено тогда и только тогда, когда ограничено множество, состоящее из вещественных чисел, являющихся координатами элементов $G$.
\end{sledstvie}

\begin{teorema}[Больцано-Вейерштрасса для \Rn]
Из любой ограниченной последовательности можно выделить сходящуюся подпоследовательность.
\end{teorema}
\dokvo
Пусть $\{x_k\}\subset\R^n$ -- последовательность.
Выделим из неё сначала подпоследовательность $\{x_{k_1}\}$ так, что последовательность первых координат $\{x_{k_1}^1\}$ - сходится;
(это возможно по теореме Больцано-Вейерштрасса для $\R$, так как множество значений первых координат ограничено)
затем выделим из $\{x_{k_1}\}$ подпоследовательность $\{x_{k_2}\}$, такую, что последовательность первых координат $\{x_{k_1}^2\}$ - сходится.
Продолжая действовать подобным образом, получим требуемую последовательность $\{x_{k_n}\}$, сходящуюся покоординатно.
\dokno
