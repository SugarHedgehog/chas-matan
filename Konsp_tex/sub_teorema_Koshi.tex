\subsubsection{Теорема Коши.}

Пусть $f:[a;b]\to \R$, $g:[a;b]\to \R$, причём:

1) $f$ и $g$ непрерывны на $[a;b]$;

2) $f$ и $g$ дифференцируемы на $(a;b)$;

3)$\nexists (x \in (a;b))[g(x)=0]$

Тогда

$$
\exists (c \in (a;b))\left[ \frac{f(b)-f(a)}{g(b)-g(a)}=\frac{f'(c)}{g'(c)}\right].
$$

\subsubsection{Замечание 1.}

Теорема Коши не является следствием из теоремы Лагранжа; наоборот, теорема Лагранжа - частный случай теоремы Коши для $g(x)=x$.

\subsubsection{Замечание 2.}

Равенство $ \frac{f(b)-f(a)}{g(b)-g(a)}=\frac{f'(c)}{g'(c)}$ называют формулой конечных приращений Коши.

