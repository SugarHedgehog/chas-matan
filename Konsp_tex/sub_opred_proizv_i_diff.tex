Пусть $X \subset \mathbb {R}$ $X$ - промежуток (отрезок, ...) $f : X \rightarrow \mathbb {R}$.

\opred

Производная $f(x)$ $f'(x)$ - это конечный $lim_{h \to 0} \frac {f(x_0 + h) - f(x_0)}{h}$, где $x_0 \in X$.

$\Delta f(x,h)$ - приращение функции в $x_o$.

$\Delta x = h$ - приращение аргумента.

$\Delta f(x,h) = f(x_0 + h) - f(x_0)$.

Тогда $f'(x_0)= lim_{h \to 0} \frac{\Delta f(x_0,h)}{h}$,

Если $x_0$ - фиксированное, то $f'(x_0) \in \mathbb {R}$

Если $x_0$ - нефиксированное, $a$ пробегает $X$, то $f'(x_0)$ - новая функция от $x$.

\opred
	
Функция $f(x)$ - дифференцируема в $x_0$, если существует такая линейная функция, что $\forall(h: x_0 + h \in X)[f(x_0 + h)-f(x_0) = l(x_0)h +	\omega(x_0, h)]$,

где $\omega(x_0, h)$ - бесконечно малая функция при $h \rightarrow 0$ более высокого порядка, чем $h$.

Значит $\omega(x_0, h) = 0(h)$, т.е. $lim_{h \to 0} \frac {\omega(x_0, h)}{h} = 0$.

Тогда $\Delta f(x_0, h) = l(x_0)h + \omega(x_0,h)$.

\subsubsection{Например}

$f(x)=x^3$

$\Delta f(x_0, h) = f(x_0 + h)-f(x_0) = x_0^3+3hx_0^2+3h^2x_0+h^3-x_0^3 = $

$3x_0^2h + 3h^2x_0 + h^3$.

В данном примере $f(x)=x^3$ - дифференцируема и $l(x_0) = 3x_0^2$.

\opred

Дифференциал $f(x)$ на элементе $h$ в точке $x_0$ $(df(x_0,h)) =$ значение $l(x_0)$.

Значит в нашем примере : $df = 3x_0^2$ 

\begin{teorema}

Связь между производной и дифференциалом.

Для того, чтобы $f: X \rightarrow \mathbb {R}$, где $X$ - промежуток из $\mathbb {R}$, была дифференцируема в $x_0$.

Необходимо и достаточно, чтобы $\exists (f'(x_0))$, и в уравнении 

$f(x_0 + h) - f(x_0) = l(x_0)h +	\omega(x_0, h)$. $[l(x_0=f'(x_0)]$

\end{teorema}

Доказательство:

Необходимость: дано - $f$ - дифференцируема, 

доказать $f(x_0 + h) - f(x_0) = l(x_0)h +	\omega(x_0, h)$

$\frac {f(x_0 + h) - f(x_0)}{h} = \frac {l(x_0)h}{h} + \frac {\omega(x_0, h)}{h}$;

$lim_{h \to 0}\frac {f(x_0 + h) - f(x_0)}{h} = lim_{h \to 0}\frac {l(x_0)h}{h} + lim_{h \to 0}\frac {\omega(x_0, h)}{h} = l(x_0)$;	

$lim_{h \to 0}\frac {\omega(x_0, h)}{h} = 0$ т.к. $\omega(x_0, h)$ - б.м. функция более высокого порядка, чем $h$).

Отсюда: если $\exists$ конечный $lim_{h \to 0} l(x_0)$ ($l(x_0)$ - предел от правой части равенства). Значит $\exists$ конечный $lim_{h \to 0} \frac {f(x_0 +h) - f(x_0)}{h}$.

По определению $lim_{h \to 0} \frac {f(x_0 +h) - f(x_0)}{h} = f'(x_0)$.

Отсюда $f'(x_0) = l(x_0)$.

Достаточность: Дано: $\exists (f'(x_0) = l(x_0))$. Доказать $f'(x_0)$ отличается от $lim_{h \to 0} \frac {f(x_0 +h) - f(x_0)}{h}$ на бесконечно малое число. 

Значит, если введем $\alpha(x_0, h) = \frac {f(x_0 +h) - f(x_0)}{h} - f'(x_0)$, то при $h \to 0$, $\alpha \to 0$ 

Отсюда:$f(x_0 +h) - f(x_0) - f'(x_0) = f'(x_0)h + \alpha(x_0,h)h$.

Пусть $\omega = \alpha(x_0, h)h$. Тогда $lim_{h \to 0}\frac {\omega(x_0, h)}{h} = 0$.

Т.е. $\omega(x_0, h) = 0(h)$. $\omega(x_0, h)$ - б.м. более высокого порядка малости, чем $h$.

Получается: 

$$f(x_0 +h) - f(x_0) = f'(x_o)h + \omega(x_0,h).      (1)$$

\opred

$$f(x_0 +h) - f(x_0) = l(x_o)h + \omega(x_0,h).       (2) $$

а $f'(x_0)$ при нефиксированном $x_0$ - новая функция от $x_0$, т.е. $l(x_o)$.

Значит, (1) идентична (2) и $(f(x)$ - дифференцируема. Что и требовалось доказать.

Формула дифференциала: $df(x_0, h)=f'(x_0)h$

$f(x)$ - дифференцируема если:

$\Delta f(x_0,h) = l(x_0)h + \omega(x_0, h)$, где $\omega(x_0, h) = 0(h)$/

$\Delta f(x_0,h) = df(x_0,h) + \omega(x_0, h)$.

Т.е. $f(x)$ - главная линейная часть приращения функции.

На последней формуле основано нахождение значения функций в точках окрестности точки $x$,

Если известно $f(x)$: $df(x_0, h)\approx \Delta f(x_0, h)$.

$f(x_0, h) \approx f(x_0) + df(x_0, h)$.

\subsubsection{Например}.

$\sqrt[3] x $. Пусть $x=8,1 : x_0 = 8; h = 0,1$. 

Тогда $\sqrt[3] (8,1) = f(8 + 0,1) \approx f(8) +f'(8)h = \sqrt[3] 8 + \frac {1}{3} * \frac {1}{\sqrt[3] 8^2}= 2 + \frac {1}{120}$

Точки разрыва функций $f(x)$ - точки множества $X$, в которых $f(x)$ разрывна.

Если(пусть) $y=f(x)$. Тогда $df(x_0, h) = dy = f'(x_0)\Delta x$ 

$\Delta x = (x+h)-x = h$.

Если $y=x$, $\Delta x = dx$

$$dy = f'(x_0)\Delta x = f'(x_0)dx$$

$$f'(x_0) = \frac {dy}{dx}$$