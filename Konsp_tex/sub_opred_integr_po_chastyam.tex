Взятие определённого интеграла по частям применяют в тех же случаях, что и неопределённого.
Сформулируем теорему, являющуюся следствием из формулы Ньютона-Лейбница.

\begin{teorema}
Пусть функции $u$ и $v$ непрерывно дифференцируемы на $[a;b]$.
Тогда
$$
\intl_a^b (uv')(x) dx=(uv)(x)\Bigl.\Bigr|_a^b-\intl_a^b(u'v)(x)dx
$$
\end{teorema}

\dokvo
$$
(uv)'(x)=(u'v)(x)+(uv')(x)
$$

Проинтегрировав на $[a;b]$, получаем:

$$
\intl_a^b (uv)'(x)dx=\intl_a^b (u'v)(x) dx + \intl_a^b (uv')(x) dx
$$
То есть

$$
\intl_a^b (uv)'(x)dx-\intl_a^b (u'v)(x) dx = \intl_a^b (uv')(x) dx
$$


По формуле Ньютона-Лейбница
$$
\intl_a^b (uv)'(x)dx = (uv)(x)\Bigl.\Bigr|_a^b
$$

\dokno

