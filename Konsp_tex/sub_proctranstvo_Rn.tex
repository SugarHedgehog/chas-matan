\begin{opr}
Пространство \Rn -- множество упорядоченных наборов из $n$ вещественных чисел:
$$
x\in \R^n \Leftrightarrow x=(x^1, ... , x^n),~~x^i\in\R,~~i=1,...,n
$$
\end{opr}
\begin{zamech}
\Rn -- линейное пространство.
Оно более детально изучается в курсе линейной алгебры.
\end{zamech}

\begin{zamech}
Индекс (номер) координаты вектора пишется вверху, т. к. нижний индекс необходим в выкладках, содержащих последовательности.
Как правило, такие обозначения не приводят к недоразумению и путанице с обозначением степени.
Внимательный читатель заметит, что эти обозначения сходны с обозначениями тензорной алгебры; однако же растановкой индексов мы и ограничимся, а сокращённую запись суммы и другие соглашения заимствовать не будем.
\end{zamech}

Выпишем определения операций в \Rn -- сложения и внешнего умножения:

$$
\forall(x=(x^1,...,x^n)\in\R^n,y=(y^1,...,y^n)\in\R^n)[x+y=(x^1+y^1,...,x^n+y^n)]
$$

$$
\forall(\lambda\in\R)\forall(x=(x^1,...,x^n)\in\R^n)[\lambda x=(\lambda x^1,...,\lambda x^n)]
$$

Нулевой вектор, как и скалярный нуль, и нулевой оператор, и т. д., будем обозначать символом $0$. Опять же, в большинстве случаев к недоразумению такое обозначение не приводит.

Все выкладки будем давать в стандартном базисе $e$:
$$\begin{aligned}
e_1=(1,0,...,0)\\
...\\
e_n=(0,...,0,1)
\end{aligned}$$

Напомним также тот факт, что любой вектор разложим по базису:
$$
\forall(x\in\R^n)\exists(\alpha_1,...,\alpha_n\in\R)[x=\alpha_1 e_1+...+\alpha_n e_n]
$$

Примеры пространств:

$\R^1=\R$

$\R^2$ -- точки плоскости.

$\R^3$ -- точки пространства.

