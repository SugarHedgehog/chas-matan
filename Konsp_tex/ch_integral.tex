\section{Неопределенный интеграл}
\subsection{Первообразная и неопределенный интеграл}
Основной задачей дифференциального исчисления является нахождение производной функции. Интегральное же исчисление решает обратную задачу -- находит функцию по её производной. Например, если дан пройденный путь в каждый момент времени (зависимость пройденного пути от времени), а нужно найти скорость в каждый момент времени -- это задача дифференциального исчисления; если дана скорость в каждый момент времени, а нужно найти путь -- это задача интегрального.

Заметим, что интегрирование, в отличие от дифференцирования функции, является неоднозначной операцией.

\opred
Функция $F(x)$ называется первообразной функции $f(x)$ на некотором промежутке $X \subset \R$, если $F$ дифференцируема на этом промежутке и 

$$
\forall(x\in X)[F'(x)=f(x)]
$$

\paragraph{Пример.}
Пусть $f(x)=\sin 3x$. Тогда одна из первообразных $F(x)=\frac{-\cos 3x}{3}$.

\subsubsection{Свойство 1.}
Если $F(x)$ -- первообразная функции $f(x)$, то $\forall(C\in\R)[F(x)+C$ -- также первообразная $f(x)]$.

\dokvo
$F'(x)=f(x)$

$(F(x)+C)'=F'(x)+C'=F'(x)+0=F'(x)=f(x)$

\dokno

\subsubsection{Свойство 2.}
Любые две первообразные $F_1(x)$ и $F_2(x)$ функции $f(x)$ отличаются на постоянную.

\dokvo

По определению $F_1'(x)=f(x)$, $F_2'(x)=f(x)$.
Докажем, что $F_1(x)-F_2(x)=const$.
Пусть $\phi(x)=F_1(x)-F_2(x)$.
Тогда $\phi'(x)=f(x)-f(x)=0$.
Значит, $\phi(x)=const$, т. е. $F_1(x)-F_2(x)=const$.

\dokno

Таким образом, по производной можно восстановить функцию с точностью до постоянного слагаемого (его называют произвольной адддитивной постоянной и обозначают $C$).

\opred
Совокупность всех первообразных функции $f$ называется неопределённым интегралом функции $f$ и обозначается $\int f(x) dx$.

$\int$ - знак интеграла. Введён в печать Яковом Бернулли в 1690 году. Значок $\int$ произошёл от латинской буквы $S$ - сокращения ``summa'', а название ``интеграл'' -- от латинского слова ``integro'' -- ``восстанавливать, объединять''

В записи $$\int f(x) dx$$
$x$, стоящая под знаком дифференциала $d$, называется переменной интегрирования;

$f(x)$ называется подынтегральной функцией;

$f(x)dx$ называется подынтегральным выражением.


Если известна одна из первообразных функции $f(x)$, то, поскольку первообразные отличаются на постоянную, известна и вся совокупность первообразных, т. е. неопределённый интеграл.

\subsubsection{Пример.}

$$\int \sin 3x dx=-\frac{1}{3} \cos 3x +C$$


\subsection{Свойства неопределенного интеграла}
\subsubsection{Свойство 1.}
Приозводная непределённого интеграла равна подынтегральной функции:

$$\left(\int f(x) dx \right)'=f(x)$$
$$d\left(\int f(x) dx \right)=f(x)d(x)$$

\subsubsection{Свойство 2.}
Интеграл от производной функции равен этой функции с точностью до постоянной:

$$\int f'(x)dx=f(x)+C$$
$$\int df(x)=f(x)+C$$

Эти два свойства вытектают из определения.

\subsubsection{Свойство 3.}
Если функции $f(x)$ и  $g(x)$ имеют первообразную на $X$, то их линейная комбинация тоже имеет первообразную на $X$ и 

$$\int(\alpha f(x) + \beta g(x))dx=\alpha \int f(x)dx+ \beta \int g(x)dx$$

Доказать это равенство несложно -- достаточно продифференцировать правую и левую часть.
Таким образом, неопределённый интеграл линеен.

\subsubsection{Замечание.}

При последовательных преобразованиях выражения, содержащего неопределённые интегралы, произвольную аддитивную постоянную $C$, возникающую при взятии интеграла, пишут только в тех частях равенства, где нет других интегралов, и опускают в тех частях, где интегралы есть.

\subsubsection{Замечание.}

Знак интеграла $\int$ никогда не используется отдельно от указания переменной интегрирования, наример, $dx$.

Сформулируем также следующую теорему, которая будет доказана позже:

\subsubsection{Теорема}

Если функция непрерывна на промежутке, то она интегрируема на этом промежутке.

\subsection{Таблица интегралов}
Все приведённые равенства устанавливаются дифференцировнаием правой части и верны на общей области определения правой и левой частей.

Формулы, являющиеся следствием таблицы производных:

\newcounter{N1} % для создания списков, маркированных со стилями, нужен счётчик
\begin{list}{\arabic{N1}.}{\usecounter{N1}}

\item
$$
\int x^\alpha dx= \frac{x^{\alpha+1}}{\alpha+1}+C, \alpha \neq -1
$$

\item
$$
\int \frac{dx}{x}= \ln|x|+C, x\neq 0
$$

\item
$$
\int a^x dx= \frac{a^x}{\ln a}+C
$$

В частности,

$$
\int e^x dx= e^x +C
$$

\item
$$
\int \sin x dx= -\cos x+C
$$

\item
$$
\int \cos x dx= \sin x+C
$$

\item
$$
\int \frac{dx}{\cos^2 x}= \tg x +C
$$

\item
$$
\int \frac{dx}{\sin^2 x}= -\ctg x +C
$$

\item
$$
\int \frac{dx}{\sqrt{1-x^2}}= \arcsin x+C
$$

Обобщение:

$$
\int \frac{dx}{\sqrt{a^2-x^2}}= \arcsin \frac{x}{a}+C
$$

\item
$$
\int \frac{dx}{1+x^2}= \arctg x+C
$$

Обобщение:

$$
\int \frac{dx}{a^2+x^2}= \frac{1}{a} \arctg \frac{x}{a}+C
$$


\item

``Логарифм длинный''
$$
\int \frac{dx}{\sqrt{x^2 \pm 1}}= \ln|x+\sqrt{x^2 \pm 1}|+C
$$

Обобщение:

$$
\int \frac{dx}{\sqrt{x^2 \pm a^2}}= \ln|x+\sqrt{x^2 \pm a^2}|+C
$$

\item

``Логарифм высокий''

$$
\int \frac{dx}{1-x^2}= \frac{1}{2} \ln \left| \frac{1+x}{1-x}\right|+C
$$

Обобщение:
$$
\int \frac{dx}{a^2-x^2}= \frac{1}{2a} \ln \left| \frac{a+x}{a-x}\right|+C
$$

\end{list}

Напомним теперь читателю определение гиперболических функций.
Вопрос об их интегрировании целесообразно рассмотреть ввиду того, что при интегрировании других функций часто используется т. наз. гиперболическая замена.

\opred

Гиперболический синус $$\sh x=\frac{e^x - e^{-x}}{2}$$

\opred

Гиперболический косинус $$\ch x=\frac{e^x + e^{-x}}{2}$$

\opred

Гиперболический тангенс $$\th x=\frac{\sh x}{\ch x}$$

\opred

Гиперболический котангенс $$\cth x=\frac{\ch x}{\sh x}$$

Продолжим таблицу интегралов:

\begin{list}{\arabic{N1}.}{\usecounter{N1}}

\item
$$
\int \sh x dx= \ch x+C
$$

\item
$$
\int \ch x dx= \sh x+C
$$

\item
$$
\int \frac{dx}{\ch^2 x}= \th x+C
$$

\item
$$
\int \frac{dx}{\sh^2 x}= -\cth x+C
$$

\end{list}

\subsubsection{Замечание}

При записи результатов интегрирования произвольные аддитивные постоянные объединяют:
$$\int(x^2 + \sin x + 2)dx=\frac{x^3}{3}-\cos x + 2x +C$$


\subsection{Интегрирование по частям}
\subsubsection{Метод.}
Пусть $u(x)$ и $v(x)$ на некотором промежутке $X$ -- диффернецируемые функции. Тогда
$$\int u(x)\cdot v'(x)dx=u(x)\cdot v(x) - \int v(x) \cdot u'(x)dx$$

Т. е., перейдя к дифференциалам функций,
$$\int udv=uv- \int vdu$$

\dokvo
Нам известна формула дифференцирования произведения:
$$(u(x)\cdot v(x))'=u'(x)v(x)+v'(x)u(x)$$
Интегрируем её:
$$u(x)\cdot v(x)'=\int u'(x)v(x) dx +\int v'(x)u(x) dx$$
И переносим один из интегралов в левую часть:
$$u(x)\cdot v(x) - \int v(x) \cdot u'(x)dx=\int u(x)\cdot v'(x)dx$$

\dokno

\subsubsection{Замечание 1.}

При использовании формулы интегрирования по частям подынтегральную функцию нужно представить в виде произведения одной функции на дифференциал другой.
Делают так, чтобы интеграл $\int vdu$ оказался проще, чем интеграл $\int udv$.
Иногда формулу интегрирования по частям приходится применять несколько раз.

\subsubsection{Замечание 2.}

Функция $v$ по $dv$ восстанавливается, вообще говоря, неоднозначно, с точностью до постоянного слагаемого. Его можно считать равным нулю.

\dokvo
Пусть по дифференциалу $dv$ нашлись функции $v_0$ и $v_0+C$. На левую часть, т. е. $\int udv$, $C$ не влияет, т. к. $d(v_0)=d(v_0+C)$. Рассмотрим правую часть:
$$
u\cdot (v_0+C) - \int (v_0+C)du=
uv_0+uC-\int v_0 du - C\int du=$$$$=
uv_0+uC-\int v_0 du - Cu=
uv_0-\int v_0 du
$$
\dokno

\subsubsection{Замечание 3.}
Интегрирование по частям особенно эффективно при интегрировании, если:

а) $u(x)=P_n(x)$, т. е. многочлен от $x$, а $v'(x) \in \{e^x,\sin x,\cos x\}$

б) $u(x) \in \{\ln x, \arctg x \}$, $v'(x)=P_n(x)$

\subsubsection{Пример.}

$$\int x^2 e^x dx = \int \left(\frac{x^3}{3}\right)'e^x dx=$$ $$=
\left<\begin{array}{c|c}
u=x^2 & du=2xdx \\
dv=e^x dx & v=e^x
\end{array}\right>=$$ $$=
x^2 e^x - 2\int e^x \cdot x dx=$$ $$=
\left<\begin{array}{c|c}
u=x & du=dx \\
dv=e^x dx & v=e^x
\end{array}\right>=$$ $$=
x^2 e^x-2\left(e^x \cdot x - \int e^x dx \right)=x^2 e^x - 2 x e^x + 2 e^x +C
$$



\subsection{Замена переменной}
\subsubsection{Теорема.}
Пусть $F$ -- первообразная для $f$ -- непрерывной функции на промежутке $T$, т. е.
$$\int f(t)dt=F(t)+C$$
и на промежутке $X$ задано $\varphi:X\to T$ -- непрерывное дифференцируемое отображение.

Тогда на промежутке $X$
$$\int f(\varphi(x))\cdot \varphi '(x) dx=F(\varphi(x))+C$$
Т. е.
$$\int f(\varphi(x))\cdot d\varphi(x)=F(\varphi(x))+C$$

\dokvo
$$(F(\varphi(x))+C)'=f(\varphi(x))\cdot \varphi'(x)$$
\dokno

\subsubsection{Пример.}

$$\int x e^{x^2} dx = 
\left<\begin{array}{c}
t=x^2 \\
dt=2xdx
\end{array}\right>= %$$ $$=
\frac{1}{2}\int e^t dt = \frac{1}{2} e^t +C = \frac{1}{2}e^{x^2}+C
$$

\subsubsection{Пример.}

$$\int \cos^2 x \sin x dx = 
\left<\begin{array}{c}
t=\sin x \\
dt=-\cos x
\end{array}\right>=$$ $$=
-\int t^2 dt = -\frac{t^3}{3}+C = -\frac{\cos^3 x}{3} +C
$$

\subsubsection{Следствие.}
Если $F'(x)=f(x)$ и $\{a;b\}\in\R$, то
$$\int f(ax+b)dx=\frac{1}{a}F(ax+b)+C$$

\subsubsection{Пример.}

$$\int\cos(7x+3)dx=-\frac{1}{7}\sin(7x+3)+C$$

\subsubsection{Замечание 1.}
Полезно помнить следующие интегралы:

$$\int \frac{g'(x)}{g(x)} dx = 
\left<\begin{array}{c}
t=g(x) \\
dt=g'(x)dx
\end{array}\right>=$$ $$=
\int \frac{dt}{t}=\ln|g(x)|+C
$$

$$\int \frac{g'(x)}{\sqrt{g(x)}} dx = 
\left<\begin{array}{c}
t=g(x) \\
dt=g'(x)dx
\end{array}\right>=$$ $$=
\int \frac{dt}{\sqrt{t}}=2\sqrt{g(x)}+C
$$

\subsubsection{Замечание 2.}
Замену переменной под знаком неопределённого интеграла часто производят иначе: вместо того, чтобы принимать за новую переменную $t$ некоторую функцию $f(x)$, рассматривают $x$ как дифференцируемую функцию от $z$, т. е. $x=\psi(z)$. Тогда
$$\int f(x)dx=\int f(\psi(x))\psi'(z)dz$$
Однако при применении этого метода нужно убедиться, что существует обратная функция $\psi^{-1}(x)=z$, позволяющая вернуться от $z$ к исходной переменной $x$.

\subsubsection{Пример.}
$$\int \sqrt{1-x^2}dx =
\left<\begin{array}{c}
t=\sin z \\
|x|\leq 1; |z|\leq \frac{\pi}{2}
dx=\cos z dz
\end{array}\right>=$$ $$=
\int\sqrt{1-\sin^2 z} \cos z dz=$$$$=
\int\frac{1+\cos 2z}{2}=\frac{1}{2}\int dz +\frac{1}{2}\cdot \frac{1}{2} \sin 2z +C=$$$$=
\frac{\arcsin x}{2}+\frac{\sin(2\arcsin x)}{4}+C=\frac{\arcsin x + x\sqrt{1-x^2}}{2}+C
$$

\subsection{Интегрирование рациональных функций}
\subsection{Интегралы от тригонометрических выражений}
Рассмотрим интегралы вида 
$$\int R(\sin x,\cos x)dx$$

\subsubsection{Универсальная тригонометрическая подстановка.}
Пусть $t=\tg\frac{x}{2}$, тогда 
$$x=2\arctg t, dx=\frac{2dt}{1+t^2}$$
$$\sin x=\frac{2t}{1+t^2}$$
$$\cos x=\frac{1-t^2}{1+t^2}$$

Таким образом, эта подстановка (известная читателю ещё из курса средней школы, где она применялась для решения тригонометрических уравнений) позволяет гарантированно рационализировать искомый интеграл:

$$\int R(\sin x,\cos x)dx=\int R(\frac{2t}{1+t^2},\frac{1-t^2}{1+t^2})\cdot \frac{2dt}{1+t^2}=\int R_1(t)dt$$

\subsubsection{Пример.}

$$\int \frac{dt}{3+\cos x}=
\left<\begin{array}{c}
t=\tg\frac{x}{2}
\end{array}\right>=
\int\left(\frac{2dt}{1+t^2}\cdot\frac{1}{3+\frac{1-t^2}{1+t^2}}\right)=$$
$$=2\int\frac{dt}{3t^2+3+1-t^2}=\int\frac{dt}{t^2+2}=
\frac{1}{\sqrt{2}}\arctg\frac{t}{\sqrt{2}}+C=$$$$=
\frac{1}{\sqrt{2}}\arctg\left(\frac{1}{\sqrt{2}}\cdot \tg\frac{x}{2}\right)+C$$

Однако неудобство этого метода заключается в том, что степень знаменателя рациональной функции $R_1$ получается сравнительно большой, поэтому применяются и другие, менее универсальные приёмы.

\subsubsection{Приём.}
$$\int R(\sin x)\cdot \cos x dx \begin{zamena}t=\sin x\\dt=\cos x dx\end{zamena}\int R(t)dt$$
Для $\int R(\cos x) \cdot \sin x dx$ - аналогично.

\subsubsection{Приём.}
$$\int R(\sin^2 x, \cos^2 x) dx \begin{zamena}t=\tg x\\x=\arctg t, dx=\frac{dt}{1+t^2}\\cos^2 x=\frac{1}{1+t^2}\\cos^2 x=\frac{t^2}{1+t^2}\end{zamena}\int R_1(t^2)dt$$

\subsubsection{Приём.}
$$\int R(\sin^2 x, \cos^2 x) dx =\int R(\frac{1-\cos 2x}{2},\frac{1+\cos 2x}{2})dx=\int R_1(\cos 2x)dx$$

\subsubsection{Замечание.}
Кроме того, при интегрировании произведения тригонометрических функций от линейной функции от $x$ удобно применить представление произведения тригонометрических функций в виде полусуммы.

\subsubsection{Пример.}
$$\int \sin (2x+3) \cdot \cos(3x+2) dx=\frac{1}{2}\int(\sin(2x+3+3x+2) \cdot \sin(2x+3-(3x+2))dx=...$$




\subsection{Интегралы от иррациональных выражений}
Рассмотрим интегралы вида
$$\int R(x,y(x))dx$$
Чтобы свести такой интеграл к интегралу от рациональной функции, нужно найти подстановку $x=x(t)$ такую, чтобы $x(t)$ (а, значит, и $x'(t)$) и $y(x(t))$ были рациональными функциями от $t$:

$$\int R(x,y(x))dx=\int R(x(t),y(x(t)))x'(t)dt=\int R_1(t)dt$$

Рассмотрим сначала случай $y=\sqrt[n]{\frac{\alpha x+ \beta}{\gamma x+ \delta}}$. Пусть
$$t^n=\frac{\alpha x+ \beta}{\gamma x+ \delta}$$
Тогда $$(\gamma x + \delta)t^n=\alpha x + \beta$$
Отсюда $$ (\gamma t^n - \alpha)x = \beta - \delta t^n$$
Т. е. $$x = \frac{\beta - \delta t^n}{\gamma t^n - \alpha}=R_x(t)$$

Интеграл рационализирован.

\subsubsection{Пример.}

$$\int\frac{\sqrt{x}}{1+x}dx\begin{zamena}t=\sqrt{x}\\x=t^2\\dx=2tdt\end{zamena}2\int\frac{t^2 dt}{1+t^2}=$$$$=
2\int\left(1-\frac{1}{1+t^2}\right)dt=2t-2\arctg t+C=2\sqrt{x}-2 \arctg\sqrt{x}+C$$

Обобщим теперь наш опыт на случай интеграла

$$\int R\left( x, \left(\frac{\alpha x + \beta}{\gamma x + \delta}\right)^{r_1},...,\left(\frac{\alpha x + \beta}{\gamma x + \delta}\right)^{r_k}\right)dx,$$

где $r_1,...,r_n \in \Q$. Тогда $r_i = \frac{p_i}{q_i}$. Пусть $m$ - наименьшее общее кратное чисел $q_1,...,q_n$. Введём замену
$$t^m=\frac{\alpha x+ \beta}{\gamma x+ \delta}$$
Легко видеть, что в этом случае интеграл рационализируется.

\subsubsection{Пример.}

$$\int\frac{dx}{\sqrt{x}+\sqrt[3]{x}}=\int\frac{dx}{x^\frac{3}{6}+x^\frac{2}{6}}
\begin{zamena}x=t^6,~~t=x^\frac{1}{6}\\dx=6t^5 dt\end{zamena}6\int\frac{t^5 dt}{t^3+t^2}=$$$$=
6\int\frac{t^3}{t+1}dt=6\left(\int\frac{t^3+1}{t+1}dt-\int\frac{1}{t+1}dt\right)=$$$$=
6\left(\int(t^2-t+1)dt-\ln|t+1|\right)=2t^3-3t^2+6t-\ln|t+1|+C=$$$$=
2\sqrt{x}-3\sqrt[3]{x}+6\sqrt[6]{x}-\ln|\sqrt[6]{x}+1|+C
$$



\subsection{Подстановки Эйлера}
Перейдём теперь к вопросу об интегрировании функции 

\begin{equation}\label{integral_podst_Eilera}
\int R(x,\sqrt{ax^2+bx+c})dx
\end{equation}

Случай, когда $a=0$, фактически рассмотрен нами ранее и потому интереса не представляет.
Введём стандартное обозначение дискриминанта: $D=b^2-4ac$.
Рассмотрим теперь случаи, когда $D=0$.
Если $a<0$, то функция определена лишь в одной точке, и говорить об интеграле нет смысла (т. к. интеграл определяется на промежутке).
Если же $a>0$, то корень извлекается, и задача сводится к взятию интеграла вида $\int R (x,|x-x_0|)dx$, что не представляет особой сложности.

Пусть теперь $a>0$, $D>0$.
Тогда
\begin{equation}\label{vydel_poln_kvadr}
ax^2+bx+c=a\left(x+\frac{b}{2a}\right)^2+\left(c-\frac{b^2}{4a}\right)=a\left(x+\frac{b}{2a}\right)^2-\frac{D}{4a}
\end{equation}

Положим теперь 
\begin{equation}\label{zamena_pered_podst_Eilera}
\tau = \sqrt{a}\left(x+\frac{b}{2a}\right),
\alpha^2=\frac{D}{4a}, ~ 
\text{тогда} ~ 
x=\frac{\tau}{\sqrt{a}}-\frac{b}{2a}, ~ 
dx=\frac{1}{\sqrt{a}}d\tau
\end{equation}
Выражение (\ref{vydel_poln_kvadr}) примет вид $\tau^2-\alpha^2$, а исследуемый интеграл (\ref{integral_podst_Eilera}) преобразуется в:

$$
\int R\left(\frac{\tau}{\sqrt{a}}-\frac{b}{2a},\sqrt{\tau^2-\alpha^2}\right)\cdot\frac{1}{\sqrt{a}}d\tau
$$

Теперь рассмотрим случай, когда $a>0$, $D<0$. Замена будет аналогична замене (\ref{zamena_pered_podst_Eilera}), за исключением того, что $\alpha^2=-\frac{D}{4a}$. Интеграл (\ref{integral_podst_Eilera}) примет вид

$$
\int R\left(\frac{\tau}{\sqrt{a}}-\frac{b}{2a},\sqrt{\tau^2+\alpha^2}\right)\cdot\frac{1}{\sqrt{a}}d\tau
$$

В случае, если $a<0$, $D>0$, замена снова будет аналогична (\ref{zamena_pered_podst_Eilera}), за исключением того, что $\tau = \sqrt{a}\left(x+\frac{b}{2a}\right)$. Интеграл (\ref{integral_podst_Eilera}) примет вид

$$
\int R\left(\frac{\tau}{\sqrt{a}}-\frac{b}{2a},\sqrt{\tau^2-\alpha^2}\right)\cdot\frac{1}{\sqrt{-a}}d\tau
$$

И, наконец, если $D<0$, $a<0$, то подынтегральная функция не имеет смысла.

Таким образом, задача отыскания интеграла (\ref{integral_podst_Eilera}) свелась к отысканию следующих интегралов (здесь $t=\frac{\tau}{\alpha}$, постоянные множители вынесены за знак интеграла):

$$
\int\hat R(t,\sqrt{1-t^2})dt
$$$$
\int\hat R(t,\sqrt{1+t^2})dt
$$$$
\int\hat R(t,\sqrt{t^2-1})dt
$$

Проницательный читатель заметит, что в первых двух случаях можно применить гиперболическую замену, а в третьем - тригонометрическую, но существуют подстановки, позволяющие свести взятие этих интегралов непосредственно к интегрированию рациональной функции. Эти подстановки названы в честь первооткрывателя -- Эйлера.

Для взятия интеграла вида
$$\int\hat R(t,\sqrt{t^2-1})dt$$
применяют замену
$$\sqrt{t^2-1}=u(t\pm 1)$$
или
$$\sqrt{t^2-1}=\pm(t-u)$$

Для взятия интеграла вида
$$\int\hat R(t,\sqrt{t^2+1})dt$$
применяют замену
$$\sqrt{t^2+1}=tu\pm 1$$
или
$$\sqrt{t^2+1}=\pm(t-u)$$

Для взятия интеграла вида
$$\int\hat R(t,\sqrt{1-t^2})dt$$
применяют замену
$$\sqrt{1-t^2}=u(1\pm t)$$
или
$$\sqrt{1-t^2}=tu\pm1$$

Поясним на примере последней, как они работают:
$$\sqrt{1-t^2}=tu-1$$
$$1-t^2=t^2 u^2 -2tu+1$$
$$2tu=(1+u^2)t^2$$
$$2u=(1+u^2)t$$
$$t=\frac{2u}{(1+u^2)}$$
$$\sqrt{1-t^2}=tu-1=\frac{2u^2}{(1+u^2)}$$

Дифференциал $u'(t)du$ также будет рациональной функцией; выписать его предоставляем читателю. Таким образом, интеграл рационализировался.

\subsection{Интегралы от дифференциальных биномов}
\opred
Дифференциальным биномом (или биномиальным дифференциалом) называется выражение вида
$$x^m(a+bx^n)^p dx$$

Рассмотрим вопрос об интегрировании дифференциального бинома, т.~е. об отыскании интеграла вида
\begin{equation}\label{integral_ot_diff_binoma}
\int x^m(a+bx^n)^p dx
\end{equation}
Сделаем замену $t=x^n$, тогда $x=t^{\frac{1}{n}}$, $dx=\frac{1}{n}t^{\frac{1}{n}-1}$, и 
$$
\int x^m(a+bx^n)^p dx=
\int t^{\frac{m}{n}}(a+bt)^p\cdot \frac{1}{n}\cdot t^{\frac{1}{n}-1}dt=
\frac{1}{n}\int t^{\frac{m+1}{n}-1}(a+bt)^p dt
$$
Положив $q=\frac{m+1}{n}-1$, интеграл (\ref{integral_ot_diff_binoma}) мы представим в виде
$$\varphi(p,q)=\frac{1}{n}\int t^q(a+bt)^p$$

\subsubsection{Теорема.}
Если хотя бы одно из чисел $p$, $q$ или $p+q$ является целым, то интеграл $\varphi(p,q)$ рационализируется.

\dokvo

1. Пусть $p\in\Z$. Тогда $\varphi(p,q)=\int R(t,t^q)dt$. Интегралы такого вида уже были рассмотрены нами ранее.

2. Пусть $q\in\Z$. Тогда $\varphi(p,q)=\int R((a+bt)^p,t)dt$. Интегралы такого вида уже были рассмотрены нами ранее.

3. Пусть, наконец, $p+q\in\Z$. Тогда $\varphi(p,q)=\int R\left(\left(\frac{a+bt}{t}\right)^p,t^{p+q}\right)dt$. И снова получили интеграл уже изученного вида.

\dokno

\subsubsection{Пример.}
$$\int x^2 \sqrt{x}(1-x^2)dx\begin{zamena}m=\frac{5}{2},~n=2,~p=1\in\Z\\x=t^2,~dx=2tdt\end{zamena}$$$$=
\int t^5(1-t^4)2tdt=2\int (t^6-t^{10}) dt=...$$

Завершить вычисление интеграла предоставляем читателю самостоятельно.

\subsubsection{Замечание.}
Великий русский математик Пафнутий Львович Чебышев доказал, что в случае, когда условие  доказанной теоремы не выполнено, интеграл не представим через элементарные функции, т. е. является неберущимся. О неберущихся интегралах читатель узнает буквально на следующей странице.
 
\subsection{Неберущиеся интегралы}
\opred

Интеграл, не выражающийся через элементарные функции, называется неберущимся.

\subsubsection{Примеры.}
$$\int x^m(a+bx^n)^p dx$$
если $q=\frac{m+1}{n}, p \notin \Z, q \notin\Z, p+q\notin \Z$

$$\int \frac{e^x}{x^n}dx$$

$$\int \frac{\sin x}{x^n}dx$$

$$\int \frac{\cos x}{x^n}dx$$

$$\int \frac{e^{-x^2}}{x^n}dx$$

Часто в приложениях возникает интеграл вида $\int R(x,\sqrt{P_n(x)})dx$. Случаи, когда $n=1$ или $n=2$, исследованы нами ранее. В случае $n \geq 3$, вообще говоря, такой интеграл может быть неберущимся.

С помощью неберущихся интегралов определяются некоторые новые классы трансцендентных функций. Например, эллиптическими интегралами I, II и III рода называются соответственно:
$$\int \frac{dx}{\sqrt{(1-x^2)(1-k^2 x^2)}}$$
$$\int \frac{x^2 dx}{\sqrt{(1-x^2)(1-k^2 x^2)}}$$
$$\int \frac{dx}{(1+hx^2)\sqrt{(1-x^2)(1-k^2 x^2)}}$$

Здесь $0<k<1$.



\section{Определенный интеграл Римана}
\subsection{Задача о вычислении площади криволинейной трапеции}
К понятию определённого интеграла привела задача о площади криволинейной трапеции.

\opred
Криволинейной трапецией называется фигура на координатной плоскости, ограниченная осью абсцисс, некоторыми прямыми $x=a$ и $x=b$ ($a<b$) и графиком некоторой непрерывной и неотрицательной на $[a;b]$ функции $f$.

\opred 
Разбиением $T$ отрезка $[a;b]$ называется совокупность точек ${x_0,...,x_n}$, таких, что 
$$a=x_0<x_1<x_2<...<x_{n-1}<x_n=b$$

\opred
Если разбиение $T$ состоит из точек ${x_0,...,x_n}$, то эти точки называются точками деления разбиения $T$.

\opred
Отрезки $[x_{j-1};x_j]$, где $j=1...n$, называются подотрезками разбиения $T$ и обозначаются $\Delta_j$, а их длины обозначаются $\Delta x_j=x_j-x_{j-1}$.

\opred
Наибольшая из длин подотрезков разбиения $T$ называется диаметром разбиения $T$ и обозначается $d(T)=\max\limits_j \Delta_j$

\opred
Если на каждом подотрезке $\Delta_j$ разбиения $T$ выбрать произвольную точку $\xi_j$, то разбиение $T$ называется разбиением с отмеченными точками и обозначается $(T,\xi)$.

Чтобы найти площадь $S_T$ криволинейной трапеции, на отрезке $[a;b]$ строят некоторое разбиение $(T,\xi)$ и затем суммируют площади прямоугольников с шириной $\Delta_j$ и высотой $f(\xi_j)$:
$$S_T \approx \sum_{j=1}^{n}f(\xi_j)\cdot \Delta x_j$$
Здесь $n$ - количество подотрезков разбиения $T$.

Интуитивно ясно, что чем меньше диаметр разбиения, тем лучше приближена площадь трапеции. Строгое математическое доказательство этому будет дано ниже.


\subsection{Определение определенного интеграла}
\subsection{Необходимое условие интегрируемости функции}
\subsection{Критерий Коши интегрируемости функции}
\subsection{Необходимое и достаточное условие интегрируемости}
\subsection{Интегралы Дарбу}
\subsection{Признак Дарбу существования интеграла}
\subsection{Свойства интеграла Римана}
\subsection{Первая теорема о среднем}
\subsection{Вторая теорема о среднем} 
\subsection{Формула Ньютона-Лейбница}
\subsection{Формула интегрирования по частям для определенного интеграла}
\subsection{Замена переменной в определенном интеграле}
\subsection{Понятия о приближенных методах вычисления определенных интегралов}
...

\section{Приложения определенного интеграла}
\subsection{Аддитивная функция промежутка}
\subsection{Длина параметризованной кривой} 
\subsection{Площадь поверхности вращения}
\subsection{Площадь фигуры}
\subsection{Объем тела вращения}
\subsection{Понятие о несобственных интегралах}

