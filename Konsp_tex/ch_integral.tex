\section{Неопределенный интеграл}
\subsection{Первообразная и неопределенный интеграл}
\subsection{Свойства неопределенного интеграла}
\subsection{Таблица интегралов}
\subsection{Интегрирование по частям}
\subsection{Замена переменной}
\subsection{Интегрирование рациональных функций}
\subsection{Интегралы от тригонометрических выражений}
\subsection{Подстановки Эйлера}
\subsection{Интегралы от иррациональных выражений}
\subsection{Интегралы от дифференциальных биномов} 
\subsection{Неберущиеся интегралы}
...

\section{Определенный интеграл Римана}
\subsection{Задача о вычислении площади криволинейной трапеции}
\subsection{Определение определенного интеграла}
\subsection{Необходимое условие интегрируемости функции}
\subsection{Критерий Коши интегрируемости функции}
\subsection{Необходимое и достаточное условие интегрируемости}
\subsection{Интегралы Дарбу}
\subsection{Признак Дарбу существования интеграла}
\subsection{Свойства интеграла Римана}
\subsection{Первая теорема о среднем}
\subsection{Вторая теорема о среднем} 
\subsection{Формула Ньютона-Лейбница}
\subsection{Формула интегрирования по частям для определенного интеграла}
\subsection{Замена переменной в определенном интеграле}
\subsection{Понятия о приближенных методах вычисления определенных интегралов}
...

\section{Приложения определенного интеграла}
\subsection{Аддитивная функция промежутка}
\subsection{Длина параметризованной кривой} 
\subsection{Площадь поверхности вращения}
\subsection{Площадь фигуры}
\subsection{Объем тела вращения}
\subsection{Понятие о несобственных интегралах}

