\begin{teorema}
Для того, чтобы $f \in R[a;b]$ НиД чтобы $\underline{\overline{s}}=\overline{\overline{s}}$
\end{teorema}

\dokvo
\subsubsection{Необходимость}
дано: $f\in R[a;b]$ доказать $\underline{\overline{s}}=\overline{\overline{s}}$
\\
$$f\in R[a;b] \Rightarrow \exists(\overline{s}=\int_{a}^{b}f(x)dx) \Rightarrow \forall(\epsilon>0)\exists(\sigma>0)\forall(T)[d(T)<\sigma \Rightarrow \overline{s}-\epsilon<S(f,(T,\varphi))<\overline{s}+\epsilon]$$ 
По свойству №2 сумм Дарбу:
$$
\overline{S}(f,T)=\sup_{T}S(f,(T,\varphi)); \underline{S}(f,T)=\inf_{T}S(f,(t,\varphi))
$$
Поэтому из того, что 
$$
[\overline{s}-\epsilon<S(f,(T,\varphi))<\overline{S}+\epsilon] \Rightarrow \overline{S}-\epsilon \leq \underline{S}(f,T) \leq S(f,(T,\varphi)) \leq \overline{S}(f,T) \leq \overline{s}+\epsilon
$$
По определению: $\overline{\underline{s}}=\sup \underline{S}(f,T); \overline{\overline{s}}=\inf \overline{S}(f,T)$
Значит:
$
\underline{I}=\sup \underline{S}=\int_{a}^{b}f(x)dx
$
$
\overline{I}=\inf \overline{S}=\int_{a}^{b}f(x)dx
$
\dokno

\subsubsection{Достаточность}
дано:$\tilde{\underline{s}}=\overline{\tilde{s}}$
доказать: $f\in R[a;b]$
Пусть $\tilde{\underline{s}}=\overline{\tilde{s}}=\tilde{s}$
Докажем, что $\tilde{s}=\int_{a}^{b}f(x)dx$
Теорема Дарбу: $\lim_{d(T) \to 0} \underline{S}(f,T)=\underline{I}=I$
\\
$\lim_{d(T) \to 0} \overline{S}(f,T)=\overline{I}=I$
$$
\forall(\epsilon>0)\exists(\sigma_1>0)\forall(T)[d(T)<\sigma_1|\Rightarrow \overline{s}-\epsilon < \underline{S}(f,T)<\overline{s}+\epsilon]
$$
$$
\forall(\epsilon>0)\exists(\sigma_2>0)\forall(T)[d(T)<\sigma_2|\Rightarrow \overline{s}-\epsilon < \overline{S}(f,T)<\overline{s}+\epsilon]
$$
Пусть $\sigma=min(\sigma_1 и \sigma_2).$ Тогда:
$$
\forall(T:d(T)<\sigma)[(\overline{s}-\epsilon<\underline{S}(f,T)<\overline{s}+\epsilon) \wedge (\overline{s}-\epsilon<\overline{S}(f,T)<\overline{s}+\epsilon)]
$$
По свойству №1 сумм Дабу:
$$
\overline{s}-\epsilon < \underline{S}(f,T)\leq S(f,(T,\varphi))\leq \overline{S}(f,T) < \overline{s}+\epsilon|\Rightarrow
$$
$$
\Rightarrow \forall(\epsilon>0)\exists(\sigma>0)\forall(T:d(T)<\sigma)[
\overline{s}-\epsilon < S(f,(T,\varphi)) < \overline{s} + \epsilon]\Rightarrow
$$
$$
\Rightarrow [|S(f,(T,\varphi))-\overline{s}| < \epsilon] \Rightarrow
 $$
$$
 \Rightarrow f\in R[a;b]
$$
\dokno

\subsubsection{Следствие}
$$
f\in R[a;b]\Leftrightarrow \forall(\epsilon>0)\exists(\sigma>0)\forall(T)[d(T)<\sigma \Rightarrow \sum_{i=1}^{n(T)}\omega(f_i,\vartriangle_i)\vartriangle x_i < \epsilon]
$$
или используя определение предела, переформулируем:
$$
f \in R[a;b] \Leftrightarrow \lim_{d(T) \to 0} \sum_{i=1}^{n(T)}\omega(f,\vartriangle_i)\vartriangle x_i = 0
$$

