\documentclass[a4paper,14pt]{report} %размер бумаги устанавливаем А4, шрифт 12пунктов
\usepackage[T2A]{fontenc}
\usepackage[utf8]{inputenc}
\usepackage[english,russian]{babel} %используем русский и английский языки с переносами
\usepackage{amssymb,amsfonts,amsmath,mathtext,cite,enumerate,float} %подключаем нужные пакеты расширений
\usepackage[pdftex,unicode]{hyperref}
\usepackage{indentfirst} % включить отступ у первого абзаца
\usepackage[dvips]{graphicx} %хотим вставлять рисунки?
\graphicspath{{illustr/}}%путь к рисункам

\makeatletter
\renewcommand{\@biblabel}[1]{#1.} % Заменяем библиографию с квадратных скобок на точку:
\makeatother %Смысл этих трёх строчек мне непонятен, но поверим "Запискам дебианщика"

\usepackage{geometry} % Меняем поля страницы. 
\geometry{left=1cm}% левое поле
\geometry{right=1cm}% правое поле
\geometry{top=1cm}% верхнее поле
\geometry{bottom=2cm}% нижнее поле

\renewcommand{\theenumi}{\arabic{enumi}}% Меняем везде перечисления на цифра.цифра
\renewcommand{\labelenumi}{\arabic{enumi}}% Меняем везде перечисления на цифра.цифра
\renewcommand{\theenumii}{.\arabic{enumii}}% Меняем везде перечисления на цифра.цифра
\renewcommand{\labelenumii}{\arabic{enumi}.\arabic{enumii}.}% Меняем везде перечисления на цифра.цифра
\renewcommand{\theenumiii}{.\arabic{enumiii}}% Меняем везде перечисления на цифра.цифра
\renewcommand{\labelenumiii}{\arabic{enumi}.\arabic{enumii}.\arabic{enumiii}.}% Меняем везде перечисления на цифра.цифра
\LARGE

\begin{document}
\newcommand{\pp}{Предположим противное}
\tableofcontents % это оглавление, которое генерируется автоматически

\LARGE

\chapter{Введение в анализ}
\section{Элементарные сведения из логики и теории множеств}
\subsection{Высказывания, предикаты связки}
\subsection{Кванторы}
\subsection{Множества, равенство двух множеств, подмножества}
\subsection{Простейшие операции над множествами}
\subsection{Принцип двойственности}
\subsection{Понятие счетного множества}

\section{Теория вещественных чисел}
\subsection{Множество рациональных чисел и его свойства}
\subsection{Вещественные числа, основные свойства вещественных чисел}
\subsection{Промежутки и их виды}
\subsection{Основные леммы теории вещественных чисел}

\section{Ограниченное множество, границы}
\subsection{Границы множества}
\subsection{Существование точной верхней границы у ограниченного сверху множества}
\subsection{Сечения в множестве рациональных чисел}
\subsection{Свойства $sup$ и $inf$}
\subsection{Отделимость множеств, лемма о системе вложенных отрезков}
\subsection{Лемма о последовательности стягивающихся отрезков}

\section{Отображения, функции}
\subsection{Отображения, виды отображений и т.д}
\subsection{Вещественные функции}

\section{Предел последовательности}
\subsection{Последовательность элементов множества, числовая последовательность,определия предела числовой последовательномти и бесконечно малой последовательности}
\subsection{Единственность предела последовательности}
\subsection{Подпоследовательности, связь пределов последовательности и подпоследовательности}
\subsection{Лемма о двух милиционерах}
\subsection{Основные теоремы о пределах последовательности}
\subsection{Понятие бесконечно большой последовательности}
\subsection{Монотонные последовательности, критерий существования предела монотонной послед}
\subsection{Существование предела последовательности $(1+1/n)^n$, число е }

\section{Понятие предельной точки числового множества, теорема Больцано-Вейерштрасса, критерий Коши}
\subsection{Предельная точка множества}
\subsection{Теорема о последовательности, сходящейся к предельной точке}
\subsection{Теорема Больцано-Вейерштрасса}
\subsection{Критерий Коши}

\section{Верхний и нижний пределы последовательности}
\subsection{Понятие расширенной числовой прямой, понятие бесконечных пределов}
\subsection{Понятие частичных верхних и нижних пределов последовательности. Теорема о существовании у каждой последовательности ее верхнего и нижнего предела}
\subsection{Характеристические свойства верхнего и нижнего предела последовательности}
\subsection{Критерий существования предела последовательности}


\chapter{Вещественная функция вещественного аргумента}
\section{Предел вещественной функции вещественного аргумента}
\subsection{Определение предела функции по Коши, примеры}
\subsection{Определение предела функции по Гейне, примеры, эквивалентность определений}
\subsection{Обобщение понятия предела функции на расширенную числовую ось}

\section{Свойства пределов функции и функций, имеющих предел}
\subsection{Свойства, связанные с неравенствами}
\subsection{Свойства, связанные с арифметическими  операциями}

\section{Односторонние пределы функции}
\subsection{Определение односторонних пределов, связь между существованием предела и односторонних пределов функции}
\subsection{Теорема о существовании односторонних пределов у монотонной функции и ее следств}

\section{Критерий Коши, замечательные пределы, бесконечно малые функции}
\subsection{Критерий Коши существования предела функции}
\subsection{Первый замечательный предел}
\subsection{Второй замечательный предел}
\subsection{Бесконечно малые функции и их классификация}

\section{Непрерывные функции. Общие свойства}
\subsection{Понятие непрерывности функции в точке}
\subsection{Непрерывность функции на множестве}
\subsection{Понятие колебания функции на множестве и в точке. Необходимое и достаточное условие непрерывности функции в точке}
\subsection{Односторонняя непрерывность}
\subsection{Классификация точек разрыва}
\subsection{Локальные свойства непрерывных функций}
\section{Функции, непрерывные на отрезке}

\subsection{Теорема Больцано-Коши и следствия из неё}
\subsection{Первая теорема Вейерштрасса}
\subsection{Вторая теорема Вейерштрасса}
\subsection{Понятие равномерной непрерывности функции. Теорема Кантора, следствия из неё}
\subsection{Свойства монотонных функций. Теорема об обратной функции}
\subsection{Непрерывность элементарных функций}


\chapter{Основы дифференциального исчисления}
\section{Дифференциальное исчисление функции одной независимой переменной}
\subsection{Определение производной и дифференциала, связь между этими понятиями}
\subsection{Связь между понятиями дифференцируемости и непрерывности функций}
\subsection{Дифференцирование и арифметические операции}
\subsection{Теорема о производной сложной функции. Инвариантность формы первого дифференциала}
\subsection{Теорема о производной обратной функции}
\subsection{Производные основных элементарных функций. Доказательство}
\subsection{Касательная к кривой. Геометрический смысл производной и дифференциала}
\subsection{Физический смысл производной и дифференциала}
\subsection{Односторонние и бесконечные производные}
\subsection{Производные и дифференциалы высших порядков}

\section{Основные теоремы дифференциального исчисления}
\subsection{Теорема Ферма}
\subsection{Теорема Ролля}
\subsection{Теорема Лагранжа и следствия из нее}
\subsection{Теорема Коши}

\section{Формула Тейлора}
\subsection{Формула Тейлора для многочлена}
\subsection{Формула Тейлора для произвольной функции. Различные формы остаточного члена формулы Тейлора}
\subsection{Локальная формула Тейлора}
\subsection{Формула Маклорена. Разложение по формуле Маклорена некоторых элементарных функций}

\section{Правило Лопиталя}


\end{document}
