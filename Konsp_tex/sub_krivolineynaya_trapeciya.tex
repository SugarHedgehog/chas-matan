К понятию определённого интеграла привела задача о площади криволинейной трапеции.

\opred
Криволинейной трапецией называется фигура на координатной плоскости, ограниченная осью абсцисс, некоторыми прямыми $x=a$ и $x=b$ ($a<b$) и графиком некоторой непрерывной и неотрицательной на $[a;b]$ функции $f$.

\opred 
Разбиением $T$ отрезка $[a;b]$ называется совокупность точек ${x_0,...,x_n}$, таких, что 
$$a=x_0<x_1<x_2<...<x_{n-1}<x_n=b$$

В дальнейшем, говоря о разбиениях, слова ``на отрезке $[a;b]$'' мы будем почти всегда опускать, предполагая, что этот отрезок нам известен.

\opred
Если разбиение $T$ состоит из точек ${x_0,...,x_n}$, то эти точки называются точками деления разбиения $T$.

\opred
Отрезки $[x_{j-1};x_j]$, где $j=1...n$, называются подотрезками разбиения $T$ и обозначаются $\Delta_j$, а их длины обозначаются $\Delta x_j=x_j-x_{j-1}$.

\opred
Наибольшая из длин подотрезков разбиения $T$ называется диаметром разбиения $T$ и обозначается $d(T)=\max\limits_j \Delta_j$

\opred
Если на каждом подотрезке $\Delta_j$ разбиения $T$ выбрать произвольную точку $\xi_j$, то разбиение $T$ называется разбиением с отмеченными точками и обозначается $(T,\xi)$.

Чтобы найти площадь $S_T$ криволинейной трапеции, на отрезке $[a;b]$ строят некоторое разбиение $(T,\xi)$ и затем суммируют площади прямоугольников с шириной $\Delta_j$ и высотой $f(\xi_j)$:
$$S_T \approx \sum_{j=1}^{n}f(\xi_j)\cdot \Delta x_j$$
Здесь $n$ - количество подотрезков разбиения $T$.

Интуитивно ясно, что чем меньше диаметр разбиения, тем лучше приближена площадь трапеции. Строгое математическое доказательство этому будет дано ниже.

