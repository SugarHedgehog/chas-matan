Подобно тому, как мы распространяли понятие предела на случай, когда в выражении участвует бесконечность, можно распространить и понятие интеграла на бесконечные (неограниченные) криволинейные трапеции.
Такие интегралы называют несобственными.

\begin{opr}\label{opr_nesobstv_intl_1}
Пусть $y=f(x)$, $f:[a;+\infty]\to\R$, $\forall(b>a)[f\in R[a;b]]$.
Если 
\begin{equation}\label{lim_nesobstv_intl_1}
\exists \lim_{b\to\infty}\intl_a^b f(x)dx\neq \pm\infty
\end{equation}
то говорят, что интеграл
\begin{equation}\label{nesobstv_intl_1}
\intl_a^{+\infty}f(x)dx
\end{equation}
сходится и равен пределу (\ref{lim_nesobstv_intl_1}), в противном случае -- что интеграл расходится.
Интеграл по бесконечному промежутку называют несобственным интегралом первого рода.
\end{opr}

Запись
$$
\intl_a^{+\infty}f(x)dx=\infty
$$
не используют.

\begin{primer}
$$
\intl_0^{+\infty}\frac{dx}{1+x^2}=\lim_{b\to+\infty}\intl_0^{b}\frac{dx}{1+x^2}=
\lim_{b\to+\infty}(\arctg b - \arctg 0)=\frac{\pi}{2}
$$
\end{primer}

Несобственный интеграл для $-\infty$ в качестве предела вводится аналогично.

Рассмотри теперь другой тип несобственных интегралов - интегралы от неограниченных функций, называемые несобственными интегралами второго рода.

\begin{opr}
Пусть функция $f$ неограниченно возрастает при стремлении справа к точке $a$:
$$
\lim_{x\to a +}f(x)=\infty
$$
и
$$
\forall(\epsilon>0)[f\in R[a+\epsilon;b]]
$$
то полагают
$$
\intl_a^bf(x)dx=\lim_{\epsilon\to 0+}\intl_{a+\epsilon}^b f(x)dx
$$
если предел в правой части равенства существует.
В противном случае говорят, что интеграл расходится.
\end{opr}
Случай для стремления слева к правой границе определяется аналогично.

\begin{opr}
Точки $\pm\infty$ и точки, в которых подынтегральная функция неограниченно возрастает, если они принадлежат промежутку интегрирования, называются особенностями интеграла.
\end{opr}
Если в интеграле несколько особенностей, то его разбивают на сумму интегралов, каждый из которых имеет не более одной особенности.

\begin{opr}
Несобственный интеграл
$$
\intl_a^b f(x) dx
$$
(первого или второго рода) называется абсолютно сходящимся, если сходится интеграл
$$
\intl_a^b |f(x)| dx
$$

\end{opr}
