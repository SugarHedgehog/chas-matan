Следующие два свойства фактически дополняют определение:

\begin{equation}
\intl_a^a f(x) dx = 0
\end{equation}

\begin{equation}
\intl_a^b f(x) dx = - \intl_b^a f(x) dx
\end{equation}

Ещё два свойства характеризуют интеграл как линейный оператор на пространстве интегрируемых функций.
Аддитивность:

\begin{multline}
\intl_a^b (f+g)(x) dx = \lim_{d(T)\to 0}S(f+g,(T,\xi))= 
\\=
\lim_{d(T)\to 0}S(f,(T,\xi))+\lim_{d(T)\to 0}S(g,(T,\xi))=\intl_a^b f(x) dx+\intl_a^b g(x) dx
\end{multline}

Однородность ($c$ -- константа):

$$
\intl_a^b (cf)(x)dx = \lim_{d(T)\to 0}S(cf,(T,\xi))=c\lim_{d(T)\to 0}S(f,(T,\xi))=c\intl_a^b f(x) dx
$$

Предостережём читателя: столь же красивой формулы для интеграла от произведения функций нет.

Введём вспомогательное определение.

\begin{opr}
Сужением разбиения $(T,\xi)$, содержащего среди точек деления $c$ и $d$, отрезка $[a;b]$ на подотрезок $[c;d]\subset[a;b]$ называется разбиение $(T_1,\xi)$, точками деления которого являются точки деления $T$, лежащие на отрезке $[c;d]$, а отмеченными точками -- соответствующие отмеченные точки разбиения $T$.
\end{opr}

Если функция интегрируема на отрезке, то она интегрируема и на любом подотрезке.

\dokvo

Пусть $f\in R[a;b]$, $[\alpha;\beta]\subset[a;b]$.
Рассмотрим те разбиения, в которые входят точки $\alpha$ и $\beta$, и положим $a=\xi_1$, $\alpha=\xi_p$, $\beta=\xi_q$, $b=\xi_n$
Тогда по необходимому и достаточному условию интегрируемости
$$
\forall(\epsilon>0)\exists(\delta>0)\forall((T,\xi):d(T)<\delta)\left[\sum_{i=1}^n \omega(f,\Delta_i)\Delta x_i < \epsilon\right]
$$
Так как колебание функции на отрезке есть величина положительная, а $1<p<q<n$ то
$$
\sum_{i=p}^q \omega(f,\Delta_i)\Delta x_i < \sum_{i=1}^n \omega(f,\Delta_i)\Delta x_i < \epsilon
$$
Пусть $T_1$ -- сужение разбиения $T$ на отрезок $[\alpha,\beta]$.
Но ${\sum\limits_{i=p}^q \omega(f,\Delta_i)\Delta x_i<\epsilon}$ -- сумма колебаний функции $f$, соответствующая разбиению $T_1$.
Значит, выполнено необходимое и достаточное условие интегрируемости для функции $f$ на отрезке $[a;b]$.

\dokno

Если функция интегрируема на отрезке, то этот отрезок можно разбить на две части, и сумма интегралов на частях будет равна интегралу на отрезке:
$$
f\in R[a;b], c\in [a;b] \Rightarrow \intl_a^b f(x)dx = \intl_a^c f(x) dx + \intl_c^b f(x) dx
$$

\dokvo

Тот факт, что интегралы на подотрезках существуют, вытекает из предыдущего свойства.

Рассмотри теперь бесконечно измельчающуюся последовательность разбиений $\{(T_n,\xi^{(n)})\}$, таких, что $c$ -- одна из точек деления.

По определению \ref{eqiv_opr_opr_intl} 
$$
\{S(f,(T,\xi))\} \to \intl_a^b f(x) dx
$$

Если мы обозначим через $T_n'$ и $T_n''$ сужения $T_n$ на $[a;c]$ и $[c;b]$ соответственно, то получим
$$
S(f,(T_n,\xi^{(n)}))=S(f,(T_n',\xi'^{(n)}))+S(f,(T_n'',\xi''^{(n)}))\to \intl_a^c f(x) dx + \intl_c^b f(x) dx
$$

\dokno

Вернёмся теперь к вопросу об интегрировании произведения функций.
Здесь имеет место лишь неконструктивное утверждение: произведение двух интегрируемых на отрезке функций интегрируемо на этом отрезке, т. е.
$$
\{f,g\}\subset R[a;b] \Rightarrow (f\cdot g)\in R[a;b]
$$

\dokvo

Так как $f$ и $g$ интегрируемы на $[a;b]$, то они ограничены на $[a;b]$.
Значит,
$$
\exists(M>0)\forall(x\in[a;b])[f(x)\leq M, g(x)\leq M]
$$
Оценим теперь колебание произведения функций $fg$ на $\Delta_i$, положив $\{x',x''\}\subset[a;b]$:
\begin{multline*}
|f(x')g(x')-f(x'')g(x'')|=
|g(x')(f(x')-f(x'')+f(x'')(g(x')-g(x''))|\leq
\\\leq
|g(x')|\cdot|f(x')-f(x'')|+|f(x'')|\cdot|g(x')-g(x'')|\leq
M\omega(f,\Delta_i)+M\omega(g,\Delta_i)
\end{multline*}

Значит,
$$
\sum_{i=1}^n M\omega(fg,\Delta_i) \leq M \sum_{i=1}^n \left( \omega(f,\Delta_i) + \omega(g,\Delta_i) \right)
$$

Но выражение справа сколь угодно мало по необходимому и достаточному условию интегрируемости, значит, и выражение слева сколь угодно мало (сумма колебаний неотрицательна), значит, снова применив необходимое и достаточное условие интегрируемости, получим, что $(f\cdot g)\in R[a;b]$.

\dokno

Введём теперь определение неотрицательной и неположительной части функций:

\begin{opr}
$$
f_+(x)=\left\{
\begin{array}{l}
f(x), \mbox{~если~} f(x) > 0 \\
0,    \mbox{~если~} f(x) \leq 0
\end{array}
\right.
$$
\end{opr}


\begin{opr}
$$
f_-(x)=\left\{
\begin{array}{l}
f(x), \mbox{~если~} f(x) < 0 \\
0,    \mbox{~если~} f(x) \leq 0
\end{array}
\right.
$$
\end{opr}

Легко убедиться, что
$$
f_+(x)+f_-(x)=f(x)
$$

$$
f_+(x)-f_-(x)=|f(x)|
$$

Неотрицательная и неположительная части интегрируемой функции интегрируемы, т. е.
$$
f\in R[a;b] \Rightarrow \{f_+,f_-\}\in R[a;b]
$$

\dokvo

Заметим, что колебание неотрицательной (неположительной) части функции на некотором отрезке не превосходит колебания самой функции на данном отрезке.
Пусть $T$ - разбиение отрезка $[a;b]$ и ${\sum\limits_{i=1}^n \omega(f,\Delta_i)<\epsilon}$.
Тогда
$$\sum\limits_{i=1}^n \omega(f_+,\Delta_i)<\sum\limits_{i=1}^n \omega(f,\Delta_i)<\epsilon$$
$$\sum\limits_{i=1}^n \omega(f_-,\Delta_i)<\sum\limits_{i=1}^n \omega(f,\Delta_i)<\epsilon$$

Применив необходимое у достаточное условие интегрируемости функции, получим, что $\{f_+,f_-\}\in R[a;b]$

Как следствие, модуль интегрируемой функции сам является интегрируемым:
$$
f \in R[a;b] \Rightarrow |f| \in R[a;b] 
$$

Обратное, однако, неверно.
Пример -- функция $f(x)=\frac{1}{2}D(x)$, где $D(x)$ -- функция Дирихле.

Наконец, докажем следующее свойство:
$$
\{f,g\}\subset R[a;b], \forall(x\in[a;b])[f(x)\leq g(x)] \Rightarrow \intl_a^b f(x)dx < \intl_a^b g(x) dx
$$

\dokvo

Интерграл есть предел интегральных сумм.
Но, так как $f(x)\leq g(x)$, то
$$
S(f,(T,\xi))\leq S(g,(T,\xi))
$$
Переходя к пределу при $d(T)\to 0$, получим требумое неравенство.

\dokno

Как следствие, интеграл любой непрерывной положительной функции положителен:
$$
f(x)\in R[a;b], \forall(x\in[a;b])[f(x)>0] \Rightarrow \intl_a^b f(x) dx > 0
$$

Более того,
$$
f\in R[a;b] \Rightarrow \intl_a^b f(x) dx \leq \intl_a^b |f(x)| dx 
$$
