\begin{opred}
	Объем - это как площадь только в $\R^3$.
	Поэтому все определения объема эквивалентны определениям площади. Например:
	квадрируемость$\approx$кубируемость;
	граница$\approx$поверхность;
	фигура$\approx$тело, и т.д.
\end{opred}

\subsubsection{Лемма 3.}
Если $F_1,F_2$ - кубируемые тела такие, что $F_1\cap F_2$ - кубируемо и $V(F_1 \cap F_2)=0$, тогда $F_1\cup F_2$ - кубируема, и $V(F_1 \cup F_2)=V(F_1)+V(F_2)$
\dokvo
(см. Лемму 1)

\subsubsection{Лемма 4.}
Если поверхность тела F кубируема и $V(\delta F)=0)$, то F-кубируема.
\dokvo
(см. Лемму 2)

\subsubsection{Лемма 5.}
Граница тела вращения, определяемая непрерывной неотрицательной на [a;b] функции (т.е. тела, заметаемого вращением криволинейной трапеции x=a, x=b, y=f(x) вокруг $O_x$) имеет V=0.
\dokvo
(аналогично Упражнению после Леммы 2)

\subsubsection{Теорема 2.}
Объём тела вращения, определяемого непрерывной, неотрицательной на [a;b] функции y=f(x) вычисляется по формуле $V(F)=\pi\int_{a}^{b}f^2(x)dx$
\dokvo
Из Лемм 4,5 следует, что часть тела F, заключенного между плоскостями x=$\alpha;x=\beta$ кубируемо при $\forall(\alpha,\beta \in [a;b])$
\\
Пусть $V(\alpha,\beta)$ - объем этой части из Леммы 4 следует, что $V(\alpha,\beta)=V(\beta)-V(\alpha)$, где V($\alpha$)- объём тела вращения, ограниченного плоскостями x=a и x=b.
\\
Если $\alpha>\beta$, то получим, что $V(\alpha,\beta)$ - аддитивная функция промежутка. Очевидно, что:
$V_{мал.цилиндра}\leqslant V(\alpha,\beta) \leqslant V_{бол.цилиндра}$
\\
$\pi m^2(\beta-\alpha) \leqslant V(\alpha,\beta) \leqslant \pi M^2(\beta-\alpha)$,
где $m=min f(x), M=max f(x)$.
Тогда по теореме об аддитивной функции промежутка:
$V(F)=\pi\int_{a}^{b} f^2(x) dx$