\begin{opred}
Пусть f(x) определена на множестве E. Тогда колебанием функции f на множестве E называется
$$
\omega(f,E)=\sup_{x',x''\in E}|f(x')-f(x'')|=\sup_{x\in E}f(x)-\inf_{x\in E}f(x)
$$
если f(x) - ограничена на E, то $\omega(f,E) \neq\infty$
\end{opred}
\\
Рассмотрим промежуток [a;b], где $\vartriangle x_i = x_i - x_{i-1}; \vartriangle_i=[x_{i-1};x_i];$
$$
\vartriangle_{ij}=[x_{i-1;j-1};x_{ij}],
$$
Если $T_1$ - измельчение T.

\begin{teorema}
Для того, чтобы ограниченная функция f(x) была интегрируема по Риману в [a;b] НиД
$$
\forall(\epsilon>0)\exists(\sigma>0)\forall(T)[d(T)<\sigma\Rightarrow\sum_{i=1}^{n}\omega(f,\vartriangle_i)\cdot\vartriangle x_i<\epsilon]
$$
(если $\omega(f,\vartriangle_i)=\omega_i ,$ то $\sum_{i=1}^{n} \omega(f,\vartriangle_i)\cdot x_i= \sum_{i=1}^{n}\omega_i \vartriangle x_i$)
\end{teorema}

\dokvo
\subsubsection{Необходимость}
Дано: $f\in R[a;b]$
\\
Возьмем $\forall(\epsilon>0).$ Т.к. $f\in R[a;b],$ то выполняется критерий Коши:

$$
\forall(\epsilon>0)\exists(\sigma>0)\forall((T',\varphi'),(T'',\varphi''))[d(T')<\sigma_1 d(T')<\sigma\Rightarrow|S(f,(T',\varphi'))-S(f,(T'',\varphi''))|<\frac{\epsilon}{3}]
$$
\\
Пусть $T'=T''=T$, а $\varphi'' \neq \varphi'$.
\\
$\varphi'=(\varphi_1',...,\varphi_n')$ выберем так, чтобы:
$$
f(\varphi_i')>\sup_{x\in\vartriangle_i}f(x)-\frac{\epsilon}{3(b-a)}=M_i-\frac{\epsilon}{3(b-a)}
$$
где $M_i=\sup_{x\in\vartriangle_i}f(x)$
\\
$\varphi''=(\varphi_1'',...,\varphi_n'')$ выберем так чтобы:
$$
f(\varphi_i')>\inf_{x\in\vartriangle_i}f(x)-\frac{\epsilon}{3(b-a)}=m_i-\frac{\epsilon}{3(b-a)}
$$
Преобразуем условие:
$$
\sum_{i=1}^{n}\omega_i\vartriangle x_i=\sum_{i=1}^{n}(M_i-m_i)\vartriangle x_i=
$$

$$
=\sum_{i=1}^{n}(M_i-f(\varphi_i')+f(\varphi_i')f(\varphi_i'')+f(\varphi_i'')-m_i)\cdot\vartriangle x_i=
$$

$$
=\sum_{i=1}^{n}(M_i-f(\varphi_i'))\vartriangle x_i + \sum_{i=1}^{n}(f(\varphi_i')-f(\varphi_i''))\vartriangle x_i + \sum_{i=1}^{n}(f(\varphi_i'')-m_i)\vartriangle x_i<
$$

$$
\sum_{i=1}^{n}(\frac{\epsilon}{3(b-a)}\cdot\vartriangle x_i) + \sum_{i=1}^{n}f(\varphi_i')\vartriangle x_i + \sum_{i=1}^{n}f(\varphi_i'')\vartriangle x_i + \sum_{i=1}^{n}\frac{\epsilon}{3(b-a)}\cdot \vartriangle x_i =
$$

$$
=\frac{2\epsilon}{3(b-a)}\cdot\sum_{i=1}^{n}\vartriangle x_i + S(f,(T,\varphi')) - S(f,(T,\varphi''))<
$$

$$
<\frac{2\epsilon}{3(b-a)}\cdot (b-a) + \frac{\epsilon}{3}=\epsilon
$$
\dokno

\begin{teorema}
$f\in R[a;b] \Leftrightarrow$
\begin{equation}\label{nid_usl_int_koleb}
	\Leftrightarrow \forall(\epsilon>0)\exists(\delta>0)\forall(T:d(T)<\delta)
	\left[\sum_{i=1}^m\omega(f,\Delta_i) \Delta x_i <\epsilon\right]
\end{equation}
\end{teorema}

\dokvo

\subsubsection{Достаточность}
Дано: усл. Доказать: $f\in R[a;b]$
\\
В силу критерия Коши надо доказать что выполняется условие (*) (см.2.2).
\\
Возьмем два разбиения: ($T,\varphi$) и ($\tilde{T},\tilde{\varphi}$):$T \subset \tilde{T}$, т.е. $\tilde{T}$ - измельчение T. Пусть T имеет n  точек разбиения, и в каждом i-том подотрезке будет $n_i$ разбиений: значит, $\sum_{j=1}^{n}\vartriangle x_{ij}=\vartriangle x_i.$ Тогда
$$
|S(f,(T,\varphi))-S(f(\tilde{T},\tilde{\varphi}))|=|\sum_{i=1}^{n}f(\varphi_i)\vartriangle x_i - \sum_{i=1}^{n}\sum_{j=1}^{n}f(\varphi_{ij})\vartriangle x_{ij}|
$$

$$
|\sum_{i=1}^{n}\sum_{j=1}^{n}(f(\varphi_i)-f(\varphi_{i,j}))\vartriangle x_{ij}|\leq
$$

\subsubsection{Пояснение:}
почему $\sum_{i=1}^{n}f(\varphi_i)\vartriangle x_i$ можно заменить на $\sum_{i=1}^{n}\sum_{j=1}^{n}f(\varphi_{i})\vartriangle x_{ij}$?
\\
Вспомним, что такое $\sum_{i=1}^{n}f(\varphi_i)\vartriangle x_i$;
\\
$\sum_{i=1}^{n}\sum_{j=1}^{n}f(\varphi_{i})\vartriangle x_{ij}$
(та же самая площадь, только разбитая на более мелкие кусочки)
\\
Или алгебраически:
$\sum_{i=1}^{n}\sum_{j=1}^{n}f(\varphi_{i})\vartriangle x_{ij} = \sum_{i=1}^{n}f(\varphi_i)\sum_{j=1}^{n_i}\vartriangle x_{ij}=\sum_{i=1}^{n}f(\varphi_i)\vartriangle x_i$

$$
\leq |f(\varphi_i)-f(\varphi_{ij})\leq\omega(f,\vartriangle_i)=\omega_i|\leq
$$

$$
\leq\sum_{i=1}^{n}\sum_{j=1}^{n}\omega_i\vartriangle x_{ij} = \sum_{i=1}^{n}\omega_i\sum_{j=1}^{n}\vartriangle x_{ij}=
$$

$$
=\sum_{i=1}^{n}\omega_i \vartriangle x_i < \frac{\epsilon}{2}
$$
\\
Возьмём $\forall(\epsilon>0)$. Тогда $\exists(\sigma>0)[d(T)<\sigma\Rightarrow\sum_{i=1}^{n}\omega_i \vartriangle x_i < \frac{\epsilon}{2}]$
\\
Но!Возьмём 2 разбиения: $\forall$($(T',\varphi')$ и $(T'',\varphi'')$).
\\
Пусть $T=T'\cup T'' \Rightarrow T-$ измельчение T' и T'' $\Rightarrow d(T)<\sigma\Rightarrow$

$$
\Rightarrow|S(f,(T',\varphi'))-S(f,(T'',\varphi''))|=\leq
$$

$$
\leq|S(f,(T,\varphi))+S(f,(T',\varphi'))+S(f,(T,\varphi))-S(f,(T'',\varphi''))|\leq
$$

$$
\leq ||S(f,(T,\varphi))-S(f,(T',\varphi'))| + |S(f,(T,\varphi))-S(f,(T'',\varphi''))|<\frac{\epsilon}{2} + \frac{\epsilon}{2} = \epsilon
$$
Значит выполняется (*) из критерия Коши следует что $f\in R[a;b]$
\dokno
