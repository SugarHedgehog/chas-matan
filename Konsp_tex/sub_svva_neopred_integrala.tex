\subsubsection{Свойство 1.}
Производная непределённого интеграла равна подынтегральной функции:

$$\left(\int f(x) dx \right)'=f(x)$$
$$d\left(\int f(x) dx \right)=f(x)dx$$

\subsubsection{Свойство 2.}
Интеграл от производной функции равен этой функции с точностью до постоянной:

$$\int f'(x)dx=f(x)+C$$
$$\int df(x)=f(x)+C$$

Эти два свойства вытектают из определения.

\subsubsection{Свойство 3.}
Если функции $f(x)$ и  $g(x)$ имеют первообразную на $X$, то их линейная комбинация тоже имеет первообразную на $X$ и 

$$\int(\alpha f(x) + \beta g(x))dx=\alpha \int f(x)dx+ \beta \int g(x)dx$$

Доказать это равенство несложно -- достаточно продифференцировать правую и левую часть.
Таким образом, неопределённый интеграл линеен.

\subsubsection{Замечание.}

При последовательных преобразованиях выражения, содержащего неопределённые интегралы, произвольную аддитивную постоянную $C$, возникающую при взятии интеграла, пишут только в тех частях равенства, где нет других интегралов, и опускают в тех частях, где интегралы есть.

\subsubsection{Замечание.}

Знак интеграла $\int$ никогда не используется отдельно от указания переменной интегрирования, например, $dx$.

Сформулируем также следующую теорему, которая будет доказана позже:

\subsubsection{Теорема}

Если функция непрерывна на промежутке, то она интегрируема на этом промежутке.
