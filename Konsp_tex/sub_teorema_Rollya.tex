\subsubsection{Теорема.}

Если $f:[a;b]\to \R$ такова, что

1) $f$ непрерывна на $[a;b]$;

2) $f$ дифференцируема на $(a;b)$;

3) $f(a)=f(b)$,

то $\exists(c \in (a;b))[f'(c)=0]$.

\subsubsection{Замечание 1.}

Геометрическая интерпретация теоремы: пусть кривая задана функцей $y=f(x)$.
Тогда между любыми двумя точками с равными ординатами, лежащими на данной кривой, найдётся такая точка, в которой касательная к данной кривой параллельна оси абсцисс.

\subsubsection{Замечание 2.}

Условие (1) избыточно: т. к. уже требуется, чтобы $f$ была дифференцируема на $(a;b)$, достаточно потребовать непрерывности $f$ в $a$ и $b$. Остальные условия существенны.

\subsubsection{Следствие. Теорема о корнях производной.}

Между любых двух корней дифференцируемой функции лежит корень её производной.

\dokvo

Применим теорему Ролля к случаю, когда $f(a)=f(b)=0$.





