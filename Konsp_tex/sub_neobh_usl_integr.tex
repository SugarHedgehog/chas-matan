\begin{teorema}
Если $f\in R[a;b]$, то $f$ ограничена на $[a;b]$.
\end{teorema}

\dokvo

Идея доказательства заключается в том, что если функция неограничена, то при любом, сколь угодно мелком разбиении найдётся подотрезок, на котором она неограничена, и, двигая по этому отрезку отмеченную точку, можно добиться сколь угодно большой разницы интегральных сумм.

Итак, строгое доказательство.

Так как $f\in R[a;b]$, то $\exists(J\in \R)\left[J=\intl_a^b f(x) dx\right]$.

\pp: $f$ неограничена на $[a;b]$.
Рассмотрим некоторое разбиение $(T,\xi)$.
Тогда $\exists(i)[f \mbox{~неограничена на~}\Delta_i]$, то есть
$$
\exists(i)\forall(M>0)\exists(\xi_M \in \Delta_i)[|f(\xi_M)|>M]
$$

Обозначим
$$
S_i=\sum_{j=1,j\neq i}^n f(\xi_j)\Delta x_j
$$

Тогда
$$
|S(f,(T,\xi))|=|S_i+f(\xi_i)\Delta x_i|\geq
|f(\xi_i)\Delta x_i|-|S_i|
$$

Положим теперь
$$
M=\frac{|J|+1+|S_i|}{\Delta x_i}
$$

Тогда 
$$
|S(f,(T,\xi))|\geq |J|+1+|S_i|-|S_i|=|J|+1
$$

То есть
$$
|S(f,(T,\xi))-J|\geq|S(f,(T,\xi))|-|J|\geq 1
$$

А это означает, что для $\epsilon=1$ определение \ref{opr_opred_integral_1} не выполнено.
Мы пришли к противоречию, следовательно, наше допущение неверно, и функция $f$ ограничена на $[a;b]$.

\dokno

\begin{zamech}
Обратное неверное.
Так, функция Дирихле ограничена на любом отрезке, но не интегрируема на нём.
\end{zamech}


