\subsubsection{Теорема.}
Пусть $f:[a;b]\to \R$ и $f$ непрерывна на $[a;b]$, при этом $f(a) \cdot f(b) <0$,
т. е. на концах отрезка $[a;b]$ непрерывная на нём функция $f$ принимает значения разного знака.
Тогда $\exists(c \in (a;b))[f(c)=0]$,
т. е. хотя бы в одной точке интервала $(a;b)$ функция обращается в нуль.

\subsubsection{Замечание.}
Теорема Больцано-Коши не только утверждает существование точки, в которой функция обращается в нуль, но и фактически даёт способ её найти - методом половинного деления отрезка. Этот факт может быть применён при нахождении корня уравнения численными методами.

\subsubsection{Следствие 1 (теорема о промежуточном значении).}
\fXR, при этом $f$ непрерывна на некотором промежутке $Y \subset X$, $\{a;b\}\subset Y$, $a<b$.
Тогда $\forall(\gamma$ между $f(a)$ и $f(b))\exists(c:c\in[a;b])[f(c)=\gamma]$.

\subsubsection{Следствие 2.}
\fXR, $X$ - промежуток и $f$ непрерывна на нём.
Тогда $f(X)$ - тоже промежуток.


 

