\opred

Интеграл, не выражающийся через элементарные функции, называется неберущимся.

\subsubsection{Примеры.}
$$\int x^m(a+bx^n)^p dx$$
если $q=\frac{m+1}{n}, p \notin \Z, q \notin\Z, p+q\notin \Z$

$$\int \frac{e^x}{x^n}dx$$

$$\int \frac{\sin x}{x^n}dx$$

$$\int \frac{\cos x}{x^n}dx$$

$$\int \frac{e^{-x^2}}{x^n}dx$$

Часто в приложениях возникает интеграл вида $\int R(x,\sqrt{P_n(x)})dx$. Случаи, когда $n=1$ или $n=2$, исследованы нами ранее. В случае $n \geq 3$, вообще говоря, такой интеграл может быть неберущимся.

С помощью неберущихся интегралов определяются некоторые новые классы трансцендентных функций. Например, эллиптическими интегралами I, II и III рода называются соответственно:
$$\int \frac{dx}{\sqrt{(1-x^2)(1-k^2 x^2)}}$$
$$\int \frac{x^2 dx}{\sqrt{(1-x^2)(1-k^2 x^2)}}$$
$$\int \frac{dx}{(1+hx^2)\sqrt{(1-x^2)(1-k^2 x^2)}}$$

Здесь $0<k<1$.

