\opred
	Пусть $x_0 \in x$. Функция f(x) непрерывна в $x_0$ справа, если: $$\forall(\epsilon>0)\exists(\beta>0)\forall(x \in X)[x_0 < x < x_0+\beta\rightarrow |f(x)-f(x_0)| < \epsilon]$$

	непрерывна слева, если:  $$\forall(\epsilon>0)\exists(\beta>0)\forall(x \in X)[x_0 - \beta < x < x_0\rightarrow |f(x)-f(x_0)| < \epsilon]$$

\subsubsection{Утверждение 1.}
	Сравнивая это определение с определением правого(левого) $\lim f(x)$ по Коши, легко доказать, что справедливо утверждение 1:
	Если $x\rightarrow x_0 + 0$ $(x_0-0)$, то f(x) - непрерывная справа(слева) $\leftrightarrow\lim\limits_{x\rightarrow x_0\pm0}=f(x_0)$.

\opred
	Функця f(x) - непрерывна справа(слева) в $x_0$, если $$\forall({x_n}:x_n\subset X\wedge x_n \rightarrow x_0 \pm0 )[f(x_n)\rightarrow f(x_0)]$$

\subsubsection{Утверждение 2.}f(x) неперывна в $x_0 \in X \leftrightarrow f(x)$ - непрерывна справа и слева в этой точке $x_0$.

\opred
	Функция разрывна в $x_0$ справа(слева), если она не непрерывна справа(слева).

\subsubsection{Упражнение} Доказать, что f(x) = [x] - непрерывна справа в $\forall$ целой точке и разрывна слева в этой точке. А в $\forall$ точке x $(x\notin \mathbb {Z})$ она неперрывна справа и слева, т.е просто непрерывна.
