\section{Элементарные сведения из логики и теории множеств}
\subsection{Высказывания, предикаты связки}
\subsection{Кванторы}
\subsection{Множества, равенство двух множеств, подмножества}
\subsection{Простейшие операции над множествами}
\subsection{Принцип двойственности}
\subsection{Понятие счетного множества}
...

\section{Теория вещественных чисел}
\subsection{Множество рациональных чисел и его свойства}
\subsection{Вещественные числа, основные свойства вещественных чисел}
\subsection{Промежутки и их виды}
\subsection{Основные леммы теории вещественных чисел}
...

\section{Ограниченное множество, границы}
\subsection{Границы множества}
\subsection{Существование точной верхней границы у ограниченного сверху множества}
\subsection{Сечения в множестве рациональных чисел}
\subsection{Свойства $\sup$ и $\inf$}
\subsection{Отделимость множеств, лемма о системе вложенных отрезков}
\subsection{Лемма о последовательности стягивающихся отрезков}
...

\section{Отображения, функции}
\subsection{Отображения, виды отображений и т. д.}
\subsection{Вещественные функции}
...

\section{Предел последовательности}
\subsection{Последовательность элементов множества, числовая последовательность, определения предела числовой последовательноcти и бесконечно малой последовательности}
\subsection{Единственность предела последовательности}
\subsection{Подпоследовательности, связь пределов последовательности и подпоследовательности}
\subsection{Лемма о двух милиционерах}
\subsection{Основные теоремы о пределах последовательности}
\subsection{Понятие бесконечно большой последовательности}
\subsection{Монотонные последовательности, критерий существования предела монотонной послед}
\subsection{Существование предела последовательности $(1+1/n)^n$, число $e$ }
...

\section{Понятие предельной точки числового множества, теорема Больцано-Вейерштрасса, критерий Коши}
\subsection{Предельная точка множества}
\subsection{Теорема о последовательности, сходящейся к предельной точке}
\subsection{Теорема Больцано-Вейерштрасса}
\subsubsection{Теорема.}

Любое бесконечное ограниченное множество вещественных чисел имеет хотя бы одну предельную точку.

\mnemo

Название теоремы удобно запоминать по первым буквам прилагательных:

<<\textbf{Б}есконечное \textbf{о}граниченное множество \textbf{в}ещественных чисел>>

<<\textbf{Бо}льцано-\textbf{В}ейерштрасса>>

\subsection{Критерий Коши}
...

\section{Верхний и нижний пределы последовательности}
\subsection{Понятие расширенной числовой прямой, понятие бесконечных пределов}
\subsection{Понятие частичных верхних и нижних пределов последовательности. Теорема о существовании у каждой последовательности ее верхнего и нижнего предела}
\subsection{Характеристические свойства верхнего и нижнего предела последовательности}
\subsection{Критерий существования предела последовательности}
...

