\subsubsection{Теорема Ферма о производной в точке локального экстремума.}

\fXR, $f$ дифференцируема в точке внутреннего локального экстремума $x_0$.
Тогда $f'(x_0)=0$.

\mnemo
Чтобы запомнить содержание теоремы по её названию, нужно представить себе первую букву в нём (но не заглавную) - латинскую букву $f$. Тогда верхний и нижний "завитки" будут символизировать локальные экстремумы, а горизонтальная черта - горизонтальную касательную в точке, где производная равна нулю.  

\subsubsection{Замечание 1.}

В невнутренней точке локального экстремума производная может, вообще говоря, быть не равной нулю.
Пример: $f:[-1;1]\to \R$, невнутренний локальный максимум $x_0 = 1$, $f'(x_0)=2$.

\subsubsection{Замечание 2.}
Теорема Ферма необратима.
Пример: $f:\R\to\R$, $f(x)=x^3$, $f'(0)=0$, но $f$ не имеет локальных экстремумов.



