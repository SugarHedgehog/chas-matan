\begin{teorema}
Пусть $A\subset\R^n$, $f:A \to \R$ и $f$ дважды непрерывно дифференцируема в $x_0\in A$, $x_0$ - стационарная точка $f$,
второй дифференциал $f$ в точке $x_0$, т. е. $d^2 f(x_0,h)$ является невырожденной квадратичной формой.
Тогда $d^2 f(x_0,h)$ определяет наличие в точке $x_0$ локального экстремума, причём если $d^2 f$ --- положительно определённая квадратичная форма (напомним, от $h$), то функция $f$ имеет в точке $x_0$ локальный минимум, если $d^2 f$ --- отрицательно определённая квадратичная форма, то функция $f$ имеет в точке $x_0$ локальный максимум, если же $d^2 f$ --- неопределённая квадратичная форма, то локального экстремума в точке $x_0$ у функции $f$ нет.
\end{teorema}
\dokvo
