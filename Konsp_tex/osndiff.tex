\section{Дифференциальное исчисление функции одной независимой переменной}
\subsection{Определение производной и дифференциала, связь между этими понятиями}
\subsection{Связь между понятиями дифференцируемости и непрерывности функций}
\subsection{Дифференцирование и арифметические операции}
\subsection{Теорема о производной сложной функции. Инвариантность формы первого дифференциала}
\subsection{Теорема о производной обратной функции}
\subsection{Производные основных элементарных функций. Доказательство}
\subsection{Касательная к кривой. Геометрический смысл производной и дифференциала}
\subsection{Физический смысл производной и дифференциала}
\subsection{Односторонние и бесконечные производные}
\subsection{Производные и дифференциалы высших порядков}

\section{Основные теоремы дифференциального исчисления}
\subsection{Теорема Ферма}
\subsection{Теорема Ролля}
\subsection{Теорема Лагранжа и следствия из нее}
\subsection{Теорема Коши}
123

\section{Формула Тейлора}
\subsection{Формула Тейлора для многочлена}
\subsection{Формула Тейлора для произвольной функции. Различные формы остаточного члена формулы Тейлора}
\subsection{Локальная формула Тейлора}
\subsection{Формула Маклорена. Разложение по формуле Маклорена некоторых элементарных функций}

\section{Правило Лопиталя}
Пусть даны две непрерывные на интервале $(a; b)$ функции $f(x)$ и $g(x)$, где $\{a; b\} \subset \overline{\mathbb{R}}$. Неопределённостью типа $\left[\frac{0}{0}\right]$ в точке $a$ называется предел 
\[
\lim_{x \to a+}\frac{f(x)}{g(x)}
\]
в случае, когда
\[
\lim_{x \to a+}f(x) = \lim_{x \to a+}g(x) = 0
\]
Аналогично определяются неопределённости вида $\left[\frac{\infty}{\infty}\right]$ и в точке $b$.

Другие виды неопределённостей сводятся к этим двум. Вообще говоря, неопределённость типа $\left[\frac{\infty}{\infty}\right]$ может быть сведена к типу $\left[\frac{0}{0}\right]$. Действительно, пусть
$\lim_{x \to a+}f(x) = \lim_{x \to a+}g(x) = \infty$
\[
\frac{f(x)}{g(x)}=\frac{\frac{f(x)}{g(x)}}{\frac{f(x)}{g(x)}}
\]
 




