\subsubsection{Теорема.}

Функция, непрерывная на компакте, достигает на нём точных верхней и нижней границ множества своих значений.

\mnemo
Эту теорему можно запоминать по начертанию цифры {\LARGE 2}, разделив его на три части: горизонтальная черта снизу - отрезок, средняя часть - непрерывная функция, <<завиток>> сверху - точное верхнее значение.


\subsubsection{Следствие.}

Пусть $f:[a;b]\to\R$ и $f$ непрерывна, $\alpha=\inf(f[a;b])$, $\beta=\sup(f[a;b])$.
Тогда $f([a;b])=[f(a);f(b)]$.



