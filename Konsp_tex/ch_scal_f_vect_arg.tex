\section{Скалярные функции векторного аргумента}
\subsection{Пространство \Rn}
\begin{opr}
Пространство \Rn -- множество упорядоченных наборов из $n$ вещественных чисел:
$$
x\in \R^n \Leftrightarrow x=(x^1, ... , x^n),~~x^i\in\R,~~i=1,...,n
$$
\end{opr}
\begin{zamech}
\Rn -- линейное пространство.
Оно более детально изучается в курсе линейной алгебры.
\end{zamech}

\begin{zamech}
Индекс (номер) координаты вектора пишется вверху, т. к. нижний индекс необходим в выкладках, содержащих последовательности.
Как правило, такие обозначения не приводят к недоразумению и путанице с обонзанчением степени.
\end{zamech}

Выпишем определения операций в \Rn -- сложения и внешнего умножения:

$$
\forall(x=(x^1,...,x^n)\in\R^n,y=(y^1,...,y^n)\in\R^n)[x+y=(x^1+y^1,...,x^n+y^n)]
$$

$$
\forall(\lambda\in\R)\forall(x=(x^1,...,x^n)\in\R^n)[\lambda x=(\lambda x^1,...,\lambda x^n)]
$$

Нулевой вектор, как и скалярный нуль, и нулевой оператор, и т. д., будем обозначать символом $0$. Опять же, в большинстве случаев к недоразумению такое обозначение не приводит.

Все выкладки будем давать в стандартном базисе $e$:
$$\begin{aligned}
e_1=(1,0,...,0)\\
e_n=(0,...,0,1)
\end{aligned}$$

Напомним также тот факт, что любой вектор разложим по базису:
$$
\forall(x\in\R^n)\exists(\alpha_1,...,\alpha_n\in\R)[x=\alpha_1 e_1+...+\alpha_n e_n]
$$

Примеры пространств:

$\R^1=\R$

$\R^2$ - точки плоскости.

$\R^3$ - точки пространства.


\subsection{Нормированное пространство \Rn}
\begin{opr}
\Rn -- нормировано, если каждому вектору $x\in\R^n$ сопоставлено вещественное число $\|x\|$, так, что:

\begin{equation}\label{aks1normy}
\|x\|=0 \Leftrightarrow x=0
\end{equation}

\begin{equation}\label{aks2normy}
\forall(\lambda\in\R)[\|\lambda x\|=|\lambda|\|x\|]
\end{equation}

\begin{equation}\label{aks3normy}
\forall(y\in\R^n)[\|x+y\|\leq\|x\|+\|y\|]
\end{equation}

\end{opr}

Эти три формулы называют аксиомами нормы.
Заметим, что неотрицательность нормы нет необходимости вводить как аксиому:
$$
2\|x\|=\|x\|+|-1|\|x\|=\|x\|+\|-x\|\geq\|x+(-x)\|=\|0\|=0
$$

Норма, вообще говоря, является скалярной функцией векторного аргумента, но определение такой функции будет дано далее.

\begin{opr}
Евклидовой нормой называют норму, введённую равенством
\begin{equation}
|x|=\sqrt{\sum_{i=1}^n (x^i)^2}
\end{equation}
\end{opr}

Евклидову норму обозначают не двойными вертикальными чертами, а одинарными.
Пространство \Rn, в котором введена евклидова норма, называт евклидовым.
В $\R^1$ евклидова норма -- не что иное, как модуль.

Аксиома (\ref{aks3normy}) приводит к неравенству Буняковского-Шварца:
\begin{equation}\label{nervoBunSchvarz}
\sum_{i=1}^n|a^i b^i|\leq\sqrt{\sum(a^i)^2}+\sqrt{\sum(b^i)^2}
\end{equation}

Заметим, однако, что это неравенство возможно доказать и без применения методов математического анализа или линейной алгебры.

Другим следствием аксиомы (\ref{aks3normy}) является неравенство Коши - Миньковского:
\begin{equation}\label{nervoKoshiMink}
\sqrt{\sum_{i=1}^n(x^i+y^i)^2}\leq\sqrt{\sum_{i=1}^n(x^i)^2}+\sqrt{\sum_{i=1}^n(y^i)^2}
\end{equation}

Заметим, что можно вводить и неевклидовы нормы, например:
\begin{equation}
\|x^1,...,x^n\|=\max\limits_{1...n} x^i
\end{equation}

\begin{equation}
\|x^1,...,x^n\|=\sum_{i=1}^n x^i
\end{equation}

\begin{opr}
Две нормы $\|x\|_1$ и $\|x\|_2$ в \Rn называются эквивалентными, если
\begin{equation}\label{opred_eqiv_norm}
\exists(c_1>0, c_2>0)\forall(x\in\R^n)[c_1\|x\|_1\leq\|x\|_2\leq c_2\|x\|_1]
\end{equation}
\end{opr}

Примем пока без доказательств утверждение, что в конечномерных \Rn любые две нормы эквивалентны.
Позже оно будет доказано.

\begin{opr}
Множество $G\subset\R^n$ -- ограниченно, если
\begin{equation}\label{opr_pgran_mn}
\exists(c>0)\forall(x\in G)[\|x\|\leq c]
\end{equation}
\end{opr}

\begin{opr}
Функция $\rho(x,y)$, где $x\in\R^n$, $y\in\R^n$, называется метрикой, если выполнены следующие аксиомы (аксиомы метрики):
\begin{equation}\label{aks1metr}
\rho(x,y)=0 \Leftrightarrow x=y
\end{equation}

\begin{equation}\label{aks2metr}
\rho(x,y)=\rho(y,x)
\end{equation}

\begin{equation}\label{aks3metr}
\rho(x,y)\leq\rho(x,z)+\rho(z,y)
\end{equation}

\end{opr}

Заметим, что неотрицательность метрики следует из третьей аксиомы метрики при $y=x$.

\begin{opr}

Евклидово пространство, в котором введена метрика 
\begin{equation}\label{evkl_metrika}
\rho(x,y)=\|x-y\|
\end{equation}
называют метрическим.

\end{opr}

Заметим вскользь, что метрику можно ввести и без использования понятия нормы.

\subsection{Последовательность в \Rn.Сходимость последовательностей. Эквивалентность покоординатной сходимости}
\begin{opr}
Последовательностью в \Rn называется отображение $f:\N\to\R^n$.
\end{opr}
Это означает, что $\forall(k\in\N)\exists(x_k\in\R^n)[f(k)=x_k]$.

\begin{primer}
\begin{multline*}
\left\{x_k=\left( \frac{1}{k};k^2+1;2^k; \frac{k}{3k+1}\right)\right\}
\\
x_1=\left(1;2;2;\frac{1}{4}\right)
\\
x_2=\left(\frac{1}{2};5;4;\frac{2}{7}\right)
\end{multline*}
и т. д.
\end{primer}

\begin{opr}
Пусть $\{x_k\}\subset\R^n$ - последовательность.
Если $$\exists(x_0\in\R^n)[\{\|x_k-x_0\|\}\to0]$$ (здесь $\{\|x_k-x_0\|\}$ -- числовая последовательность), то говорят, что $\{x_k\}$ сходится к $x_0$ и пишут:
$$
\{x_k\}\to x_0
$$
или
$$
\lim x_k =  x_0
$$
или
$$
\lim_{k\to\infty} x_k =  x_0
$$
\end{opr}
Иначе говоря,
\begin{equation*}
\{x_k\}\to x_0 \Leftrightarrow \forall(\varepsilon>0)\exists(k_0\in\N)\forall(k>k_0)[\|x_k-x_0\|<\varepsilon]
\end{equation*}

Легко доказать, что если две нормы эквивалентны, то сходимость по первой из этих норм равносильна сходимости по второй.

\begin{teorema}
Сходимость по норме эквивалентна покоординатной сходимости, т. е.
$$
\{x_k\}\to x_0 \Leftrightarrow \forall(i\in\Z\cap[1;n])[x_k^i\to x_0^i]
$$
\end{teorema}
\dokvo
Так как все нормы эквивалентны, то докажем утверждение только для евклидовой нормы (\ref{evklidova_norma}):
$$|x_k-x_0|=\sqrt{\sum_{i=1}^{n}(x_k^i-x_0^i)^2} \to 0
\Rightarrow \forall(i\in\Z\cap[1;n])[x_k^i- x_0^i\to0]
$$
\dokno

\begin{sledstvie}
\begin{multline*}
\forall(\{x_k\}\to x_0:\{x_k\}\subset \R^n,\{y_k\}\to y_0:\{y_k\}\subset \R^n,\{\lambda_k\}\to \lambda_0:\{\lambda_k\}\subset \R)
\\
[\{x_k+y_k\}\to x_0+y_0 ~\cap~ \{\lambda_k x_k\}\to \lambda_0 x_0]
\end{multline*}
\end{sledstvie}

\begin{sledstvie}
Некоторое множество $G\subset\R^n$ ограничено тогда и только тогда, когда ограничено множество, состоящее из вещественных чисел, являющихся координатами элементов $G$.
\end{sledstvie}

\begin{teorema}[Больцано-Вейерштрасса для \Rn]
Из любой ограниченной последовательности можно выделить сходящуюся подпоследовательность.
\end{teorema}
\dokvo
Пусть $\{x_k\}\subset\R^n$ --- последовательность.
Выделим из неё сначала подпоследовательность $\{x_{k_1}\}$ так, что последовательность первых координат $\{x_{k_1}^1\}$ сходится;
(это возможно по теореме Больцано-Вейерштрасса для $\R$, так как множество значений первых координат ограничено)
затем выделим из $\{x_{k_1}\}$ подпоследовательность $\{x_{k_2}\}$, такую, что последовательность вторых координат $\{x_{k_1}^2\}$ сходится.
Продолжая действовать подобным образом, получим требуемую последовательность $\{x_{k_n}\}$, сходящуюся покоординатно.
\dokno

\subsection{Замкнутые, открытые, компактные множества в \Rn}
Пусть $\R^n - $нормированное пространство.
\begin{opred}
Открытый шар с центром в $x_0$ и радиусом r - множество $B(x_0,r)$, состоящее из $\{x:x\in\R^n \textasciicircum ||x-x_0||\le r\}$
\end{opred}
\subsubsection{Упражнение:}
Выяснить, что представляют собой открытые шары единичного радиуса в различных пространствах.

\begin{opred}
Пусть $G\subset\R^n. x_0\in G$ - внутренняя точка, если $x_0\in G$ вместе с некоторым открытым шаром, т.е. $\exists(r>0)[B(x_0,r)\subset G]$ или
\\
$\exists(r>0)\forall(x:||x-x_0||<r)[x\in G]$
\end{opred}

\begin{opred}
$G\subset\R^n$ - открытое, если любая его точка внутренняя.
\end{opred}

\subsubsection{Упражнение:}
Доказать, что $B(x_0,r)$ и множество G, состоящее из $\{x\in\R^n:x^i>0,i\{1;n\}\}$ - открытые.

\begin{opred}
$x_0$ - предельная точка множества G, если 
$$
\forall(B(x_0,r))\exists(x\in G: x\ne x_0)[0<||x-x_0||<r]
$$
\end{opred}

\subsubsection{Упражнение:}
Доказать, что если $x_0$ - предельная точка множества G, то $\exists\{x_k\}$ - последовательность элементов множества G, отличных от $x_0$, сходящиеся к $x_0$.

\begin{opred}
Изолированнные точки множества G - точки множества G, которые не являются предельными, или 
$$
\exists(B(x_0,r))\forall(x\in G: x\ne x_0)[||x-x_0||\ge r]
$$
\end{opred}

\begin{opred}
Множество G - замкнуто, если оно содержит в себе все свои предельные точки.
\end{opred}

\subsubsection{Свойства открытых и замкнутых множеств:}
1. Пересечение любого числа и объединение конечного числа замкнутых множеств является замкнутым множеством.
\dokvo
Пусть $G_2,(\alpha\in\Lambda$ - замкнутые множества, $G=\cap G_2)$
\\
Пусть G - незамкнутое множество. Тогда существует (предельная точка $x_0$)$[x_0\notin G] \Rightarrow$
\\
$\Rightarrow x_0$ - предельная точка $G_2\Rightarrow$
\\
$\Rightarrow \exists(\alpha_0\in\Lambda)[x_0\notin G_{\alpha_0}],$
\\
следовательно $G_{\alpha_0}$ - не замкнуто.
Противоречие.
\dokno
2. а) Дополнение замкнутого множества до всего 		пространства - открытое множество;
   б) Дополнение открытого множества до всего пространства - замкнутое множество.
\dokvo
а) Пусть G - замкнутое множество.
\\
Пусть $G_{\R^n}G$ - дополнение - не открытое множество. Значит,
\\
$\exists(x_0 - $ не внутренняя) $\forall(r>0)\exists(x\in B(x_0,r))[x\notin G_{\R^n}G]$
\\
Отсюда: $x_0$ - предельная точка $G\Rightarrow x_0\in G$ (т.к. G - замкнутое) - противоречие, т.к. $x_0$ - не внутренняя
\dokno

б) доказывается аналогично
\\
3. Объединение  любого числа и пересечение конечного числа открытых множеств является открытым множеством.
\\
4. $K\subset\R^n$ - компактное, если из $\forall$ последоват. $\{x_k\}$ элементов этого множества можно выделить подпоследовательность $\{x_{n_k}\}$, сходящуюся к элементу из множества К.
\dokvo
Достаточность:
\\
Дано: К - огр. и замкнуто;
\\
Доказать: К - компакт.
\\
Возьмем $\forall\{x_{n_k}\}$ т.к. $x_k\in K$ -ограничено, то и $\{x_k\}$ - ограничено. Тогда, в силу теоремы Больцано-Вейрштресса из последовательности $\{x_k\}$ можно выделить сходящуюся подпоследовательность $\{x_{k_n}\},$ где $x_{k_n}\to x_0\Rightarrow x_0$ - предельная точка множества K т.к. К замкнуто, то $x_0\in K,$ следовательно К- компактно
\dokno





\subsection{Функции многих переменных. Предел. Непрерывность }
\begin{opred}
Пусть $G\subset\R^n; f:G\to\R$ - скалярная функция вещественного аргумента, функция многих переменных, или функция нескольких переменных.
\end{opred}
Например:
\\
$f = arctg((x^1)^3+x^2), (x^1,x^2)\in\R^2$
\\
Определение предела по Коши
\\
Пусть $\R^n$ - нормированное пространство, $x_0$ - предельная точка G, $f:G\to\R$

\begin{opred}
Число A - предел функции f в т. $x_0,$ если
$$
\forall(\epsilon>0)\exists(\delta>0)\forall(x\in G)[0<||x-x_0||<\delta\Rightarrow|f(x)-A|<\epsilon]
$$
\end{opred}

\begin{opred}
$x_0$ предельная точка G, $f:G\to\R, A\in\R$ - предел f в $x_0$, если
$$
\forall(\{x_k\}:x_k\in G, x_k\ne x_0)[x_k\to x_0\Rightarrow\{f(x_k)\}\to A]
$$
или $[f(x_k)\to A$ при $k\to\infty]$
\\
или $[A=lim f(x)$, т.е. $f(x)\to A$ при $x\to x_0]$
\end{opred}
Так ж, как и для функций 1 переменной доказывается эквивалентность определения по Коши и по Гейне и свойство пределов, связанное с арифметическими операцтями. В сиду эквивалентности норм выполняется равентсво $A=\lim_{x\to x_0} f(x)$

\subsubsection{Упражнение:}
Сформулировать определение предела для трех норм, введенных в $\R^n$

\subsubsection{Примечание:}
$x_k$ - обозначение к-того члена последовательности;
\\
$x^k$ - обозначение к-той координаты
\\
Спецификой функций многих переменных являются повторные пределы.
\\
Пусть $H,K\subset\R; a=lim H, b=lim K.$ Рассмотрим $G=H\times K$ (декартово произведение). Тогда $lim G= x_0,$ где $x_0=(a,b)$.

\begin{teorema}
Пусть $f:G\to\R, A=\lim_{x\to x_0} f(x),$ где $x_0=(a,b).$ Тогда:
$$
\forall(x^1\in H:x^1\ne 0)\exists(\lim_{x^1\to b}f(x^1,x^1)=g(x^1))[\exists(\lim_{x1\to a} g(x^1)=A)]
$$
\end{teorema}
\dokvo
В силу определения $lim f(x^1,x^2)$ по Коши:
$$
\forall(\epsilon>0)\exists(\delta>0)\forall(x^1\in K, x^2\in H)
$$

$$
[0<||x^1-a||<\delta ^ 0<||x^2-b||<\delta\Rightarrow|f(x^1,x^2)-A|<\frac{\epsilon}{2}]
$$
т.к. дано, что $A=\lim_{x\to(a,b)}f(x),$ то $\exists(\lim_{x^2\to b}f(x^1,x^2)=g(x^1)).$
\\
Т.к. $\exists(lim f(x^1,x^2)=g(x^1)),$ то в последнем неравенстве можно перейти к $lim x^2=b.$ Тогда
$$
|g(x^1)-A|\le\frac{\epsilon}{2}<\epsilon, 0<|x^1-a|<\delta\Rightarrow
$$

$\Rightarrow \exists(\lim_{x\to a}g(x^1)=A)$
\dokno

\subsubsection{Замечание:}
Теорема обратная данно теореме НЕВЕРНА!!!
\\
Например:
1. Рассмотрим $f:\R^2\to\R:f(x^1,x^2)=\frac{x^1\cdot x^2}{(x^1)^2+2(x^2)^2}$
\\
Если $x^2\to 0, x\ne 0$ - фиксировать, то $g(x^1)=0\Rightarrow \lim_{x^1\to 0}g(x^1)=0.$ И следовательно, $\lim_{x^1\to 0}\lim_{x^2\to 0}f(x^1,x^2)=0$
\\
2. Рассмотрим прямую (пусть($x^1,x^2)\to 0$):
\\
$f(x^1x^2)=x^1$
\\
$\neg\exists f(x^1,x^2)$ при $(x^1,x^2)\to(0;0)$
\\
Пусть $x^2=0. f(x^1,0)=0, x^1\ne 0\Rightarrow$
\\
$\Rightarrow lim - $нет а повторный lim - есть.
\\
$G\subset\R^n, f:G\to\R^n$ - непрерывна в $x_0\in G$ если:
$$
\forall(\epsilon>0)\exists(\delta>0)\forall(x\subset G)[||x-x_0||<\delta\Rightarrow |f(x)-f(x_0)|<\epsilon]
$$

\subsubsection{Упражнение:}
Доказать, что в изолированных точках функция непрерывна, а в т. $x_0\in G$ - предельной точке f непрерывна $\leftrightarrow f(x_0)=\lim_{x\to x_0}f(x)$
\\
$f(\lim_{x\to x_0}x)=\lim_{x\to x_0}f(x)$ - для непрерывных функций знаки f и lim можно поменять местами

\begin{opred}
$f:G\to\R^1$ - непрерывна на G, если она непрерывна в каждой точке из G:
$$
\forall(x_0\in G, \epsilon>0)\exists(\delta>0)\forall(x\in G)
$$
$$
[||x-x_0||<\delta\Rightarrow|f(x)-f(x_0)|<\epsilon]
$$

\end{opred}

\subsubsection{Свойства непрерывной функции:}
1. Арифметические свойства:
\\
$G\subset\R^n; f,g:G\to\R^1; f,g - $непрерывны в $x_0\in G.$
\\
Тогда $(f\pm g), (f\cdot g), (f/g,$ если $g(x_0)\ne 0)$ - непрерывны в $x_0.$
\\
2. Свойства сохранения знака:
\\
$G\subset\R^n; f:G\to\R^1, f(x_0)\ne 0\Rightarrow$
\\
$\Rightarrow\exists(r>0)\forall(x\in G\cap B(x_0,r))[f(x)\cdot f(x_0)>0]$ - т.е. $f(x)$ и $f(x_0)$ имеют одинаковые знаки.
\\
3. Непрерывность суперпозиции:
\\
$G\subset\R^n, f:G\to\R^1 -$ непрерывна в $x_0\in G.$
\\
Пусть есть n функций $\varphi^i:[a;b]\to\R^1, i=\{1;n\}$ такие, что $\forall(t\in[a;b])[x(t)=(\varphi^1(t),...,\varphi^n(t))\in G]$
\\
$\varphi^i$ - непрерывна в $t_0, x(t_0)=x_0.$ Тогда сложная функция $F(t) = f(x(t))$ - непрерывна в $x_0$
\\
4. Свойства функций, непрерывных на компакте:
\\
$K\subset\R^n, K -$ компакт, $f:K\to\R^1$ - непрер на К. Тогда для f верны следующие теоремы:
\\
	а) 1 теорема Вейрштрасса
	\\
		f - ограничена на К;
	\\
	б) 2 теорема Вейрштрасса:
	\\
		f достигает min и max на К;
	\\
	в) Теорема Кантора:
	\\
		f - равномерно непрерывна на К, т.е.
		$$
		\forall(\epsilon>0)\exists(\delta>0)\forall(x_1,x_2\in K)[||x_1-x_2||<\delta\Rightarrow|f(x_1)-f(x_2)<\epsilon]
		$$
5. Теорема Больсано-Коши
\begin{opred}
Множество $G\subset\R^n$ - связанное, если любые его точки можно соединить непрерывной параметризованной кривой, т.е. $\forall(x_1,x_2\in G)\exists([a;b]$, n штук непрерывных на [a;b] функций:
$
\varphi^i[a;b]\to\R^1, i=\{1;n\})\forall(t\in[a;b])$
$$
[x(t)=(\varphi^1(t),...,\varphi^n(t))\in G  x_1=(\varphi^1(a),...,\varphi^n(a))]
$$
$x_2=(\varphi^1(b),...,\varphi^n(b))]$
\end{opred}

\begin{teorema}
Пусть f - непрерывна на G, где $G\subset\R^n$ - связанное. Тогда, принимая некоторые значения на G она(функция f) принимает все значения из множества G
\end{teorema}
\dokvo
Пусть f на G принимает какие-нибудь 2 значения: $f(x_1) = A, f(x_2) = B, x_1,x_2\in G.$
\\
Докажем, что $\forall(C\in[A;B])\exists(x_3)[f(x_3)=C]$
\\
Т.к. G- связаное, то существует непрерывная кривая x(t), соединяющая $x_1$ и $x_2(t\in[a;b]).0$
\\
$F(t)=f(x(t)), t\in[a,b] -$ непрерывна на [a;b].
\\
$F(a)=f(x(a)) = f(x_1) = A$ и $F(b)=f(x(b))=f(x_2)=B$
\\
По теореме о промежуточном значении $\exists(t_0\in[a;b])[F(t_0)=C] \Rightarrow$
\\
$f(x(t_0))=C\Rightarrow$ f принимает $\forall$ значение.
\dokno

\begin{teorema}
Эквивалентность 2-х норм в $\R^n.$\\
Докажем, что для
$$
\forall(||x||_1,||x||_2)\exists(c_1,c_2)[c1||x||_1\le||x||_2\le c_2||x||_1, x\in\R^n]
$$
\end{teorema}

\dokvo
Рассмотрим $S(0;1)=\{x:x\in\R^n, |x|=1\}$ и $\varphi(x)=||x||.$
\\
$$|\varphi(x)-\varphi(y)|=| ||x||\cdot||y|| |\le ||x-y|| = ||\sum_{i=1}^{n}(x^i - y^i)e_i||\le$$
$$
\le \sum_{i=1}^{n}|x^i - y^i|\cdot||e_i||\le|x-y|\sum_{i=1}^{n}||e_i||,
$$
Если х и у - различны по базису, то
$$
x = \sum_{i=1}^{n}x^i e_i, y = \sum_{i=1}^{n}y^i e_i\Rightarrow
$$
$\Rightarrow \varphi$ - равномерно непрерывна на S.
\\
S(0,1) - ограничено и замкнуто $\Rightarrow\varphi$ достигает max и min на S(m=min, M = max)
\\
$m\le\frac{x}{|x|}\le M\Rightarrow m|x|\le||x||\le M|x|$
\\
Здесь в качестве $c_1$ взято m, в качестве $c_2$ взято M. При х=0 это выражение так же будет справедливо.


\section{Дифференцирование скалярных функций векторного аргумента}
\subsection{Линейные функционалы в \Rn}
\begin{opred}
Линейный функционал на $\R^n$ - это $f:\R^n \to \R^1$ для которого выполняется:
\\
1.Свойство аддитивности:
$$
\forall(x,y\in\R^n)[f(x+y)=f(x)+f(y)]
$$
2.Свойство однородности:
$$
\forall(x\in\R^n)\forall(\alpha\in\R)[f(\lambda_x)=\lambda f(x)]
$$
\end{opred}

\begin{opred}
	Внутреннее (скалярное) произведение элементов x,y - 
	\\
	это $\sum_{i=1}^{n}x^i\cdot y^i$ : $<x,y>=\sum_{i=1}^{n}x^i\cdot y^i$ для $\forall(x=(x^1,...,x^n))$ 
	\\
	и $y=(y^1,...,y^n)\in\R^n)$
\end{opred}

\subsubsection{Свойства скалярного произведения (для $x,y\in\R^n$)}
$1.\forall(x,y)[<x,y>=<y,x>];$
\\
$2.\forall(x,y,z\in\R^n)[<(x+y),z>=<x,z>+<y,z>];$
\\
$3.\forall(x,y,z\in\R^n,\lambda)[\lambda<x,y>=<\lambda x,y>];$
\\
$4.\forall(x\in\R^n)[(<x,x>\geq 0 ^ <x,x>=0)\Leftrightarrow(x=0)];$
\\
$5.|<x,y>|\le |x|\cdot |y| ,$
\\
 так как $|x|=\sqrt{\sum_{i=1}^{n}(x')^2} = \sqrt{<x,x>}$, а 
\\
$
|<x,y>| = \sum_{i=1}^{n} |a_i b_i|\le\sqrt{\sum_{i=1}^{n}a^2}\cdot\sqrt{\sum_{i=1}^{n}B^2}=|x|\cdot |y|
$
\\
(по неравенству Бониковского-Шварца)

\subsubsection{Предположение}
$f:\R^n\to\R^1$ - линейный, $f(x)=<x,u>$, где $u\in\R^n$ - фиксированный.
\\
$e_1,...,e_n$- стандартный базис "бегающая 1".
\\
Пусть $f(e_i)=u_i;i=\{1,n\}$. Тогда 
$$
\forall(x\in\R^n)[x=\sum_{i=1}^{n}x^i\cdot e_i]\Rightarrow f(x)=\sum_{i=1}^{n}x^i f^i(e_i)=\sum_{i=1}^{n} x^i\cdot u_i=<x,u>
$$
где $u=(u_1,...,u_n)$
\\
Получается:$\forall(x\in\R^n)[|f(x)|=|<x,u>|\le|x|\cdot |u|]$, по неравенству Бониковского-Шварца.
\\
\begin{opred}
	Норма функционала f - норма вектора u, порождающего функционал f. (||f||=|u|)
	\\
	Таким образом $|f(x)|\le|x|\cdot|u|=|x|\cdot||f||$
\end{opred}
\\
Любой линейный функционал на $\R^n$ является непрерывным т.к. $\forall(x,x_0 \in \R^n)[|f(x)-f(x_0)|=|f(x-x_0)|\le||f||\cdot |x-x_0|]$
\\
Отсюда $|f(x)-f(x_0)|$ будет как угодно мал, когда $|x-x_0|<\delta|$. Тогда за $\delta$ можно взять:
$$
|x-x_0|<\frac{\epsilon}{||f||}=\delta
$$








\subsection{Определение дифференциала скалярной функции векторного аргумента. Связь между понятиями дифференцируемости и непрерывности}
\begin{opred}
Функция $f:E\to\R_1$ дифференцируема в $x\in E$ (дифференцируема по Фреше), если $\exists$(линейный функционал 
\\$e(x):\R^n\to\R^1)\forall(h\in\R^n:x+h\in E)[f(x+h)-f(x)=l(x)(h)+\omega(x,h)]$, где $\omega(x,h)=0 (||h||)$ при $h\to 0$, т.е. $\frac{\omega (x,h)}{||h||}\to 0$ при $h\to 0$.
\end{opred}

\begin{opred}
	Производная функции f в т. x (f'(x)) - это линейный функционал $e(x):\R^n \to \R^1$ 
\end{opred}

\begin{opred}
	Дифференциал функции f в т. x - значение линейного функционала на элементе h. df(x,h). Отсюда:
	$df(x,h)=f'(x)h$
\end{opred}

$\vartriangle x(h)=(x+h)-x=h; \vartriangle f(x,h) = f(x+h)-f(x)$
\\
Подставим эти обозначения в определение дифференцируемости по Фреше:
\\
$\vartriangle f(x,h) = df(x,h) + \omega (x,h)$, где $\omega (x,h) = o(||h||), h\to 0$
\\
Пусть $E \subset \R^n$ - открытое подмножество $\R^n$. Тогда: 
\begin{teorema}
Единственность производной по Фреше.
\end{teorema}

\dokvo
Пусть $f:E\to\R^1$ дифференцируема в $x\in E$ и  имеет 2 производные. Тогда по определению производной по Фреше:
$$
\exists (f_1'(x),f_2'(x))[f(x+h)-f(x)=f_1'(x)h+\omega_1 (x,h)] (1)
$$

$$
[f(x+h)-f(x)=f'(x)h+\omega_2 (x,h)] (2)
$$
\\
где $\omega\omega_1(x,h)=o(||h||)$ и $\omega_2(x,h)=o(||h||), h\to 0$
\\
$(1)-(2):o=f_1'(x)h+\omega_1(x,h)-f_2'(x)h-\omega_2 (x,h).$
\\
$(f_1'(x)-f_2'(x))h=\omega_2(x,h)-\omega_1(x,h)$
\\
$|\frac{f_1'(x)-f_2'(x)}{||h||}\cdot h|=|\frac{\omega_2(x,h)-\omega_1(x,h)}{||h||}\le|\frac{\omega_2(x,h)}{||h||}|+|\frac{\omega_1(x,h)}{||h||}|$
\\
$\frac{\omega_2(x,h)}{||h||} \to 0$ и $\frac{\omega_1(x,h)}{||h||} \to 0$, при $h\to 0$
\\
Возьмем $\forall (h_0\in\R^n, h=t\cdot h_0, t\in\R^1, h_0$ - фиксирована) 
\\
Тогда $\frac{|(f_1'(x)-f_2'(x))t h_0|}{|t|||h_0||}\to 0$ при $t\to 0$
\\
$\frac{(f_1'(x)-f_2'(x))h_0|}{||h_0||}\to 0$ при $t\to 0$, но т.к. это выражение от t не зависит, то это выражение = const.
\\
если $h_0 = 0$, то $f_1'(x(o))=f_2'(x(o))=0$
\\
если $h_0 \neq 0$, то 
\\
$f_1'(x)-f_2'(x)=0\Rightarrow f_1'(x)=f_2'(x)$
\\
\dokno

\begin{teorema}
Если $f:E\to\R^2$ - дифференцируема по Фреше в $x\in E$, то f - непрерывна в x.
\end{teorema}
\dokvo
$f(x+h)-f(x)=f'(x)h+\omega(x,h)$, где $\omega(x,h) = o(||h||), h\to 0$
\\
Пусть $h\to 0$. Тогда, в силу линейности:
\\
$f'(x)h+\omega(x,h)\to 0\Rightarrow f(x+h)\to f(x) \Rightarrow$ функция непрерывна в т. х.
\\
\dokno

\subsubsection{Следствие}
Если f(x) дифференцируема во всех точках Е, то она непрерывна на Е.

\begin{opred}
f(x) дифференцируема на множестве E, если $f:E\to\R^1$ - дифференцируема в любой точке E.
\end{opred}
\\
Примеры:
\\
$1.f:E\to\R^1$, где $E \subset\R^2, f(x^1,x^2)=(x^1)^2-(x^2)$
\\
$E \subset \R^2 \Rightarrow h = (h^1,h^2)$
\\
$f(x+h)-f(x)=(x^1+h^1)^2 - (x_2 + h_2) - (x^1)^2 + (x^2) = 2xh^1 - h_1^2 + (h^1)^2$
\\
$\frac{(h^1)^2}{||h||}=\frac{(h^1)^2}{\sqrt{(h^1)^2+(h^2)^2}}\le\frac{(h^1)^2}{|h^1|}\to 0$, при $h\to 0$, т.е.
\\
$(h^1)^2=o(||h||)=\omega(x,h)\Rightarrow$
\\
$\Rightarrow$ выполняется определение дифференцируемости по Фреше $\Rightarrow f(x_1,x_2)$ - дифференцируема на $\R^2$ и $d(x,h) = 2x^1h^1-h^2$
\\
2.Пусть l-линейный функционал, определенный на пр-ве $\R^n$. Тогда 
\\
$\forall(x,h\in\R^n)[l(x+h)-l(x)=l(h)=lh]$
\\
значит определение дифференцируемост по Фреше выполняется.
\\
$(\omega(x,h) = 0)\Rightarrow$ линейный функционал дифференцирован на $\R^n$ и $l'(x)h=lh\Rightarrow l'(x)=l$.
\\
Отсюда: производная функционала в любой точке х есть тот же самый функционал.
\\
3.Рассмотрим $f:\R^n\to\R^1, f(x)=||x||$.
\\
Докажем, что в точке х=0 f(x) - не дифференцируема.
\\
\dokvo
$f(o+h)-f(o)=lh+\omega(h)$, где $\omega(h)=o(||h||), h\to 0$,
\\
но $f(o+h)-f(o)=||o+h||-||o||=||h||$
\\
Пусть $\forall(h_0\in\R^n,h_0 \ne 0)\exists(t\in\R^1)[h=th_0].$
\\
Тогда $|t|\cdot ||h_0||=l(th_0)+\omega(th_0)=t\cdot l(h_0)+\omega(th_0)$
\\
$||h_0||=\frac{t}{|t|}l(h_0)+\frac{\omega(th_0)}{|t|}$
\\
$||h_0||-\frac{\omega(th_0)}{|t|}=\frac{t}{|t|}l(h_0)$
\\
$\lim_{t\to 0} (||h_0||-\frac{\omega (th_0)||h_0||}{|t|||h_0||})=\lim_{t\to 0}||h_0||=||h_0||$
\\
т.е. $\exists(\lim_{t\to 0} (||h_0||-\frac{\omega (th_0)||h_0||}{|t|||h_0||})=||h_0||)$, но $\urcorner\exists(\lim_{t\to 0} \frac{t}{|t|}l(h_0))$ Противоречие.
\\
\dokno










\subsection{Простейшие свойства операции дифференцирования}
\begin{teorema}
Если $f,g:E\to\R^1$ - дифференцируемы в $x\in E$, то:
\\
a) $\lambda f+\mu g$ (где $\lambda,\mu\in\R^1)$ - дифференцируема в х, и $[(\lambda f+ \mu g)'(x)=(\lambda f'+\mu g')(x)]$
\\
б) $f\cdot g$ - дифференцируема в x, и 
$[(f\cdot g)'(x)=(f'\cdot g)(x)+(f\cdot g')(x)]$
\\
в) $f/g$ (если $g(x)\ne 0$) - дифференцируема в х, и
\\
$[(f/g)'(x)=(\frac{f'\cdot g - g'\cdot f}{g^2})(x)]$
\end{teorema}

\dokvo
а) Возьмем $\forall$ приращение: $\forall(h\in\R^n:x+h\in E):$
\\
$$(\lambda f+\mu g)(x+h)-(\lambda f+\mu g)(x)=$$
$$=(\lambda\cdot f(x+h)-\lambda\cdot f(x))+(\mu\cdot g(x+h)-\mu\cdot g(x))=$$
$$=\lambda(f(x+h)-f(x))+\mu(g(x+h)-g(x))=$$
$$=\lambda(f'(x,h)+\omega_1(x,h))+\mu(g'(x,h)+\omega_2(x,h))=$$
$$=\lambda f'(x,h)+\mu g'(x,h)+(\lambda\omega_1(x,h)+\mu\omega_2(x,h))=$$
$$=(\lambda f'(x)+\mu g'(x))(h)+\omega(x,h).$$
\\
$\lim_{h\to 0}\omega(x,h)=\lim_{h\to 0}(\lambda\omega_1(x,h)+\mu\omega_2(x,h))=0\Rightarrow$
\\
$\Rightarrow \omega(x,h)=o(||h||)$
\\
Значит, $\lambda f+ \mu g$ - дифференцируема в х, и $(\lambda f+ \mu g)'=\lambda f'+ \mu g'$ 
\\
\dokno
\\
б) аналогично пункту а)
\\
в) достаточно будет доказать, что $\frac{1}{g}$ - дифференцируема в х, $(\frac{1}{g})'(x) = -\frac{g'(x)}{g^2(x)}$, и воспользовать пунктом б) данной теоремы:
\\
$(\frac{f}{g})'(x) = (f\cdot\frac{1}{g})'(x) = (f'\cdot\frac{1}{g})(x)+(f\cdot\frac{-g'}{g^2})(x) = (\frac{f'g-g'f}{g^2})(x)$
\\
докажем вышесказанное:
$\forall(h\in\R^n:x+h\in\R^2)$
\\
$$
[\frac{1}{g(x+h)}-\frac{1}{g(x)}=-\frac{g(x+h)-g(x)}{g(x+h)\cdot g(x)} = -\frac{g'(x)+\omega(x,h)}{g(x+h)\cdot g(x)}]
$$
где $\omega(x,h)=o(||h||)$, при $h\to 0$.
\\
Функция g - дифференцируема $\Rightarrow$ непрерывна в х.
\\
Поэтому $\frac{1}{g(x+h)\cdot g(x)}\to\frac{1}{g^2(x)}$ при $h\to 0$.
\\
$\frac{1}{g(x+h)\cdot g(x)}\to\frac{1}{g^2(x)}+\alpha(x,h)$, где $\alpha(x,h)\to 0, h\to 0$
\\
Отсюда:
\\
$\frac{1}{g(x+h)}-\frac{1}{g(x)}=-((g'(x)+\omega(x,h))\cdot\frac{1}{g(x+h)\cdot g(x)})=$
\\
$=-(g'(x)h+\omega(x,h))\cdot(\frac{1}{g^2(x)}+\alpha(x,h))=-\frac{g'(x)h}{g^2(x)}+\omega_1(x,h)$
\\
где $\lim_{h\to 0}\omega_1(x,h)=\lim_{h\to 0}(g'(x)h\cdot\alpha(x,h)+\omega(x,h)(\frac{1}{g^2(x)}+\alpha(x,h)))=0$
\\
$\lim_{h\to 0}\frac{\omega_1(x,h)}{||h||}=\lim_{h\to 0}(g'(x)\cdot\frac{h}{||h||}\cdot\alpha(x,h)+\frac{\omega(x,h)}{||h||}(\frac{1}{g^2(x)}+\alpha(x,h)))=0$
\\
Значит, $\omega_1(x,h)=o(||h||)$, при $h\to 0\Rightarrow$
\\
$\Rightarrow\frac{1}{g(x)}$ - дифференцируема в х, и $(\frac{1}{g})'(x)=-\frac{g'(x)}{g^2(x)}$
\\
\dokno









\subsection{Определение производной по направлению. Связь между понятиями дифференцируемости функции по Фреше и Гато}
\begin{opred}
Пусть Е - открытое множество, $E\subset\R^n$
\\
Пусть $f:E\to\R^1,l=(l^1,...,l^n)\in\R^n$ - фиксированный вектор единичной длины (||l||=1), $x\in E$ - фиксированно.
\\
Тогда, при достаточно малых $t:x+tl\in E$
\\
$\lim_{t\to 0}\frac{f(x+tl)}{t}$ (если этот lim существует) - производная функции f в т. х по направлению l.
\end{opred}
\\
Часто производную по направлению $h\ne 0$ определяют как $\lim_{t\to 0} \frac{f(x+th)-f(x)}{t}$, не предполагая, что ||h||=1. В этом случае:
$$
f_h'(x)=||h||\cdot f_{h1}'(x)
$$
где $h_1=\frac{h}{||h||}$
\\
\dokvo
$$
f_h'(x)=\lim_{t\to 0}\frac{f(x+th)-f(x)}{t}=\lim_{t\to 0}\frac{f(x+h_1\cdot||h||\cdot t)-f(x)}{t}=
$$

$$
=||h||\cdot\lim_{t\to 0}\frac{f(x+||h||t\cdot h_1)-f(x)}{||h||t}=<k=t\cdot ||h||;k\to 0; t\to 0>=
$$

$$
=||h||\cdot\lim_{t\to 0}\frac{f(x+k\cdot h_1)-f(x)}{k}=||h||\cdot f_{h_1}'(x)
$$
\dokno

\begin{opred}
Функция, дифференцируемая по Гато в т.х - функция дифференцируемая в т. х по $\forall$ направлению.
\end{opred}

Из дифференцируемости по Гато функции в некоторой точке НЕ СЛЕДУЕТ непрерывность в точке, как это было для дифференцируемости по Фреше.
\subsubsection{Пример:}

$f:\R^2\to\R^1:f(x^1,x^2) = 
\left\{\begin{array}{c c}
0 & ,(x^1,x^2)=0 \\
\frac{(x^1)^3 \sqrt[4]{(x^2)^2}}{(x^1)^4+(x^2)^2} & ,(x^1,x^2)\ne 0
\end{array}\right.$

f(0;0)=0
\\
в качестве $x^2$ возьмём $(x^1)^2 (x^2)=(x^1)^2)-$ парабола)
\\
$\lim_{x^1\to 0}\frac{(x^1)^3\cdot |x^1|}{(x^1)^4+(x^1)^4}=\lim_{x^1\to 0}\frac{|x^1|}{2x^1}=\pm\frac{1}{2}=\left[\begin{array}{c c}
0,5 & ,x^1\to +0 \\
-0,5 & ,x^1\to -0
\end{array}\right.$
\\
$\lim_{x^1\to -0} f(x^1,(x^1)^2)=-\frac{1}{2}, \lim_{x^1\to +0} f(x^1,(x^1)^2)=\frac{1}{2}, f(0,0)=0\Rightarrow$
\\
$\Rightarrow f(x^1,(x^1)^2)$ - разрывна в (0;0).
\\
Покажем, что $\forall(h\in \R^2, h(h_1;h_2))\exists(f_n'(0;0)):f_n'(0;0) = $
\\
$ = \lim_{t\to 0}\frac{1}{t}(f(0+th)-f(0))=\lim_{t\to 0}\frac{f(th)}{t} = $
\\
$ = \lim_{t\to 0}\frac{(th_1)^3\sqrt[4]{(th_2)^2}}{(th_1)^4+(th_2)^2}\cdot \frac{1}{t} = \lim_{t\to 0}\frac{(h_1)^3\sqrt[4]{t^2\cdot (h_2)^2}}{t^2(h_1)^4+(h_2)^2}=0,$ т.е.
\\
$\forall(h\in\R^2:h=(h_1;h_2))\exists(f_n'(0;0)=0)$
\\
$\Rightarrow f(x^1,x^2)$ - дифференцируема по Гато и разрывна в точке (0;0).

\begin{teorema}
Связь между дифференцируемостью функции по Фреше и Гато:
\\
если f дифференцируема по Фреше в $x_0$, то она дифференцируема и по Гато в $x_0$ т.е.
$$
\forall(h\in\R^n)\exists(f_h'(x_0))[f_h'(x_0)=df(x_0,h)]
$$
\end{teorema}
\dokvo
Дано: f - дифференцируема по Фреше. Доказать: 
\\
$\exists(\lim_{t\to 0}\frac{f(x_0+th)-f(x_0)}{t}).$
\\
$\lim_{t\to 0}\frac{f(x_0+th)-f(x_0)}{t}=\lim_{t\to 0}\frac{f'(x)th+\omega(x,th)}{t}, (\omega(x,th)=o(||th||)=$
\\
$
= \lim_{t\to 0}(f'(x)h+\frac{\omega(x,th)}{t})=\lim_{t\to 0}(f'(x)h+\frac{\omega(x,th)}{||th||}\cdot\frac{||th||}{|t|})=
$
\\
$
= \lim_{t\to 0}f'(x)h=f'(x)h=df(x,h)\Rightarrow
$
\\
$
\Rightarrow\exists(f_h'(x))\forall(h)[f_h'(x)=df(x,h)]
$
\\
\dokno
Обратная теорема НЕВЕРНА!!! (см.пример)











\subsection{Теорема Лагранжа}
\begin{opred}
Пусть $x,y\in\R^n$. Тогда множество точек $\R^n$ вида $\{\lambda x+(1-\lambda)y, \lambda\in[0;1]\}$ называется отрезком, соединяющим точки х и у ([x;y]).
\\
Если Е - открытое множество из $\R^n:E\subset\R^n,$
\\
$\forall(x\in E)\forall(h\in\R^n)\exists(\delta > 0)\forall(\alpha:|\alpha| < \delta)[[x;x+\alpha h]\subset E]$
\end{opred}

\begin{teorema}
Аналог теоремы Лагранжа для функции многих переменных:
Пусть $f:E\to\R^1$ имеет в $\forall(x\in E)\exists(f_h'(x))\forall(h).$
\\
Тогда $\forall(h\in\R^n)\forall(\delta > 0)\forall(\alpha:|\alpha| < \delta)[[x;x+\alpha h]\subset E]$
\\
справедлива формула Лагранжа:
\\
$f(x+\alpha h)-f(x)=\alpha f_h'(x+Q\alpha h)$, где
\\
$Q=(x,\alpha,h),$ т.е. Q зависит от $x,\alpha,h$ и $Q\in (0;1)$
\end{teorema}
\dokvo
Введём $\varphi(t)=f(x+th)$, где $t\in(-\delta;\delta)$
\\
Исследуем $\varphi(t)$ - функцию 1 переменной, докажем что $\varphi(t)$ - дифференцируема и $\varphi '(t) = f_h'(x+th)$
\\
Рассмотрим $\forall(t\in(-\delta;\delta)):$
\\
$\varphi ' = \frac{\varphi(t+\vartriangle t)-\varphi(t)}{\vartriangle t} = \frac{f(x+(t+\vartriangle t)h)-f(x+th)}{\vartriangle t}=\frac{f((x+th)+\vartriangle th)-f(x+th)}{\vartriangle t}$
\\
т.к. дано, что $\forall(h)\exists(f_h'(x))$, то 
$$
\exists(\lim_{\vartriangle t\to 0}\frac{f(x+th+\vartriangle th)-f(x+th)}{\vartriangle t}=f_h'(x+th))
$$
\dokno
\\
Значит, $\varphi (t)$ - дифференцируема для $\forall(t\in(-\delta;\delta))$
\\
Возьмём $\forall(\alpha:|\alpha|<\delta).$ Тогда $\varphi(t)$ дифференцируема на $[-\alpha;\alpha]$. Поэтому, т.к. $\varphi (t)$ - скалярная функция 1 переменной, дифференцируемая на отрезке, то к ней можно применить теорему Лагранжа.
\\
Возьмем $[0;\alpha]$ по теореме Лагранжа для функции скалярного аргумента:
$$
\exists(\theta\in(0;1))[\varphi(\alpha)-\varphi(o)=(\alpha - 0)\varphi'(\theta\alpha)=\alpha\varphi'(\theta\alpha)]
$$ 
т.к. $\varphi(\alpha)=f(x+\alpha h), \varphi (0) = f(x), \varphi'(\theta\alpha)=f_h'(x+\theta\alpha h),$
\\
то $f(x+\alpha h) - f(x) = \alpha\cdot f_h'(x+\theta\alpha h)$
\dokno








\subsection{Частные производные скалярных функций векторного аргумента. Связь между существованием частных производных и дифференцируемостью функции по Фреше и Гато}
E - открытое множество, $E\subset\R^n$
\\
\begin{opred}
Частные производные функции f в т. $x_0$ по переменным $x^1,...,x^n$ - производная функции $f:E\to\R^1$ в т. $x_0\in E$ по направлениям ортов.
($\frac{\delta f}{dx^1}(x_0),...\frac{\delta f}{dx^n}(x_0)$, или $f_{x^1}'(x_0),.. f_{x^n}'(x_0)$)
\\
(иногда штрихи опускаются), или
\\
$D_1 f(x_0),... D_n f(x_0)$, или $\delta_1 f(x_0),... \delta_n f(x_0)$
\end{opred}

Согласно определениям:
$\frac{\delta f}{\delta x^i}(x_0) = \lim_{t\to 0}\frac{f(x_0+te_i)-f(x_0)}{t} = $
\\
$ = \lim_{t\to 0}\frac{f(x_0^1,...,x_0^{i-1},x_0^i+t,x_0^{i+1},...,x_0^n)}{t}$
\\
Иногда вместо t пишут $\vartriangle x^i$, подчёрквивая, что приращение получает лишь i-тая координата.
\\
Определение показывает, что для нахождения частной производной нужно все координаты точек, кроме i-той, рассматривать как const и находить производную от функции f по переменной $x^i$ как от функции одной переменной. Таким образом, нахождение частных производных производится по обычным правилам дифференцирования вещественной функции вещественной переменной.
\subsubsection{Пример:}
$f:\R^2\to \R^1, f(x^1,x^2)=(x^1)^2,$ где $x^1>0.$
\\
Тогда $\frac{\delta f}{dx^1}(x) = x^2\cdot(x^1)^{x^2-1},$ а $\frac{\delta f}{\delta x^2}(x)=\frac{(x^1)^{x^2}}{ln (x^2)}$
\\
\begin{teorema}
Если $f:E\to\R^1$ - дифференцируема в т. $x_0$ по Фреше, то в т. $x_0 \exists$ все частные производные, т.е.
$$
\exists(\frac{\delta f}{\delta x^i}(x_0), i\in\{1,..n\})[df(x_0,h)=\sum_{i=1}^{n}\frac{\delta f}{\delta x^i}(x_0)h^i]
$$  
\end{teorema}
\dokvo
1. f - дифференцируема в $x_0$ по Фреше $\Rightarrow$ f - дифференцируема в $x_0$ по Гато (т.е.$\forall(h\in\R^n)\exists(f_h')$) $\Rightarrow$
\\
$\Rightarrow f$ - дифференцируема и по ортмам $\Rightarrow \exists(\frac{\delta f}{\delta x^i}(x_0))\forall(i\in\{1,..n\})$
\\
2. т.к. f - дифференцируема по Фреше, то 
$$
f(x_0+h)-f(x_0) = df(x_0,h)+\omega(x_0,h)
$$
где $\omega(x_0,h) = o(||h||)$ при $h\to 0$
\\
$df(x_0,h) = <u,h>$, где $u\in\R^n$ - элемент, определяющий линейный функционал.
$$
f(x_0+h) - f(x_0) = <u,h> + \omega(x_0,h) = \sum_{i=1}^{n}u^i h^i + \omega(x_0,h)
$$
Выберем $h^{i_0}\ne 0, i_0\in[1;n]$ так, чтобы 
$$
h_0 = (0,0,...,0,h^{i_0},0,...,0)
$$

Подставим $h_0$ вместо h:
$$
\frac{f(x_0^{1},...,x_0^{i0-1},x_0^{i0}+h^{i0},...,x_0^{i0+1},...,x_0^n) - f(x_0^1,...,x_0^n)}{h^{i0}} = 
$$

$$
= \frac{u^{i0}\cdot h^{i0}}{h^{i0}}+\frac{\omega(x_0,h_0)}{h^{i0}} = u^{i0} + \frac{\omega(x_0,h_0)}{h^{i0}}
$$

$$
\lim_{h_0\to 0}\frac{f(x_0+h_0)-f(x_0)}{h^{i0}}=\lim_{h_0\to 0}(u^{i0}+\frac{\omega(x_0,h_0)}{h^{i0}}) = u^{i0}
$$
Отсюда:
$$
\lim_{h_0\to 0}\frac{f(x_0+h_0)-f(x_0)}{h^{i0}} = \frac{\delta f}{\delta x^{i0}}(x_0) = u^{i0}
$$

$$
df(x_0,h) = <u,h> = \sum_{i=1}^{n}\frac{\delta f}{\delta x^i}(x_0)h^i
$$
\dokno
\subsubsection{Замечание:}
часто приращение $h=(h^1,...,h^n)$ обозначают через $dx = (dx^1,...,dx^n).$ Тогда 
$$
df(x_0,h)=\sum_{i=1}^{n}\frac{\delta f}{\delta x^i}(x_0)dx^i
$$
\begin{teorema}
Связь между существованием частных производных и дифферецируемостью.
\\
Если $f:E\to\R^1, \forall(x\in U(x_0))\exists(\frac{\delta f}{\delta x^i}(x), i\in\{1;n\})$ и $\frac{\delta f}{\delta x^i}(x), i\in\{1;n\}$ - непрерывны в $x_0$ тогда в $x_0$
\\
$\forall(l\in\R^n)\exists(f_l'(x_0))[f_l'(x_0)=\sum_{i=1}^{n}\lambda^i \frac{\delta f}{\delta x^i}(x_0)],$ где $\lambda^i = <l;e_i> - $ i-тая координата вектора l.  
\end{teorema}
\dokvo
$$
\frac{f(x_0+tl) - f(x_0)}{t} = \frac{1}{t}[(f(x_0+\sum_{k=1}^{n}t\lambda^k e_k) - f(x_0+\sum_{k=2}^{n}t\lambda^k e_k))+(f(x_0+\sum_{k=2}^{n}t\lambda^k e_k) - f(x_0+\sum_{k=3}^{n}t\lambda^k e_k))+...+ (f(x_0+t\lambda^n e_n) - f(x-0))] =
$$

$$
= \sum_{i=1}^{n}\frac{1}{t}[f(x_0+\sum_{k=i}^{n}t\lambda^k e_k) - f(x_0+\sum_{k=i+1}^{n}t\lambda^k e_k)]
$$
частичное покоординатное приращение

$$
= \sum_{i=1}^{n}\lambda^i\frac{\delta f}{\delta x^i}(x_0+\sum_{k=i+1}^{n}t\lambda^k e_k + \theta t \lambda^i e_i)
$$
где $\theta\in(0;1)$
\\
$\lim_{t\to 0}(\sum_{k=i+1}^{n}t\lambda^k e_k +\theta t\lambda^i e_i)=0.$
\\
т.к. в $x_0$ все частные производные непрерывны, то при $t\to 0$, в силу непрерывности функции,

$$
\lim_{t\to 0}\sum_{i=1}^{n}\lambda^i\frac{\delta f}{\delta x^i}(x_0+\sum_{k=i+1}^{n}t\lambda^k e_k+\theta t\lambda^i e_i) = \sum_{i=1}^{n}\lambda^i \frac{\delta f}{\delta x^i}(x_0)
$$
Отсюда:
$$
\lim_{t\to 0}\frac{f(x_0+tl)-f(x_0)}{t}=\sum_{i=1}^{n}\lambda^i\frac{\delta f}{\delta x^i}(x_0)\Rightarrow
$$
по определению:
$$
\forall(l\in\R^n)\exists(f_l'(x_0))[f_l'(x_0)=\sum_{i=1}^{n}\lambda^i\frac{\delta f}{\delta x^i}(x_0)]
$$
\dokno
\subsubsection{Замечание 1:}
если $||l||=1, ||e_i|| =1,$ то, в $\R^3 \lambda^i - $ направляющие косинусы ($l,e_i$), т.к.
$$
\lambda^i = <l, e_i> = ||l||\cdot ||e_i||\cdot cos(l,e_i) = cos (l, e_i)
$$
По аналогии с $\R^3: \lambda^i = cos(\lambda,e_i)$.
\\
Тогда: $f_l'(x_0) = \sum_{i=1}^{n}\frac{\delta f}{\delta x^i}(x_0)\cdot cos(l,e_i)$

\subsubsection{Замечание 2:}
если в некоторой точке $x_0 \forall(i\in\{1;n\})\exists(\frac{\delta f}{\delta x^i}(x_0)),$ то функция может не иметь производной в $x_0$ по какому-нибудь направлению.
\subsubsection{Пример:}
$f:\R^2\to\R^1.$ $f(x)=f(x^1,x^2)=\left\{\begin{array}{c c}
0 & ,x^1\cdot x^2 = 0 \\
1 & ,x^1\cdot x^2 \ne 0
\end{array}\right.$

$$
\frac{\delta f}{\delta x^1}(0;0) = \lim_{\vartriangle x^1\to 0}\frac{f(\vartriangle x^1,0)-f(0,0)}{\vartriangle x^1} = 0
$$

$$
\frac{\delta f}{\delta x^2}(0;0) = \lim_{\vartriangle x^2\to 0}\frac{f(\vartriangle x^2,0)-f(0,0)}{\vartriangle x^2} = 0
$$
существуют все частные производные. Но!
\\
пусть $l=(\frac{1}{\sqrt{2}};\frac{1}{\sqrt{2}})$. Тогда
$$
\lim_{t\to 0}\frac{f(\frac{1}{\sqrt{2}};\frac{1}{\sqrt{2}}) - f(0,0)}{t}= \lim_{t\to 0} \frac{1}{t}
$$
Нет конечного предела следовательно $f_l'(0,0)$
\\
Значит, существования всех частных производных недостаточно для существования всех производных по направлению.

\begin{teorema}
Дифференцируемость функции по Фреше
Если $f:E\to\R^1$ имеет в $U(x_0)$ все частные производные, непрерывные в $x_0$, то функция f - дифференцируема по Фреше и $df(x_0,h)=\sum_{i=1}^{n}\frac{\delta f}{\delta x^i}(x_0)h^i$
\end{teorema}
\dokvo
Возьмём приращение $h\in\R^n:x_0+h\in E, h =\sum_{k=1}^{n}h^k e_k$
\\
$$f(x_0+h)-f(x_0) = f(x_0+\sum_{k=1}^{n}h^k e_k) - f(x_0+\sum_{k=2}^{n}h^k e_k)+...+f(x_0+\lambda^n e_n) - f(x_0)=$$

$$
=\sum_{i=1}^{n}(f(x_0+\sum_{k=i}^{n}h^k e_k)-f(x_0+\sum_{k=i+1}^{n}h^k e_k)) = 
$$
по теореме Лагранжа
$$
=\sum_{i=1}^{n}\frac{\delta f}{\delta x^i}(x_0+\sum_{k=i+1}^{n}h^k e_k + \theta h^i e_i)\cdot h^i,
$$
где $\theta\in(0;1)$
\\
Отсюда:
$$
f(x_0+h)-f(x_0) = \sum_{i=1}^{n}\frac{\delta f}{\delta x^i}(x_0)h^i + \sum_{i=1}^{n}(\frac{\delta f}{\delta x^i}(x_0+\sum_{k=i+1}^{n}h^k e_k +\theta h^i e_i)- \frac{\delta f}{\delta x^i}(x_0))h^i
$$

$$
\lim_{h\to 0}\frac{\delta f}{\delta x^i}(x_0+\sum_{k=i+1}^{n}h^k e_k + \theta h^i e_i) = \frac{\delta f}{\delta x^i} (x_0)
$$

Значит:
$$
\sum_{i=1}^{n}(\frac{\delta f}{\delta x^i}(x_0+\sum_{k=i+1}^{n}h^k e_k + \theta h^i e_i) - \frac{\delta f}{\delta x^i}(x_0))\cdot h^i = o(||h||) = \omega(x_0,h)
$$

Отсюда:
$$
f(x_0+h)-f(x_0) = \sum_{i=1}^{n}\frac{\delta f}{\delta x^i}(x_0)\cdot h^i+\omega(x_0,h) - 
$$
определение дифференцируемости по Фреше $\Rightarrow$
\\
$$
\Rightarrow df(x_0,h) = \sum_{i=1}^{n}\frac{\delta f}{\delta x^i}(x_0)h^i
$$
\dokno
\subsubsection{Следствие:}
Если функция $f:E\to\R^1$ имеет в $U(x_0) \\
\forall(l\in\R^n)\exists(f_e'(x) -$ непрер. в $x_0, x\in U(x_0), x_0\in E)$
\\
то f - дифференцируема в $x_0$ по Фреше и $df(x_0,h)=f_h'(x_0)$

\begin{opred}
Градиент функции f в $x_0$ - n-мерный вектор, координаты которого равны частным производным функции $f:E\to\R^1$
($grad f(x_0) = (\frac{\delta f}{\delta x^1}(x_0),...,\frac{\delta f}{\delta x^n}(x_0))$ и $df(x_0,h)=<grad f(x_0),h>$)
\end{opred}









\subsection{Теорема о дифференцируемости сложной функции и следствие из неё}
\begin{teorema}
	Пусть $E\subset\R^n$ - открытое, $f:E\to\R^1, \varphi^i:H\to\R^1, H\subset\R^m, (\varphi^1(t),...,\varphi^n(t))\subset E \forall(t\in H)$
	\\
	Пусть в некотором $t_0\in H, \varphi^i(t)$ - дифференцируема, а в $x_0=(\varphi^1(t_0),...,\varphi^n(t_0))$ - дифференцируема f(x).
	\\
	Тогда сложная функция $g(t)=f(\varphi^1(t),...,\varphi^n(t))$ - дифференцируема в $t_0$ и $dg(t_0,k) = \sum_{j=1}^{m}(\sum_{i=1}^{n}\frac{\delta f}{\delta x^i}(x_0)\cdot \frac{\delta \varphi}{\varphi t^j}(t_0))\cdot k^j$, где $k=(k^1,...,k^m)\in\R^m$
\end{teorema}

\dokvo
Возьмём приращение $k=(k^1,...,k^m)\in\R^n:t_0+k\in H$
\\
$$
g(t_0+k) - g(t_0) = f(\varphi^1(t_0+k),...,\varphi^n(t_0+k)) - f(\varphi^1(t_0),...,\varphi^n(t_0))=
$$

$$
= f(x_0^1+h^1,...,x_0^n+h^n)-f(x_0^1,...,x_0^n)
$$
где $x_0^i = \varphi^i(t_0), h^i = \varphi^i(t_0+k)-\varphi_i(t_0)$
\\
В силу дифференцируемости функци f имеем:
$$
f(x_0+h)-f(x_0) = \sum_{i=1}^{n}\frac{\delta f}{\delta x^i}(x_0)h^2+\omega(x_0,h)
$$
где $\frac{\omega(x,h)}{||h||}\to 0$ при $h\to 0$
\\
Из дифференцируемости функции в $x_0$ вытекает:
$$
h^i = \varphi_i(t_0+k)-\varphi_i(t_0) = \sum_{j=1}^{m}\frac{\delta \varphi^i}{\delta t^j}(t_0)k^j+\omega_1(t_0,k)
$$
где $\frac{\omega_1(t_0,k)}{||k||}\to 0$ при $k\to 0$
\\
Получим (т.к. функции $\varphi'(t), i=\{1;n\}$ дифф-емы в $t_0$):
$$
g(t_0+k)-g(t_0) = \sum_{i=1}^{n}\frac{\delta f}{\delta x^i}(x_0)(\sum_{j=1}^{m}\frac{\delta\varphi^i}{\delta t^j}(t_0)+\omega_1)+\omega=
$$

$$
\sum_{j=1}^{m}(\sum_{i=1}^{n}\frac{\delta f}{\delta x^i}(x_0)\frac{\delta\varphi^i}{\delta t^j}(t_0))k^j+\omega_2,
$$
где $\omega_2 = \omega_1\cdot\sum_{i=1}^{n}\frac{\delta f}{\delta x^i}(x_0)+\omega$
\\
Покажем, что $\omega_2=o(||k||)$:
$$
\frac{\omega_2}{||k||}=\frac{\omega_1}{||k||}\cdot\sum_{i=1}^{n}\frac{\delta f}{\delta x^i}(x_0)+\frac{\omega}{||k||}
$$
Значит
$$
\lim_{k\to 0}\frac{\omega_2}{||k||} = \lim_{k\to 0}(\frac{\omega_1}{||k||}\cdot\sum_{i=1}^{n}\frac{\delta f}{\delta x^i}(x_0)+\frac{\omega}{||k||})=\lim_{k\to 0}\frac{\omega(x,h)}{||k||}
$$

$$
\lim_{k\to 0}\frac{\omega(x,h)}{||h||}\cdot\frac{||h||}{||k||} = 0 \leftrightarrow
$$

$
\leftrightarrow \frac{||h||}{||k||}-
$
ограничено.
\\
а) в силу непрерывности $\varphi^i,$ при $k\to 0$
\\
$\forall(i\in\{1;n\})[h\to 0]\Rightarrow h\to 0$ т.е.

$$
\lim_{k\to 0}\frac{\omega(x,h)}{||h||}= \lim_{h\to 0}\frac{\omega(k,h)}{||h||}=0
$$

б) докажем, что $\frac{||h||}{||k||}$ - ограничено при $k\to 0$ - огр.
\\
Тогда $g(t_0+k)-g(t_0)=\sum_{j=1}^{m}(\sum_{i=1}^{n}\frac{\delta f}{\delta x^i}(x_0)\frac{\delta\varphi^i}{\delta t^j}(t_0))k^j+\omega_2$,
\\
где $\omega_2=(||k||).$ Тогда, по определению Фреше:
$$
dg(t_0,k)=\sum_{j=1}^{m}(\sum_{i=1}^{n}\frac{\delta f}{\delta x^i}(x_0)\frac{\delta\varphi^i}{\delta t^j}(t_0))k^j
$$
\dokno

\subsubsection{Следствие 1.}
для выполнения условий теоремы для частных производных функции имеет место формула:
$$
\frac{\delta g}{\delta t^k}(t_0) = \sum_{i=1}^{n}\frac{\delta f}{\delta x^i}(x_0)\frac{\delta\varphi^i}{\delta t^k}(t_0), k\in\{1;m\}
$$

\subsubsection{Замечание:}
можно доказать, что последняя формула верна и при более слабыэ утверждениях, чем условие теоремы. Достаточно требовать, чтобы f - дифференцируема в $x_0$, a $\varphi^i$ имели все частные производные в $t_0.$ Однако, одного существования частных производных функции f в $x_0$ и функции $\varphi^i$ в т. $t_0$ недостаточно для справедливости последней формулы.

\subsubsection{Пример:}
$$
f(x)=f(x^1,x^2)=\left\{\begin{array}{c c}
0 & ,(x^1,x^2) = 0\\
\frac{(x^1)^2\cdot x^2}{(x^1)^2+(x^2)^2} & ,(x^1,x^2)\ne 0
\end{array}\right.
$$ 
функция имеет частные производные во всех точках:

$$
\frac{\delta f}{\delta x^1}(0,0) = \lim_{\vartriangle x^1\to 0}\frac{f(\vartriangle x^1,0) - f(0,0)}{\vartriangle x^1}=0
$$

$$
\frac{\delta f}{\delta x^2} = \lim_{\vartriangle x^2\to 0}\frac{f(0,\vartriangle x^2) - f(0,0)}{\vartriangle x^2}=0
$$
Введём новую переменную $t: x^1=t, x^2=t\Rightarrow$
\\
$\Rightarrow g(t)=f(t,t)$ - сложная функция от t:
$$
g(t) = \frac{t^3}{2t^2} = \frac{1}{2}t
$$
g(t) - дифференцируема при $\forall(t).$
\\
$g'(t) = \frac{1}{2}$
\\
Но! $g_1'(0)=f_{x^1}'(0)-(x^1)(0)+f_{x^2}'(0)\cdot(x^2)(0) = 0 \ne \frac{1}{2}$

\subsubsection{Следствие 2}
Инвариантность формы первого дифференциала $f:E\R^1, E\subset\R^n - $ открытое, f - дифференцируема в $x\in E,$ и $\varphi^i:H\to\R^1,h\subset\R^m$ - открытое, $\varphi^i$ - дифференцируема в т. $t\in H: x =(\varphi^1(t),...,\varphi^n(t))$.
\\
Тогда:
$$
df(x,dx) = \sum_{i=1}^{n}\frac{\delta f}{\delta x^i}(x)dx^i,
$$
где $dx^i = d\varphi^i(t,dt),$
\\
т.е. дифференциал имеет тот же вид, что и в случае, когда х - независимая перенная.
\dokvo
$$g(t) = f(\varphi^1(t),...,\varphi^n(t)) = f(x).$$
Тогда
$$
dg(t,dt) = \sum_{k=1}^{m}\frac{\delta q}{\delta t^k}(t)dt^k = \sum_{k=1}^{m}(\sum_{i=1}^{n}\frac{\delta f}{\delta x^i}(x)\frac{\delta\varphi^i}{\delta t^k}(t))dt^k
$$
Поменяем порядок суммирования:
$$
\sum_{i=1}^{n}\frac{\delta f}{\delta x^i}(x)\sum_{k=1}^{m}\frac{\delta\varphi^i}{\delta t^k}(t)dt^k = \sum_{i=1}^{n}\frac{\delta f}{\delta x^i}(x)dx^i \Rightarrow
$$

$$
\Rightarrow df(x,dx) = dg(t,dt) = \sum_{i=1}^{n}\frac{\delta f}{\delta x^i}(x)dx^i
$$
т.е. дифференциал имеет тот же вид, что и в случае когда t была независимой переменной 
\dokno
\\
форма 1 дифференциала не зависит от того, является ли х зависимой или независимой переменной
$$
dg(t,dt)=df(x,dx)=\sum \frac{\delta f}{\delta x^i}(x)dx^i
$$




\subsection{Инвариантность формы первого дифференциала}
\subsection{Частные производные высших порядков}
\begin{opred}
	$f:E\to\R^1$.Если функция f имеет в Е частную производную $ \frac{\delta f}{\delta x^i}$, то эта частная производная сама является некоторой функцией $ \frac{\delta f}{\delta x^i}:E\to\R^1$. Эта функция, в свою очередь может иметь частную производную по некоторой переменной х, которая называется производной 2 порядка или второй частной производной по $x^i$ и $x^j$
	\\
	($\frac{\delta}{\delta x^j}(\frac{\delta f}{\delta x^i})(x_0) = \frac{\delta^2 f}{\delta x^i \delta x^j}(x_0)$ или $\delta_{ij}f(x_0)$, или $f_{x^j x^i}''(x_0)$)
\end{opred}
\\
Порядок индексов указывает в каком порядке производится дифференцирование по переменным.
\begin{opred}
	Если определена частная производная порядка $k: \frac{\delta^k f}{\delta x^{i_1},..,\delta x^{i_x}}$, то частная прозводная порядка к+1 определяется следующим соотношением:
	$$
	\frac{\delta^{k+1}f}{\delta x^i \delta x^{i_1} \delta x^{i_2}...\delta x^{i_k}}=\frac{\delta}{\delta x^i}(\frac{\delta^k f}{\delta x^{i_1}...\delta x^{i_k}})
	$$
\end{opred}
	
	\begin{teorema}
		Если функция $f:E\to\R^1$ имеет в некоторой окрестности т. $x_0\in E$ частную производную $\frac{\delta^2 f}{\delta x^i \delta x^j}(x)$ и $\frac{\delta^2 f}{\delta x^j \delta x^i}(x)$, которые непрерывны в $x_0$, то
		$$
		\frac{\delta^2 f}{\delta x^i \delta x^j}(x_0)=\frac{\delta^2 f}{\delta x^j \delta x^i}(x_0)
		$$
	\end{teorema}
	\dokvo
	Будем считать, что имеем дело с функциями двух переменных: $f(x^1,x^2).$ Надо доказать:
	$$
	\frac{\delta^2 f}{\delta x^1 \delta x^2}(x_0)=\frac{\delta^2 f}{\delta x^2 \delta x^1}(x_0)
	$$
	Рассмотрим вспомогательную функцию:
	$$
	F(h^1,h^2)=f(x_0^1+h^1,x_0^2+h^2)\cdot f(x_0^1+h^1,x_0^2)-f(x_0^1,x_0^2+h^2)+f(x_0^1,x_0^2),
	$$
	где $h=(h^1,h^2):x_0^1+h^1$ и $x_0^2+h^2$ не выходят за пределы окрестности, где существует производная.
	\\
	Пусть $\varphi(t)=f(x_0^1+th^1,x_0^2+h^2)-f(x_0^1+th^1,x_0^2).$
	\\
	Тогда $F(h^1,h^2)=\varphi(1)-\varphi(0)=\varphi'(\theta_1)$ - по теореме Лагранжа.
	\\
	По следствию 1 из теоремы о дифференцируемости сложной функции:
	$$
	F(h^1,h^2)=\frac{\delta f}{\delta x^1}(x_0^1+\theta_1 h^1,x_0^2+h^2)\cdot h^1\frac{\delta f}{\delta x^1}(x_0^1+\theta_1 h^1, x_0^2)h^1 =
	$$
	
	$$
	\frac{\delta^2 f}{\delta x^2 \delta x^1}(x_0^1+\theta_1 h^1,x_0^2+\theta_2 h^2)h^1 h^2
	$$
	где $\theta_2\in(0;1)$
	\\
	Пусть $\psi(t) = f(x_0^1+h^1, x_0^2+th^2) - f(x_0^1,x_0^2+th^2).$ Тогда
	\\
	$F(h^1,h^2)=\psi(1)-\psi(0) = \psi'(\theta_3)$ - по теореме Лагранжа
	\\
	$F(h^1,h^2) = \frac{\delta f}{\delta x^2}(x_0^1+h^1,x_0^2+\theta_3 h^2)h^2 - \frac{\delta f}{\delta x^2}(x_0^1,x_0^2+\theta_3 h^2)h^2 = $
	\\
	$ = \frac{\delta^2 f}{\delta x^1 \delta x^2}(x_0^1+\theta_4 h^1,x_0^2+\theta_3 h^2)h^1 h^2,$
	где $0<\theta_3\theta_4<1$
	\\
	Получается:
	$$
	\frac{\delta^2 f}{\delta x^2 \delta x^1}(x_0^1+\theta_1 h^1,x_0^2+\theta_2 h^2)h^1 h^2 = \frac{\delta^2 f}{\delta x^1 \delta x^2}(x_0^1+\theta_4 h^1, x_2+\theta_3 h^2) = F(h^1,h^2)
	$$
	
	$\frac{\delta^2 f}{\delta x^2 \delta x^1}$ и $\frac{\delta^2 f}{\delta x^1 \delta x^2}$ непрерывны в $x_0$
	\\
	Перейдем к $\lim_{h\to 0} F(h^1,h^2)$ и получим:
	$$
	\frac{\delta^2 f}{\delta x^2 \delta x^1}(x_0) =\frac{\delta^2 f}{\delta x^1 \delta x^2}(x_0) 
	$$
	\dokno
	
	\subsubsection{Замечание:}
	Если $\frac{\delta^2 f}{\delta x^2 \delta x^1}$ и $\frac{\delta^2 f}{\delta x^1 \delta x^2}$ - не непрерывны в $x_0$, то они могут оказаться неравными
	
	\subsubsection{Упражнение:}
	$f:\R^2\to\R^1$ 
	\\
	$$
	f(x^1,x^2)=\left\{\begin{array}{c c}
	x^1,x^2\cdot\frac{(x^1)^2 - (x^2)^2}{(x^1)^2+(x^2)^2} & ,(x^1,x^2)\ne 0 \\
	0 & ,(x^1,x^2)=0
	\end{array}\right.
	$$
	в этом случае $\frac{\delta^2 f}{\delta x^2 \delta x^1} \ne \frac{\delta^2 f}{\delta x^1 \delta x^2}$
	\\
	Если $f:E\to\R^1$ имеет $\forall$ частные производные до к - порядка включительно, которые непрерывны на Е, то значение к-той производной $\frac{\delta^k f(x^1,..,x^n)}{\delta x^{i_1} \delta x^{i_2},...,\delta x^{i_n}}$ не зависит от порядка $i_1,...,i_k$ остаётся прежним при $\forall$ перестановки индекса.
	Доказательство по индукции.
\subsection{Дифференциалы высших порядков}
\begin{opred}
Билинейная форма. Функция $A:\R^n\times\R^n\to\R^1,$ если
\\
$$1.A(\alpha x_1+\beta x_2,y)=\alpha A(x_1,y)+\beta A(x_2,y)$$
$$
\forall(\alpha,\beta\in\R^1; x_1,x_2,y\in\R^n)
$$

$$
2.A(x,\aleph y_1 +\beta y_2) = \alpha A(x,y_1)+\beta A(x,y_2)
$$
$$
\forall(\alpha,\beta\in\R^1; x,y_1,y_2\in\R^n)
$$
Если A(x,y) - билинейная форма, определённая на декартовом произведении $\R^n\times\R^n$ по базису $\{e_i\},$ то
$$
A(x,y) = \sum_{i=1}^{n}\sum_{j=1}^{n}x^i y^j ; x=\sum_{i=1}^{n}x^i e_i; y=\sum_{j=1}^{n} y^j e_j
$$
Пусть $a_{ij}=A(e_i,e_j).$ Тогда
$$
A(x,y) = \sum_{i=1}^{n}\sum_{j=1}^{n} a_{ij}x^i y^j
$$
\end{opred}

\begin{opred}
Пусть x=y. Тогда\\
$A(x,x) = \sum_{i=1}^{n}\sum_{j=1}^{n} a_{ij} x^i x^j$ - квадратичная форма, соответствующая билинейной форме $A(x,y)$
\end{opred}

\begin{opred}
Симметрическая билинейная форма $A(x,y)$ и $A(x,x)$ - такие, что
$$
\forall(i,j\in\{1;n\})[a_{ij}=a_{ji}]
$$
\end{opred}

\subsubsection{Пример:}
$<x,y> = \sum_{i=1}^{n}x^i y^i$ - симметрическая билинейная форма.
\\
$||x||^2 = \sum_{i=1}^{n}(x^i)^2$ - симметрическая квадратная форма.
\\
Дифференциал высшего порядка
\\
Пусть $E\subset\R^n$ - открытоею
\\
$f:E\to\R^n$ - дифференцируема на Е, т.е.$\exists(df(x,h))$
\\
Зафиксировав h, получаем скалярную функцию с 1 независимой переменной - х, которая определена на Е и $x+h\in E$
\\
Предположим, что, при фиксированном h, $\varphi(x)=df(x,h)$, как скалярная функция от х, дифференцируема в некоторой точке $x_0\in E\Rightarrow \exists$(линейный функционал $\lambda_{x_0}(x):\R^n\to\R^1)$
\\
$\forall(k\in\R^n:k+x_0\in E)[df(x_0+k,h)-df(x_0,h)=\lambda_{x_0}(h)k+\Omega]$, где $\lambda_{x_0}(h)k$ - линейно по к,
\\
$\Omega = o(||k||), $ т.е. $\frac{\Omega}{||k||}\to 0$ при $||k||\to 0$

\subsubsection{Утверждение:}
Покажем, что $\lambda_{x_0}(h)k$ линейно по h:
\\
т.е. $\forall(h_1,h_2\in\R^n, \alpha\in\R^1)$ выполняется следующее:
$$
1. \lambda_{x_0}(h_1+h_2)k=\lambda_{x_0}(h_1)k+\lambda_{x_0}(h_2)k
$$

$$
2. \lambda_{x_0}(\alpha h_1)k = \alpha\lambda_{x_0}(h_1)k
$$
\dokvo
$(h_1):\lambda_{x_0}(h_1)k+\Omega_1 = df(x_0+k,h_1) - df(x_0,h_1);$
\\
$(h_2):\lambda_{x_0}(h_2)k+\Omega_2 = df(x_0+k,h_2) - df(x_0,h_2);$
\\
$(h_1+h_2):\lambda_{x_0}(h_1+h_2)k+\Omega_3 = df(x_0+k, h_1+h_2) - df(x_0,h_1+h_2),$
где $\Omega_1=o(||k||), \Omega_2 = o(||k||), \Omega_3 = o(||k||)$
\\
$(h_1)+(h_2)-(h_1+h_2):$
\\
$$\lambda_{x_0}(h_1)k+\lambda_{x_0}(h_2)k-\lambda_{x_0}(h_1+h_2)k+(\Omega_1+\Omega_2-\Omega_3) = $$

$$
= df(x_0+k,h_1)-df(x_0,h_1)+df(x_0+k,h_2)-df(x_0,h_2)-df(x_0+k,h_1+h_2)+df(x_0,h_1+h_2)
$$

$$
0 = \lambda_{x_0}(h_1)k+\lambda_{x_0}(h_2)k-\lambda_{x_0}(h_1+h_2)k+(\Omega_1+\Omega_2-\Omega_3) = 
$$
Возьмем $\forall(k_0\in\R^n:k\ne 0).$ Если $k_0=0$, то равенство будет доказано.
\\
В качестве $k=\varepsilon\cdot k_0.$ Тогда, в силу линейности $\lambda_{x_0}:\lambda_{x_0}(h_1)\varepsilon k = \varepsilon\lambda_{x_0}(h_1)k$ Тогда:
$$
= \varepsilon(\lambda_{x_0}(h_1)k_0+\lambda_{x_0}(h_2)k_0-\lambda_{x_0}(h_1+h_2)k_0+\frac{\Omega_1+\Omega_2-\Omega_3}{\varepsilon})
$$
При $\varepsilon\to 0  k=\varepsilon\cdot k_0\to 0\Rightarrow$
\\
$\Rightarrow \frac{\Omega_1+\Omega_2-\Omega_3}{\varepsilon}\to 0$ при $\varepsilon\to 0$
\\
Тогда получаем:
$$
\forall(k_0\ne 0\in\R^n)[\lambda_{x_0}(h_1)k_0+\lambda_{x_0}(h_2)k_0-\lambda(h_1+h_2)k_0]
$$
т.к. $k_0$ может быть любым, то утверждение доказано.
(однородность докажется аналогично)
\dokno
\begin{opred}
Второй дифференциал (дифференциал второго порядка) - квадратичная форма 
$$
\lambda_{x_0}(h)h=d(df(x_0,h)(x_0),h),
$$
соответствующая билинейной форме $\lambda_{x_0}(h)k$ (по аналогии с $f(x_0+h)-f(x_0) = df(x_0,h)+\omega$)
$$
d^2f(x_0,h)=\lambda_{x_0}(h)(h).
$$
\end{opred}

\begin{opred}
Вторая производная функции f в $x_0$ - билинейное отображение, значением которого в точке (h,h) является второй дифференциал 
$$
f''(x_0)
$$
\end{opred}
Таким образом, $d^2f(x_0,h)=f''(x_0)(h,h).$
\\
По аналогии со скалярным случаем:
$$
f''(x_0)(h,h) = f''(x_0)h^2
$$
Так как 2 дифференциал - это квадратичная форма, то он представляется в виде:
$$
d^2f(x_0,h)=|\sum_{i=1}^{n}\sum_{j=1}^{n} a_{ij} h^i h^j|
$$
здесь $a_{ij}$ зависит от $x_0$
\\
Найдем $a_{ij}(x_0):$
$df(x,h)=\sum_{i=1}^{n}\frac{\delta f}{\delta x^2}(h^i)$
\\
В качестве h возьмем $h^0=(0,...,0,1,0...)$
\\
Подставим:
$$
df(x,h^0)=\frac{\delta f}{\delta x^k}(x)
$$
$df(x,h^0)$ - диф-ем как функцию от х в $x_0$
\\
Тогда $df(x,h^0)$ имеет в $x_0$ все частные производные. Значит,
$$
\exists(\frac{\delta}{\delta x^j}(\frac{\delta f}{\delta x^k})(x_0) = \frac{\delta^2 f}{\delta x^j \delta x^k}(x_0))
$$

$d^2 f(x,h^0) = \sum_{i=1}^{n}\frac{\delta}{\delta x^i}(\sum_{j=1}^{n}\frac{\delta f}{\delta x^j}(x)\cdot h^j )(x_0)h^i=$
\\
$\sum_{i=1}^{n}\sum_{j=1}^{n}\frac{\delta^2 f}{\delta x^i \delta x^j}(x_0)h^i h^j\Rightarrow$
\\
$\Rightarrow a_{ij}=\frac{\delta^2 f}{\delta x^i \delta x^j}(x_0)$
\\
Таким образом, если функция дважды дифференцируема в $x_0$, тогда второй дифференциал представляется в виде:
$$
d^2 f(x_0,h) = \sum_{i=1}^{n}\sum_{j=1}^{n}\frac{\delta^2 f}{\delta x^i \delta x^j}(x_0)h^i h^j
$$
Если, по аналогии со скалярным случаем, приращение обозначим как dx, то:
$$
d^2 f(x_0,dx) = \\sum_{i=1}^{n}\sum_{j=1}^{n}\frac{\delta^2 f}{\delta x^i \delta x^j}(x_0)dx^i dx^j
$$

Позже будет доказано, что, если функция дважды диф-ема в $x_0$, то верно, что
$$
\frac{\delta^2 f}{\delta x^i \delta x^j}(x_0)=\frac{\delta^2 f}{\delta x^j \delta x^i}(x_0)
$$
Если предположить, что любая частная производная f в $x_0$ - непрерывна, то функция f дважды диф-ема и имеет место последняя формула.
\subsubsection{Пример:}
$f:\R^2\to\R^1  f(x^1,x^2) = (x^1)^2\cdot(x^2)$
\\
$\frac{\delta f}{\delta x^1} = 2x^1 x^2; \frac{\delta f}{\delta x^2}=(x^1)^2; \frac{\delta^2 f}{d(x^1)^2} = 2x^2;\frac{\delta^2 f}{\delta (x^2)^2}=0;\frac{\delta^2 f}{\delta x^1 \delta x^2}=2x^1$

\begin{opred}
Трилинейная форма - отображение декартового произведения 3-х множтелей, которые являются линейными по каждому аргументу при фиксированных двух остальных:
$$
\forall(\alpha,\beta\in\R^1)\forall(x_1,x_2,y,z\in\R^n)
$$

$$
[A(\alpha x_1+\beta x_2,y,z) = \alpha A(x_1,y,z)+\beta A(x_2,y,z)]
$$
\end{opred}
Если $\forall(x\in E)\exists(d^2 f(x,h)),$ то дифференциал в т. $x_0\in E - $есть трилинейная форма от k,h,m:
\\
	k - приращение по x в $df(x,k)$
	\\
	h - приращение по x в $d^2 f(x,h)$
	\\
	m - приращение по y в $df(y,m)$
\\
\begin{opred}
Третий дифференциал или дифференциал третьего порядка функции f в $x_0$ - однородная форма, которая получается с трилинейной формы при k=h=m.
\\
$d^3 f(x_0,h)$
\end{opred}

\begin{opred}
Производная третьего порядка функции f в $x_0$ - трилинейное отображение, значением которого в т. (h,h,h) будет третий дифференциал.
\\
$f'''(x_0)$
\end{opred}
Таким образом, $d^3 f(x_0,h)=f'''(x_0)(h,h,h)$
\\
По аналогии со скалярной функцией:
$$
d^3 f(x_0,h) = f'''(x_0)(h,h,h)= f'''(x_0)h^3
$$
Продолжая, по индукции можно определить понятия дифференциала и производной любого порядка для скалярной функции векторного аргумента. Также, используя метод индукции, можно доказать, что:
$$
d^k f(x_0,h)=\sum_{i_1,...,i_{k+1}}^{n}\frac{\delta^k f}{\delta x^{i1}....\delta x^{ik}}(x_0)\cdot h^{i1}...h^{ik}
$$
или, если обозначить h за dx, то:
$$
d^k f(x_0,h)=\sum_{i_1,...,i_{k+1}}^{n}\frac{\delta^k f}{\delta x^{i1}....\delta x^{ik}}(x_0)\cdot dx^{i1}...dx^{ik}
$$

\begin{teorema}
О дифференцируемости сложной функции.
\\
Пусть $E\subset\R^n,$ а $f:E\to\R_1$ - p штук раз диф-ема на Е. Пусть $\varphi^i:H\to\R^n$, где $H\subset\R^n$ - открытое.
\\
Притом $\forall(t\in H)[\varphi^1(t),...,\varphi^n(t)\in H]$ и $\forall(i\in\{1;n\})$ $\varphi^i$ - р штук раз диф-ема в т $t_0\in H$ (или на Н)
\\
Частный случай теоремы: $f:\R^n\to\R^k \forall(k\in\N)$
\\
Например: $f:\R^1\to\R^2  f(x)=(x,x^2+4)$
\end{teorema}
\dokvo
Второй дифференциал и все последущие - не инвариантны.
\\
Пусть (без ограничения общности) $f:\R^n\to\R^2$
\\
$f(x^1,x^2)$ - дважды диф-ема на $E\subset\R^2$
\\
$x^1=\varphi^1(t^1,...,t^m); x^2=\varphi^2(t^1,...,t^m)$
\\
$\varphi^1$ и $\varphi^2$ - дважды диф-емы на $H\subset\R^m.$
\\
В силу инвариантности формы первого дифференциала, df(x,dx) имеют одинаковую формулу независимо от того зависима х или независима.
$$
df(x,dx)=\frac{\delta f}{\delta x^1}(x)dx^1+\frac{\delta f}{\delta x^2}(x)dx^2,
$$

$$
dx^1 = \sum_{i=1}^{m}\frac{\delta\varphi^1}{\delta t^i}dt^i, dx^2 = \sum_{i=1}^{m}\frac{\delta\varphi^2}{\delta t^i}dt^i
$$

$$
d^2 f(x,dx) = d((\frac{\delta f}{\delta x^1}(x)dx^1+\frac{\delta f}{\delta x^2}(x)dx^2),dx)=
$$

$$
=\frac{\delta^2 f}{\delta x^2 \delta x^1}(dx^1)^2+\frac{\delta^2 f}{\delta x^1 \delta x^2}(dx^1 dx^2)+\frac{\delta^2 f}{\delta x^2 \delta x^1}(dx^2 dx^1)+\frac{\delta^2 f}{\delta x^2 \delta x^2}(dx^2)^2+\frac{\delta f}{\delta x^1}(x)d^2x^1+\frac{\delta f}{\delta x^2}(x)d^2x^2=
$$

$$
\frac{\delta^2 f}{\delta (x^1)^2}(dx^1)2+2\frac{\delta^2 f}{\delta x^1 \delta x^2}(dx^1 dx^2)+\frac{\delta^2 f}{\delta(x^2)^2}(dx^2)^2+[\frac{\delta f}{\delta x^1}(x)d^2 x^1 + \frac{\delta f}{\delta x^2}(x)d^2 x^2]^*
$$
Если $x^1,x^2$ - независимые переменные, то $[...]^* = 0$ - получили обычную формулу для $d^2 f$
\\
Значит, второй дифференциал не инвариантен.
\dokno




\subsection{Формула Тейлора для скалярной функции векторного аргумента}
\begin{teorema}
Пусть $f:E\to\R$, $E\subset\R^n$, $E$ открыто, $f$ дифференцируема на $E$ до $(k+1)$-го порядка включительно.
Тогда
\begin{multline}
\forall(x\in E)\exists(r>0)\forall(h\in\R^n:\|h\|<r)\exists(\theta=\theta(x,h), 0<\theta<1)
\\
\left[
	f(x+h)=f(x)+df(x,h)+\frac{1}{2!}d^2 f(x,h)+ ... + \right. \\  +\frac{1}{k!}d^k f(x,h)+\frac{1}{(k+1)!}d^{k+1} f(x+\theta h,h) = \\ \left.=
	f(x)+\sum_{i=1}^k\frac{1}{i!} d^i f(x,h) + \frac{1}{(k+1)!} d^{k+1} f(x+\theta h,h)
\right] 
\end{multline}
\end{teorema}


...

\section{Локальные экстремумы скалярных функций векторного аргумента}
\subsection{Необходимое условие локального экстремума}
\subsection{Достаточные условия локального экстремума}
\begin{teorema}
Пусть $A\subset\R^n$, $f:A \to \R$ и $f$ дважды непрерывно дифференцируема в $x_0\in A$, $x_0$ - стационарная точка $f$,
второй дифференциал $f$ в точке $x_0$, т. е. $d^2 f(x_0,h)$ является невырожденной квадратичной формой.
Тогда $d^2 f(x_0,h)$ определяет наличие в точке $x_0$ локального экстремума, причём если $d^2 f$ --- положительно определённая квадратичная форма (напомним, от $h$), то функция $f$ имеет в точке $x_0$ локальный минимум, если $d^2 f$ --- отрицательно определённая квадратичная форма, то функция $f$ имеет в точке $x_0$ локальный максимум, если же $d^2 f$ --- неопределённая квадратичная форма, то локального экстремума в точке $x_0$ у функции $f$ нет.
\end{teorema}
\dokvo

...

\section{Теорема о неявной функции (теорема Юнга)}
\subsection{Лемма о неявной функции}
\subsection{Теорема Юнга}
\subsection{Следствие о непрерывной дифференцируемости $k$-го порядка}
\subsection{Теорема о неявной функции для скалярной функции векторного аргумента}
...

