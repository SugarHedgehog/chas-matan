\opred

Область - открытое связное множество.

\opred

$G_1$ и $G_2$ - гомеоморфны, если существует взаимооднозначное и взаимонепрерывное отображение одного множества на другое $(f: G_1 \rightarrow G_2)$.

\opred

$f$ - диффеоморфизм класса $C^k$, если $f: G_1 \rightarrow G_2$, где $G_1$ и $G_2$ - гомеоморфны и $f$, $f^-1$ имеют непрерывные производные до порядка $k$ включительно в некоторых открытых можествах, содержащих $G_1$ и $G_2$.

\opred

Понятие поверхности.

Пусть $U$ - область на плоскости с координатами $(u,v)$, ограниченная гладкой, самонепересекающийся, замкнутой кривой.

Непрерывно дифференцируемое отображение $f: \overline{U} \rightarrow \mathbb {R}^3$ такое, что $\frac{D(f^i, f^s)}{D(u,v)} \neq 0$ на всем $\overline{U}$, где $i,s = 1,2,3$ $i<s$, называется гладкая поверхность $S \in \mathbb {R}^3$.

\begin{equation*}
 \begin{cases}
   x = f^1(u,v) 
   \\
   y = f^2(u,v)
   \\
   z = f^3(u,v).
 \end{cases}
\end{equation*}

Это параметрический способ задания поверхности с параметрами $(u,v)$.

Пусть $U_1$ - некоторая гладкая область на плоскости $(u_1,v_1)$ и $S_1$ - гладкая поверхность, которая определяется непредельным дифференцируемым отображением $g: \overline{U_1} \rightarrow \mathbb {R}^3$.

\opred

$S$ и $S_1$ - тождественные, если существует непрерывно дифференцируемый геоморфизм $\varphi: \overline{U_1} \rightarrow \overline{U}$ 
\\
$[f \circ \varphi = g]$, т.е. $\forall i=\overline{1},\overline{3}[f^i(\varphi^1(u_1,v_1),\varphi^2(u_1,v_1)) = g^i(u_1,v_1)]$ и 
\\
$\forall((u_1, v_1) \in \overline{U_1})[\frac{D(\varphi_1, \varphi_2)}{D(u_1,v_1)} \neq 0]$

В этом случае отображения $а: \overline{U} \rightarrow \mathbb {R}^3$ и $g: \overline{U_1} \rightarrow \mathbb {R}^3$ называются различными параметризациями одной поверхности.

\opred

Если $f$ $k$ раз непрерывно дифференцируема, то говорят, что поверхность определяемая этим отображением принадлежит классу $C^k$. 
\\
Тогда переход от одной гладкой параметризации к другой должен осуществляться с помощью диффеоморфизма класса $C^k$.

\opred

Если $f$ не взаимнооднозначно т.е. $\exists (A_1 \neq A_2 : A_1, A_2 \subset \overline{U})[f(A_1) = f(A_2)]$, тогда точка $f(A_1)$ называется кратной.

\opred

Если в качестве пространства параметров взять $(x,y)$, тогда поверхность задается системой:

\begin{equation*}
 \begin{cases}
   x =x 
   \\
   y = y
   \\
   z = z(x,y).
 \end{cases}
\end{equation*}

И такая поверхность называется явно заданная.

\opred

Если есть уравнение $F(x,y,z)=0$, где $F$ - непрерывно дифференцируемо в непрерывной области $(x,y,z)$, тогда совокупность точек $(x,y,z) : F(x,y,z)=0$, называется поверхность, заданная неявно.

\opred

Если в некоторой точке $(x_0,y_0,z_0)$ $F$ удовлетворяет теореме о неявной функции, то часть поверхности в некоторой окрестности этой точки допускает явное представление.
\\
Тогда поверхность заданная неявно, локально сводится к поверхности заданной явно.

Поверхность можно задать в векторном виде:

$\left\{
  \begin{array}{ccc}
x = x(u,v) 
\\
y = y(u,v) 
\\
z = z(u,v) 
  \end{array}
\right.$
$\Rightarrow r = $
$\left(
  \begin{array}{ccc}
x 
\\
y 
\\
z 
  \end{array}
\right)
$
$=$
$\left(
  \begin{array}{ccc}
x(u,v) 
\\
y(u,v) 
\\
z(u,v) 
  \end{array}
\right)$
$= r(u,v)$.