Перейдём теперь к вопросу об интегрировании функции 

\begin{equation}\label{integral_podst_Eilera}
\int R(x,\sqrt{ax^2+bx+c})dx
\end{equation}

Случай, когда $a=0$, фактически рассмотрен нами ранее и потому интереса не представляет.
Введём стандартное обозначение дискриминанта: $D=b^2-4ac$.
Рассмотрим теперь случаи, когда $D=0$.
Если $a<0$, то функция определена лишь в одной точке, и говорить об интеграле нет смысла (т. к. интеграл определяется на промежутке).
Если же $a>0$, то корень извлекается, и задача сводится к взятию интеграла вида $\int R (x,|x-x_0|)dx$, что не представляет особой сложности.

Пусть теперь $a>0$, $D>0$.
Тогда
\begin{equation}\label{vydel_poln_kvadr}
ax^2+bx+c=a\left(x+\frac{b}{2a}\right)^2+\left(c-\frac{b^2}{4a}\right)=a\left(x+\frac{b}{2a}\right)^2-\frac{D}{4a}
\end{equation}

Положим теперь 
\begin{equation}\label{zamena_pered_podst_Eilera}
\tau = \sqrt{a}\left(x+\frac{b}{2a}\right),
\alpha^2=\frac{D}{4a}, ~ 
\text{тогда} ~ 
x=\frac{\tau}{\sqrt{a}}-\frac{b}{2a}, ~ 
dx=\frac{1}{\sqrt{a}}d\tau
\end{equation}
Выражение (\ref{vydel_poln_kvadr}) примет вид $\tau^2-\alpha^2$, а исследуемый интеграл (\ref{integral_podst_Eilera}) преобразуется в:

$$
\int R\left(\frac{\tau}{\sqrt{a}}-\frac{b}{2a},\sqrt{\tau^2-\alpha^2}\right)\cdot\frac{1}{\sqrt{a}}d\tau
$$

Теперь рассмотрим случай, когда $a>0$, $D<0$. Замена будет аналогична замене (\ref{zamena_pered_podst_Eilera}), за исключением того, что $\alpha^2=-\frac{D}{4a}$. Интеграл (\ref{integral_podst_Eilera}) примет вид

$$
\int R\left(\frac{\tau}{\sqrt{a}}-\frac{b}{2a},\sqrt{\tau^2+\alpha^2}\right)\cdot\frac{1}{\sqrt{a}}d\tau
$$

В случае, если $a<0$, $D>0$, замена снова будет аналогична (\ref{zamena_pered_podst_Eilera}), за исключением того, что $\tau = \sqrt{a}\left(x+\frac{b}{2a}\right)$. Интеграл (\ref{integral_podst_Eilera}) примет вид

$$
\int R\left(\frac{\tau}{\sqrt{a}}-\frac{b}{2a},\sqrt{\tau^2-\alpha^2}\right)\cdot\frac{1}{\sqrt{-a}}d\tau
$$

И, наконец, если $D<0$, $a<0$, то подынтегральная функция не имеет смысла.

Таким образом, задача отыскания интеграла (\ref{integral_podst_Eilera}) свелась к отысканию следующих интегралов (здесь $t=\frac{\tau}{\alpha}$, постоянные множители вынесены за знак интеграла):

$$
\int\hat R(t,\sqrt{1-t^2})dt
$$$$
\int\hat R(t,\sqrt{1+t^2})dt
$$$$
\int\hat R(t,\sqrt{t^2-1})dt
$$

Проницательный читатель заметит, что в первых двух случаях можно применить гиперболическую замену, а в третьем - тригонометрическую, но существуют подстановки, позволяющие свести взятие этих интегралов непосредственно к интегрированию рациональной функции. Эти подстановки названы в честь первооткрывателя -- Эйлера.

Для взятия интеграла вида
$$\int\hat R(t,\sqrt{t^2-1})dt$$
применяют замену
$$\sqrt{t^2-1}=u(t\pm 1)$$
или
$$\sqrt{t^2-1}=\pm(t-u)$$

Для взятия интеграла вида
$$\int\hat R(t,\sqrt{t^2+1})dt$$
применяют замену
$$\sqrt{t^2+1}=tu\pm 1$$
или
$$\sqrt{t^2+1}=\pm(t-u)$$

Для взятия интеграла вида
$$\int\hat R(t,\sqrt{1-t^2})dt$$
применяют замену
$$\sqrt{1-t^2}=u(1\pm t)$$
или
$$\sqrt{1-t^2}=tu\pm1$$

Поясним на примере последней, как они работают:
$$\sqrt{1-t^2}=tu-1$$
$$1-t^2=t^2 u^2 -2tu+1$$
$$2tu=(1+u^2)t^2$$
$$2u=(1+u^2)t$$
$$t=\frac{2u}{(1+u^2)}$$
$$\sqrt{1-t^2}=tu-1=\frac{2u^2}{(1+u^2)}$$

Дифференциал $u'(t)du$ также будет рациональной функцией; выписать его предоставляем читателю. Таким образом, интеграл рационализировался.
