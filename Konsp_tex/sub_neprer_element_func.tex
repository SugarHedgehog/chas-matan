1) Показательная функция. 

$y=a^x$, $(a>1)$ монотонно $\nearrow$ в промежутке $X=(-\infty;+\infty)$.
Ее значения $>0$ и заполняют весь промежуток $Y=(0;+\infty)$. Это видно из $\exists (x=log_a y)\forall (y>0)$, следовательно $f(x)$ непрерывна $\forall(x \in X)$.

2) Логарифмическая функция. 

$y=log_a x (a>0, a \neq 1)$.
Пусть $a>1$, тогда $f(x)\nearrow$ при $x \in X = (0; +\infty)$. К тому же, $f(x)$ принимает любое значение из $(-\infty;+\infty)$, следовательно $f(x)$ непрерывна.

3) Степенная функция. 

$y=x^\alpha (\alpha \neq 0)$

При $x: (0; +\infty)$

$f(x)\nearrow(\alpha > 0)$

$f(x) \searrow(\alpha < 0)$

$f(x) \in (0; +\infty)$

4) Тригонометрические функции.

$y = sin x$ - непрерывна на $X = [-\frac{\pi}{2};\frac{\pi}{2}]$ (т.к. $f(x)$ монотонна и принимает $\forall$ значений из $[-1;1]$). 

Также $f(x)$ - непрерывна на $ X = (k\pi - \frac{\pi}{2};k\pi + \frac{\pi}{2}) \forall (k \in \mathbb {Z})$.

Итого : $f(x)$ - непрерывна для $\forall (x)$.

Аналогично : $y=cos x$ непрерывна для $\forall (x)$.

Отсюда вытекает непрерывность :

$tg = \frac{sin x}{cos x}$, кроме $x = (2k + 1)\frac{\pi}{2}$,

$arctg = \frac{cos x}{sin x}$, кроме $x = k\pi$

$sec = \frac{1}{cos x}$, кроме $x = (2k + 1)\frac{\pi}{2}$,

$cos = \frac{1}{sin x}$, кроме $x = k\pi$.

5)Обратные тригонометрические функции.

$y = arcsin x, y = arccos x$ непрерывны на $[-1;1]$

$y = arctg x, y = arcctg x$ непрерывны на $[-\infty;+\infty]$
