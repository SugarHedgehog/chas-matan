\section{Дифференциальное исчисление функции одной независимой переменной}
\subsection{Определение производной и дифференциала, связь между этими понятиями}
\subsection{Связь между понятиями дифференцируемости и непрерывности функций}
\subsection{Дифференцирование и арифметические операции}
\subsection{Теорема о производной сложной функции. Инвариантность формы первого дифференциала}
\subsection{Теорема о производной обратной функции}
\subsection{Производные основных элементарных функций. Доказательство}
\subsection{Касательная к кривой. Геометрический смысл производной и дифференциала}
\subsection{Физический смысл производной и дифференциала}
\subsection{Односторонние и бесконечные производные}
\subsection{Производные и дифференциалы высших порядков}
...

\section{Основные теоремы дифференциального исчисления}
\subsection{Понятие о локальном экстремуме функции}
\opred

\fXRx.
Точка $x_0$ называется точкой локального минимума, а значение в ней - локальным минимумом функции $f$, если
$$
\exists (U(x_0)) \forall(x \in U(x_0) \cap X)[f(x) \geq f(x_0)]
$$

\opred

\fXRx.
Точка $x_0$ называется точкой локального максимума, а значение в ней - локальным максимумом функции $f$, если
$$
\exists (U(x_0)) \forall(x \in U(x_0) \cap X)[f(x) \leq f(x_0)]
$$

\opred

\fXRx.
Точка $x_0$ называется точкой строгого локального минимума, а значение в ней - строгим локальным минимумом функции $f$, если
$$
\exists (\mathring{U}(x_0)) \forall(x \in \mathring{U}(x_0) \cap X)[f(x) > f(x_0)]
$$

\opred

\fXRx.
Точка $x_0$ называется точкой строгого локального максимума, а значение в ней - строгим локальным максимумом функции $f$, если
$$
\exists (\mathring{U}(x_0)) \forall(x \in \mathring{U}(x_0) \cap X)[f(x) < f(x_0)]
$$

\opred

Точками локального экстремума называются вместе точки локального минимума или максимума.

\opred

Локальными экстремумами называются вместе локальные минимумы или максимумы.

\opred

Точками строгого локального экстремума называются вместе точки строгого локального минимума или максимума.

\opred

Строгими локальными экстремумами называются вместе строгие локальные минимумы или максимумы.

\opred

\fXR, $x_0$ - двусторонняя предельная точка $X$.
Если $x_0$ - точка локального экстремума, то она называается точкой внутреннего локального экстремума.



\subsection{Теорема Ферма}
\subsubsection{Теорема Ферма о производной в точке локального экстремума.}

\fXR, $f$ дифференцируема в точке внутреннего локального экстремума $x_0$.
Тогда $f'(x_0)=0$.

\subsubsection{Замечание 1.}

В невнутренней точке локального экстремума производная может, вообще говоря, быть не равной нулю.
Пример: $f:[-1;1]\to \R$, невнутренний локальный максимум $x_0 = 1$, $f'(x_0)=2$.

\subsubsection{Замечание 2.}
Теорема Ферма необратима.
Пример: $f:\R\to\R$, $f(x)=x^3$, $f'(0)=0$, но $f$ не имеет локальных экстремумов.




\subsection{Теорема Ролля}
\subsubsection{Теорема.}

Если $f:[a;b]\to \R$ такова, что

1) $f$ непрерывна на $[a;b]$;

2) $f$ дифференцируема на $(a;b)$;

3) $f(a)=f(b)$,

то $\exists(c \in (a;b))[f'(c)=0]$.

\subsubsection{Замечание 1.}

Геометрическая интерпретация теоремы: пусть кривая задана функцей $y=f(x)$.
Тогда между любыми двумя точками с равными ординатами, лежащими на данной кривой, найдётся такая точка, в которой касательная к данной кривой параллельна оси абсцисс.

\subsubsection{Замечание 2.}

Условие (1) избыточно: т. к. уже требуется, чтобы $f$ была дифференцируема на $(a;b)$, достаточно потребовать непрерывности $f$ в $a$ и $b$. Остальные условия существенны.

\subsubsection{Следствие. Теорема о корнях производной.}

Между любых двух корней дифференцируемой функции лежит корень её производной.

\dokvo

Применим теорему Ролля к случаю, когда $f(a)=f(b)=0$.






\subsection{Теорема Лагранжа и следствия из нее}
\subsubsection{Теорема Лагранжа о промежуточном значении (о конечных приращениях).}

Если $f:[a;b]\to \R$ такова, что

1) $f$ непрерывна на $[a;b]$;

2) $f$ дифференцируема на $(a;b)$;

то $\exists(c \in (a;b))[f(b)-f(a)=f'(c)(b-a)]$.

\subsubsection{Замечание 1.}

Равенство $f(b)-f(a)=f'(c)(b-a)$ называют формулой Лагранжа или формулой конечных приращений.

\subsubsection{Замечание 2.}

Формулу Лагранжа можно записать и в другом виде, если положить $\theta=\frac{c-a}{b-a}$:

$$
f(b)-f(a)=f'(a+\theta(b-a))(b-a)
$$

Полагая $x=a, h=b-a$, имеем

$$
f(x+h)-f(x)=f'(x+\theta h)h
$$

\subsubsection{Следствие 1.}

Функция, имеющая на промежутке равную нулю производную, постоянная на нём.

\subsubsection{Следствие 2.}

Пусть на промежутке $X$ определены и дифференцируемы две функции $f$ и $g$, притом на концах промежутка, если они в него входят, $f$ и $g$ непрерывны.
Если $\forall(x \in X)[f'(x)=g'(x)]$, то $\forall(x \in X)[f(x)-g(x)=const]$.

\subsubsection{Следствие 3.}

Функция, имеющая на промежутке ограниченную производную, равномерно непрерывна на нём.

\subsubsection{Следствие 4.}

Пусть $f:[a;b]\to \R$, $f$ непрерывна, $f$ дифференцируема на $(x_0;x_0+h)\subset [a;b]$.
Тогда правая производная $f$ в $x_0$ непрерывна.



\subsection{Теорема Коши}
\subsubsection{Теорема Коши.}

Пусть $f:[a;b]\to \R$, $g:[a;b]\to \R$, причём:

1) $f$ и $g$ непрерывны на $[a;b]$;

2) $f$ и $g$ дифференцируемы на $(a;b)$;

3)$\nexists (x \in (a;b))[g(x)=0]$

Тогда

$$
\exists (c \in (a;b))\left[ \frac{f(b)-f(a)}{g(b)-g(a)}=\frac{f'(c)}{g'(c)}\right].
$$

\subsubsection{Замечание 1.}

Теорема Коши не является следствием из теоремы Лагранжа; наоборот, теорема Лагранжа - частный случай теоремы Коши для $g(x)=x$.

\subsubsection{Замечание 2.}

Равенство $ \frac{f(b)-f(a)}{g(b)-g(a)}=\frac{f'(c)}{g'(c)}$ называют формулой конечных приращений Коши.



\section{Формула Тейлора}
\subsection{Формула Тейлора для многочлена}
\subsection{Формула Тейлора для произвольной функции. Различные формы остаточного члена формулы Тейлора}
\subsection{Локальная формула Тейлора}
\subsection{Формула Маклорена. Разложение по формуле Маклорена некоторых элементарных функций}
\subsection{Применение формулы Тейлора}
...

\section{Правило Лопиталя}
\subsection{Неопределённость. Виды неопределённостей}
Пусть даны две непрерывные на интервале $(a; b)$ функции $f(x)$ и $g(x)$, где $\{a; b\} \subset \overline{\mathbb{R}}$. Неопределённостью типа $\left[\frac{0}{0}\right]$ в точке $a$ называется предел 
\[
\lim_{x \to a+}\frac{f(x)}{g(x)}
\]
в случае, когда
\[
\lim_{x \to a+}f(x) = \lim_{x \to a+}g(x) = 0
\]
Аналогично определяются неопределённости вида $\left[\frac{\infty}{\infty}\right]$ и в точке $b$.

Другие виды неопределённостей сводятся к этим двум. Вообще говоря, неопределённость типа $\left[\frac{\infty}{\infty}\right]$ может быть сведена к типу $\left[\frac{0}{0}\right]$. Действительно, пусть
$$\lim_{x \to a+}f(x) = \lim_{x \to a+}g(x) = \infty$$
тогда
\[
\frac{f(x)}{g(x)}=\frac{\frac{1}{g(x)}}{\frac{1}{f(x)}}
\]
Однако при раскрытии неопределённостей возникает необходимость расcматривать их отдельно.

Неопределённость-произведение сводится к неопределённостям-частным двумя способами:

$$
[0 \cdot \infty]=\lim_{x\to x_0}(f(x) \cdot g(x))=\lim_{x\to x_0}\frac{f(x)}{\frac{1}{g(x)}}=\left[\frac{0}{0}\right]
$$

$$
[0 \cdot \infty]=\lim_{x\to x_0}(f(x) \cdot g(x))=\lim_{x\to x_0}\frac{g(x)}{\frac{1}{f(x)}}=\left[\frac{\infty}{\infty}\right]
$$

Неопределённости-степени сводятся с неопределённостям-произведениям (а затем - к неопределённостям-частным) через равенство 
$$
f(x) ^ {g(x)}=e^{g(x) \cdot \ln f(x)}
$$

Заметим, что это равенство, как и сам предел, имеет смысл лишь при $f(x)>0$.
Покажем, как раскрываются неопределённости-степени:

$$
[\infty ^0]=\lim_{x\to x_0}(f(x) ^{g(x)})=\lim_{x\to x_0}e^{g(x) \cdot \ln f(x)}=e^{\lim_{x\to x_0}(g(x) \cdot \ln f(x))}=e^{[\infty \cdot 0]}
$$

$$
[0^0]=\lim_{x\to x_0}(f(x) ^{g(x)})=\lim_{x\to x_0}e^{g(x) \cdot \ln f(x)}=e^{\lim_{x\to x_0}(g(x) \cdot \ln f(x))}=e^{-[0 \cdot \infty]}
$$

$$
[1 ^\infty]=\lim_{x\to x_0}(f(x) ^{g(x)})=\lim_{x\to x_0}e^{g(x) \cdot \ln f(x)}=e^{\lim_{x\to x_0}(g(x) \cdot \ln f(x))}=e^{[0 \cdot \infty]}
$$

Наконец, рассмотри раскрытие неопределённости-разности:

$$
[\infty - \infty]=\lim_{x\to x_0}(f(x) - g(x))=\lim_{x\to x_0}\left(f(x) \cdot g(x)\left(\frac{1}{f(x)}-\frac{1}{g(x)}\right)\right)=[\infty \cdot 0]
$$

Таким образом, раскрытие неопределённостей сведено к раскрытию неопределённостей-частных.

\subsection{Теорема Лопиталя}
Докажем теперь правило Лопиталя для неопределённостей вида $\left[\frac{\infty}{\infty}\right].$

\subsubsection{Лемма об обратном пределе.}

Пусть даны функции $f$ и $g$, такие, что $lim_{x \to x_0}f(x)=lim_{x \to x_0}g(x)=+\infty$ и 
существует предел $lim_{x \to x_0} \frac{f(x)}{g(x)}$, 
тогда существует и предел $lim_{x \to x_0} \frac{g(x)}{f(x)}$.

\dokvo

По определению бесконечно большой в точке $x_0$ функции $\exists(V=\mathring{U}_\delta(x_0))\forall(x \in V$

\subsubsection{Замечание.}

Очевидно, вынос знака "минус" из-под знака предела не составляет сложности и не влияет на применимость правила.

\subsubsection{Теорема.}

Пусть даны функции $f$ и $g$, такие, что:
1)$f$ и $g$ определены на полуинтервале $(a;b]$
2)$f$ и $g$ дифференцируемы на полуинтервале $(a;b]$

\subsection{Применение правила Лопиталя}

\section{Применение дифференциального исчисления к исследованию функции одной переменной}
\subsection{Монотонные функции}
\subsection{Экстремумы функций} 
\subsection{Выпуклые функции}
\subsection{Асимптоты кривых}
\subsection{Схема исследования функций}


