Согласно определению непрерывности,
$f:X \to R$ непрерывна, если 
$\forall(x_0 \in X) \forall(\epsilon >0) \exists(\delta>0)[0<|x-x_0|<\delta \Rightarrow |f(x)-f(x_0)|<\epsilon] $

В общем случае $\delta$ зависит от $\epsilon$ и $x_0$, т. е. $\delta=\delta(\epsilon,x_0)$.
Однако иногда $\delta$ зависит только от $\epsilon$ и не зависит от $x_0$, т. е. $\delta=\delta(\epsilon)$.

\opred
$f(x)$ равномерно непрерывна на $X$, если
$$ \forall(\epsilon >0) \exists(\delta>0) \forall(x_0 \in X) [0<|x-x_0|<\delta \Rightarrow |f(x)-f(x_0)|<\epsilon] $$

\subsubsection{Замечание 1.}
Если $f(x)$ равномерно непрерывна на $X$, то $f(x)$ непрерывна на $X$.
(Т.~к. квантор общности $\forall$ можно переносить вправо.)

\subsubsection{Замечание 2.}
Не всякая функция $f$, непрерывная на $X$, равномерно непрерывна на $X$.
(Например: $f(x)=x^2, f:\R \to \R$.)

