\section{Криволинейные интегралы первого рода}
\subsection{Задача о вычислении массы нити}
\subsection{Определение криволинейного интеграла первого рода}
\subsection{Сведение криволинейного интеграла первого рода к обыкновенному определённому интегралу}
...

\section{Криволинейные интегралы второго рода}
\subsection{Задача о вычислении работы}
\subsection{Определение криволинейного интеграла второго рода}
\subsection{Сведение криволинейного интеграла второго рода к обыкновенному определённому интегралу}
\subsection{Обобщение на $n$-мерный случай}
\subsection{Связь между криволинейными интегралами первого и второго рода}
\subsection{Ориентация кривой}
\subsection{Формула Грина. Нахождение площади плоской фигуры}
\subsection{Криволинейные интегралы второго рода, не зависящие от пути интегрирования}
...

\section{Элементы теории поверхностей}
\subsection{Понятие поверхности}
\opred

Область - открытое связное множество.

\opred

$G_1$ и $G_2$ - гомеоморфны, если существует взаимооднозначное и взаимонепрерывное отображение одного множества на другое $(f: G_1 \rightarrow G_2)$.

\opred

$f$ - диффеоморфизм класса $C^k$, если $f: G_1 \rightarrow G_2$, где $G_1$ и $G_2$ - гомеоморфны и $f$, $f^-1$ имеют непрерывные производные до порядка $k$ включительно в некоторых открытых можествах, содержащих $G_1$ и $G_2$.

\opred

Понятие поверхности.

Пусть $U$ - область на плоскости с координатами $(u,v)$, ограниченная гладкой, самонепересекающийся, замкнутой кривой.

Непрерывно дифференцируемое отображение $f: \overline{U} \rightarrow \mathbb {R}^3$ такое, что $\frac{D(f^i, f^s)}{D(u,v)} \neq 0$ на всем $\overline{U}$, где $i,s = 1,2,3$ $i<s$, называется гладкая поверхность $S \in \mathbb {R}^3$.

\begin{equation*}
 \begin{cases}
   x = f^1(u,v) 
   \\
   y = f^2(u,v)
   \\
   z = f^3(u,v).
 \end{cases}
\end{equation*}

Это параметрический способ задания поверхности с параметрами $(u,v)$.

Пусть $U_1$ - некоторая гладкая область на плоскости $(u_1,v_1)$ и $S_1$ - гладкая поверхность, которая определяется непредельным дифференцируемым отображением $g: \overline{U_1} \rightarrow \mathbb {R}^3$.

\opred

$S$ и $S_1$ - тождественные, если существует непрерывно дифференцируемый геоморфизм $\varphi: \overline{U_1} \rightarrow \overline{U}$ 
\\
$[f \circ \varphi = g]$, т.е. $\forall i=\overline{1},\overline{3}[f^i(\varphi^1(u_1,v_1),\varphi^2(u_1,v_1)) = g^i(u_1,v_1)]$ и 
\\
$\forall((u_1, v_1) \in \overline{U_1})[\frac{D(\varphi_1, \varphi_2)}{D(u_1,v_1)} \neq 0]$

В этом случае отображения $а: \overline{U} \rightarrow \mathbb {R}^3$ и $g: \overline{U_1} \rightarrow \mathbb {R}^3$ называются различными параметризациями одной поверхности.

\opred

Если $f$ $k$ раз непрерывно дифференцируема, то говорят, что поверхность определяемая этим отображением принадлежит классу $C^k$. 
\\
Тогда переход от одной гладкой параметризации к другой должен осуществляться с помощью диффеоморфизма класса $C^k$.

\opred

Если $f$ не взаимнооднозначно т.е. $\exists (A_1 \neq A_2 : A_1, A_2 \subset \overline{U})[f(A_1) = f(A_2)]$, тогда точка $f(A_1)$ называется кратной.

\opred

Если в качестве пространства параметров взять $(x,y)$, тогда поверхность задается системой:

\begin{equation*}
 \begin{cases}
   x =x 
   \\
   y = y
   \\
   z = z(x,y).
 \end{cases}
\end{equation*}

И такая поверхность называется явно заданная.

\opred

Если есть уравнение $F(x,y,z)=0$, где $F$ - непрерывно дифференцируемо в непрерывной области $(x,y,z)$, тогда совокупность точек $(x,y,z) : F(x,y,z)=0$, называется поверхность, заданная неявно.

\opred

Если в некоторой точке $(x_0,y_0,z_0)$ $F$ удовлетворяет теореме о неявной функции, то часть поверхности в некоторой окрестности этой точки допускает явное представление.
\\
Тогда поверхность заданная неявно, локально сводится к поверхности заданной явно.

Поверхность можно задать в векторном виде:

$\left\{
  \begin{array}{ccc}
x = x(u,v) 
\\
y = y(u,v) 
\\
z = z(u,v) 
  \end{array}
\right.$
$\Rightarrow r = $
$\left(
  \begin{array}{ccc}
x 
\\
y 
\\
z 
  \end{array}
\right)
$
$=$
$\left(
  \begin{array}{ccc}
x(u,v) 
\\
y(u,v) 
\\
z(u,v) 
  \end{array}
\right)$
$= r(u,v)$.
\subsection{Касательная плоскость и нормаль к поверхности}
Пусть $S$ - непрерывно дифференцируемая поверхность. Рассмотрим векторное представление $r = \overline{r}(u,v)$, $(u,v) \in \overline{U}$,
\\
где $r$ - непредельная дифференцируемая векторная функция на $\overline{U}$.

Предположим, что для прямых $u=u_0$ или $v=v_0$, $\overline{U} \bigcap u$ $(\overline{U} \bigcap v)$ состоит из одного отрезка (может быть вырождающейся точкой).

\opred

При сделанных предположениях и при фиксированных $u_0$ или $v_0$, отображение $r$ называется координатная линия:
\\
$r = \overline{r}(u_0,v)$ - $v$ линия,
\\
$r = \overline{r}(u,v_0)$ - $u$ линия.

\opred

$\overline{r_v} = \frac{dr}{dv}$ и $\overline{r_u} = \frac{dr}{dг}$ - касательные вектора.

\opred

Точка поверхности $S$, в которой векторы $\overline{r_u}$ и $\overline{r_v}$ неколлинеарны, называется неособая точка при данной представлении этой поверхности. 

Условие неколлинеарности векторов:
\\
$\overline{r_u} \cdot \overline{r_v} \neq 0$, если точка неособая, то, в частности $\overline{r_u} \neq 0$ и $\overline{r_v} \neq 0$.

\opred

Если $\overline{r_u}$ и $\overline{r_v}$ коллинеарны, то точка называется особая при данном ее представлении.

Рассмотрим кривую на поверхности $S$. Пусть эта кривая задана представлениями $r=r(u(t),v(t))$ где $t \in [t_o, T]$, а $u(t),v(t)$ - непрерывно дифференцируемы, причем $(u'(t))^2+(v'(t))^2 \neq 0$. Тогда 

$$d\overline{r} = \overline{r_u}du+ \overline{r_v}dv = (\overline{r_u}u_t' + \overline{r_v}v_t')dt$$.

\opred

Если точка поверхности неособая, то $dr$ будет касательной к кривой $F(u(t),v(t))$.

\opred

Плоскость, проходящая через точку $\overline{r}(u_0,v_0)$ поверхности, в которой лежат все касательные к кривым $\overline{r}(u(t),v(t))$, проходящим через эту точку, 
\\
называются касательной плоскостью, проходящей через точку $\overline{r}(u_0,v_0)$,
\\
называемую точкой касания.

Если точка неособая, то в ней существует единственная касательная плоскость. Это будет плоскость проходящая через эту точку параллельно векторам $\overline{r_u}(u_0,v_0)$, $\overline{r_v}(u_0,v_0)$.

Уравнение касательной плоскости.

Пусть $\overline{r_0}$ - радиус вектор точки касания, $\overline{r}$ - текущий вектор.
\\
Вектор $\overline{r} - \overline{r_0}$ лежит в плоскости касания и в этой же плоскости лежат вектора $\overline{r_u}$,  $\overline{r_v}$, т.е. $\overline{r_u}$,  $\overline{r_v}$, $\overline{r} - \overline{r_0}$ - компланарны.

Пусть $\overline{r} = (x,y,z)$, $\overline{r_0} = (x_0,y_0,z_0)$. Тогда:
\\
$\left|
  \begin{array}{ccc}
x - x_0 \quad y - y_0 \quad z - z_0
\\
x_u' \quad \quad \quad y_u' \quad \quad \quad z_u'
\\
x_v' \quad  \quad \quad y_v' \quad \quad \quad z_v'
  \end{array}
\right|$
$=$
\\
$=(x-x_0)$
$\left| 
  \begin{array}{ccc}
y_u' \quad z_u'
\\
y_v' \quad z_v'
  \end{array}
\right|$
$+(y-y_0)$
$\left| 
  \begin{array}{ccc}
z_u' \quad x_u'
\\
z_v' \quad x_v'
  \end{array}
\right|$
$+(z-z_0)$
$\left| 
  \begin{array}{ccc}
x_u' \quad y_u'
\\
x_v' \quad y_v'
  \end{array}
\right|$
\\
Если поверхность задана явно, т.е. $z = z(x,y)$, то:
\\
$(x-x_0)$
$\left| 
  \begin{array}{ccc}
0 \quad z_x'
\\
1 \quad z_y'
  \end{array}
\right|$
$+(y-y_0)$
$\left| 
  \begin{array}{ccc}
z_x' \quad 0
\\
z_y' \quad 1
  \end{array}
\right|$
$+(z-z_0)$
$\left| 
  \begin{array}{ccc}
1 \quad 0
\\
0 \quad 1
  \end{array}
\right|$
$=0$

$$z-z_0 = (x-x_0)z_x' + (y-y_0)z_y'$$
\\
Если поверхность задана неявно, т.е. $F(x,y,z) = 0$, то: 

$$(x-x_0)F_x' + (y-y_0)F_y' + (z-z_0)F_z' = 0$$

\opred

Прямая проходящая через точку касания поверхности с касательной плоскостью, перпендикулярная этой плоскости, называется нормальной прямой $K$ поверхности в указанной точке.

Уравнение нормальной прямой 
\\
Пусть $(x_0,y_0,z_0)$ - точка касания. $\overline{r_u}$,  $\overline{r_v}$ лежат в касательной плоскости $K$.
\\
$\overline {n} = [\overline{r_u},  \overline{r_v}] \Rightarrow \overline {n} \bot \overline{r_u}, \overline {n} \bot \overline{r_v} \Rightarrow \overline {n} \bot  K  \Rightarrow \overline {n}$ - нормальный вектор.
\\
$[\overline{r_u},  \overline{r_v}] =$
$\left| 
  \begin{array}{ccc}
i \quad j \quad k
\\
x_u' \quad y_u' \quad z_u'
\\
x_v' \quad y_v' \quad z_v'
  \end{array}
\right|$
$=(y_u'z_v' - y_v'z_u'; x_v'z_u' - x_u'z_v'; x_u'y_v' - x_v'y_u') = \overline{n} $.
\\
Значит, нормальная прямая будет выглядеть так:

$$\frac {x-x_0}{y_u'z_v' - y_v'z_u'} = \frac {y-y_0}{x_v'z_u' - x_u'z_v'} =  \frac {z-z_0}{x_u'y_v' - x_v'y_u'}$$

Если поверхность задана явно, т.е. $z = z(x,y)$, то:

$$\frac {x-x_0}{z_x'} = \frac {y-y_0}{z_y'} = -(z - z_0)$$

\opred

Любой ненулевой вектор, коллинеарный нормальной прямой, проходящий через заданную точку поверхности называется нормаль к этой поверхности в данной точке.

\opred

Вектор нормали единичной длины - единичный вектор нормали.

$(\overline{n_e} = \frac {\overline{n}}{|\overline{n}|}; |\overline{n_e}| = 1)$
\\
Если $\overline{n} = (A,B,C)$, то $\overline{r_u} \cdot \overline{r_v} = A\overline{i} + B\overline{j} + C\overline{k}$. Тогда: 

$\pm \overline{n_e} = \frac {A\overline{i}}{\sqrt{A^2 + B^2 + C^2}}+ \frac {B\overline{j}}{\sqrt{A^2 + B^2 + C^2}}+\frac {C\overline{k}}{\sqrt{A^2 + B^2 + C^2}}$
\\
Вектор нормали может быть направлен вверх или вниз. Поэтому в формуле перед $\overline{n_e}$ стоит знак $\pm$/
\\
Если поверхность задана явно $z = z(x,y)$, то:

$\pm \overline{n_e} = \frac {z_x'}{\sqrt{1 + (z_x')^2 +(z_y')^2}}; \frac {z_y'}{\sqrt{1 + (z_x')^2 +(z_y')^2}};\frac {1}{\sqrt{1 + (z_x')^2 +(z_y')^2}}$
\\
Если поверхность задана неявно т.е. $F(x,y,z) = 0$, то:

$\pm \overline{n_e} = \frac {F_x'}{\sqrt{(F_x')^2 +(F_y')^2+(F_z')^2}}; \frac {F_y'}{\sqrt{(F_x')^2 +(F_y')^2+(F_z')^2}};\frac {F_z'}{\sqrt{(F_x')^2 +(F_y')^2+(F_z')^2}}$
\\
Пусть есть $S$ и $S_1$, причем $S$ задано представлением $\overline{r}(u,v)$, а $S_1$ - $\rho(u_1,v_1)$, где $(u,v) \in \overline{U}$, a $(u_1,v_1) \in \overline{U_1}$;

$\varphi : \overline{U} \rightarrow \overline{U_1}$ - непрерывно дифференцируемый гомеоморфизм, причем:

$\Upsilon = \frac {D(\varphi^1,\varphi^2)}{D(u_1,v_1)} \neq 0$.
\\
Поверхности тождественны $\Rightarrow$ $\rho(u_1,v_1) = r(\varphi^1(u_1,v_1),\varphi^2(u_1,v_1))$
\\
Продифференируем по $u_1$ и $v_1$:

$\overline{\rho_{u_1}}= \overline{r_u}(\varphi_{u_1})^1 + \overline{r_v}(\varphi_{v_1})^1 $ ; $\overline{\rho_{v_1}}= \overline{r_u}(\varphi_{v_1})^1 + \overline{r_v}(\varphi_{v_2})^2 $.

Любая пара векторов $(\overline{r_u},\overline{r_v})$ преобразуется в $(\overline{\rho_{u_1}},\overline{rho_{v_1}})$ c помощью матрицы
$\left(
  \begin{array}{ccc}
(\varphi_{u_1})^1 \quad (\varphi_{u_1})^2
\\
(\varphi_{v_1})^1 \quad (\varphi_{v_1})^2
  \end{array}
\right)$.
По условию $\Upsilon \neq 0 \Rightarrow A \neq 0 \Rightarrow$ переход от $(\overline{r_u},\overline{r_v})$ к $(\overline{\rho_{u_1}},\overline{rho_{v_1}})$ происходит с помощью невырожденной линейной системы $\Rightarrow$ для $(u_1,v_1)$ векторы $\rho_{u_1}$ и $\rho_{v_1}$ линейно независимы $\leftrightarrow \overline{r_u},\overline{r_v}$ - линейно независимы $\leftrightarrow \overline{r_u},\overline{r_v}$ - неколлинеарны $\Rightarrow$ точка неособая.
\subsection{Ориентация поверхности}
\opred

Точка $M_0$ поверхности $S$ внутренняя, если существует окрестность точки $M_0$ такая, что множество точек окрестности, не являющихся точками поверхности представляет собой несвязное множество.

\opred

Точки поверхности не являющиеся внутренними - граничные точки поверхности.
\\
Рассмотрим поверхность $S$; $M_0$ - внутренняя точка. Будем считать, что в некоторой внутренней точке $M_0$ гладкой поверхности $S$ выбрано одно из двух направлений нормали.

\opred

Если при обходе вдоль дугового замкнутого контура $\Gamma$, лежащего на $S$ и, не имеющего общих точек с границей этой поверхности, нормаль, непрерывно изменяясь вдоль $\Gamma$, по возвращению в точку $M_0 \in \Gamma$ вернется к первоначальному направлению, то поверхность $S$ - ориентированная (двусторонняя).

Выбор одного из направлений нормали в какой-либо внутренней точке ориентированной поверхности определяет сторону поверхности.
\\
Примеры:
\\
Двусторонние поверхности:
\\
Полусфера, плоскость, эллипсоид, однополосный гиперболоид.
\\
Односторонняя поверхность - лист Мёбиуса.

Пусть $S$ - двусторонняя поверхность (есть направление нормали). Пусть есть две точки $M_0$ и $M_1 \in S$ и гладкий контур $\Gamma_1$, соединяющий их и не пересекающий границы $S$. Тогда при перемещении из одной точки в другую вдоль $\Gamma$, во вторую точку мы придем с тем же самым направлением нормали.

Если, приходя в $M_1$, по двум разным путям $\Gamma_1$ и $\Gamma_2$ мы получили бы разное направление нормали, то это бы привело к тому, что, двигаясь по замкнутому контуру $M_1 \Gamma_1 M_0 \Gamma_2 M_1$, мы бы пришли в $M_1$ с другим направлением нормали, что противоречит условию.

Таким образом, выбор направления нормали в одной точке однозначно определяет выбор направления нормали на всей $S$. 

\opred

Совокупность точек поверхности с приписанными направлениями нормали называется определённой стороной поверхности.

Пусть $S$ - незамкнутая, гладкая двусторонняя поверхность, ограниченная простым контуром. Обозначим определённое направление обхода в качестве "$+$" , если наблюдатель, двигаясь по контуру видит внутреннюю часть поверхности слева. Противоположное направление - "$-$".

Обозначим за $S^+$ сторону поверхности с "$+$" направлением обхода контура, за $S^-$ - сторону с "$-$" направлением обхода контура.

Рассмотрим негладкую поверхность $S$(кубик). 

Разобьем гладкую поверхность $S$ на две части: $S^+$ И $S^-$: сфера - тут все ясно, тор - проблема.

Но мы ограничимся случаями, когда поверхность разделяется с помощью некоторого гладкого контура и указываем в качестве "$+$" ориентации - внешнюю сторону поверхности, а в качестве "$-$" - противоположную.


\section{Площадь поверхности}
\subsection{Вантуз (сапог) Шварца}
\subsection{Определение площади поверхности}
\subsection{Преобразование элемента площади}
...

\section{Поверхностные интегралы первого рода}
\subsection{Определение поверхностного интеграла первого рода}
\subsection{Существование поверхностного интеграла первого рода и сведение его к обыкновенному двойному интегралу}
\subsection{Приложения поверхностных интегралов первого рода}
...

\section{Поверхностные интегралы второго рода}
\subsection{Определение и свойства}
\subsection{Сведение к обыкновенному двойному интегралу}
...


