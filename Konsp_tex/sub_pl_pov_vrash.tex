Пусть L - простая прямая плоскости, которая не пересекается с $O_X$.

\begin{opred}
	Поверхность вращения F - множество точек в $\R^3$, описываемое кривой L при вращении содержащей её плоскости $O_y$ вокруг $O_x$
\end{opred}
Прямую L будем считать параметризованной (с учётом того, что третья координата z(t) в точках кривой L равна 0:z=0)
\\
Параметризующее отображение $\varGamma$ имеет вид $\varGamma(t)=(x(t)); y(t);0), t \in [a;b]$.
Пусть $T=\{a=t_0;t_1;...;t_n=b\}$ - разбиение [a;b] и K - соответствующая разбиению T ломаная, вписаная в кривую L с помощью отображения $\varGamma$, т.е. вершины ломаной - точки $$\varGamma(t_k), k=\{0;n\}$$.
Обозначим через Ф - поверхность, которая получается из K вращения вокруг $O_x$.
\\
Пусть $S(\CYRF)$ - площадь поверхности.
Поверхность Ф определяется выбранным разбиением T. Будем говорить, что Ф вписана в поверхность вращения F.
\begin{opred}
F имеет площадь, если множество площадей, вписанных в неё поверхностей, построенных вышеописанным способом по всем разбиениям T имеет предел. Этот 
$$
\lim_{d(T)\to 0}S(\Phi)
$$
называется площадью поверхности F(des;S(F)).
\end{opred}
\\
Вращение каждого подотрезка создаёт усечённый конус. Площадь поверхности Ф-суммы площадей боковой поверхности усеченных конусов каждого подотрезка:
\\
$$S(\phi)=\pi\sum_{i=1}^{n}(y(t_{i-1})+y(t_i))\sqrt{(x(t_i)-x(t_{i-1}))^2+(y(t_i)-y(t_{i-1})^2},$$ 
\\
где $ y(t_{i-1}) и y(t_i)$ - радиусы оснований.
\\
Пусть $$x_i=x(t_i), y_i=y(t_i).$$ Тогда:
\\
 $$S(\phi)=\pi\sum_{i=1}^{n}(y_{i-1}+y_i)\sqrt{(x_i-x{i-1})^2+(y_i-y{i-1})^2};$$
$$\forall(t)[y(t)\geqslant0]$$

\begin{teorema}
Пусть F-поверхность, отвечающая гладкой простой кривой L(гладкая: x(t) и y(t) - непрерывно дифференцируемы). Тогда
$$
S(F)=2\pi\int_{a}^{b}y(t)\sqrt{(x'(t))^2+(y'(t))^2}dt
$$
\end{teorema}
\dokvo
(аналогично теореме о длине кривой)

\subsubsection{Упражнение:}
Доказать, что для эквивалентных параметризаций кривой L получим одинаковые площади поверхностей вращения.

\subsubsection{Замечание 1.}
Если кривая L не $\cap$ с $O_y$, то для поверхности вращения F, полученной из L при вращении плоскости $xO_y$ вокруг $O_y$ имеет место формула:
$$
S(F)=2\pi\int_{a}^{b}x(t)\sqrt{(x'(t))^2+(y'(t))^2}dt
$$

\subsubsection{Замечание 2.}
Пусть L - график неотрицательной непрерывной дифференцируемой функции, то отображение $\varGamma(x)=(x;y(x);0)$ задает гладкую параметризацию L. Тогда 
$$
S(F)=2\pi\int_{a}^{b}y(x)\sqrt{(x'(x))^2+(y'(x))^2}dx=2\pi\int_{a}^{b}y(x)\sqrt{1+(y'(x))^2}dx
$$