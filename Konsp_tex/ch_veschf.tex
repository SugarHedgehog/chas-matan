\section{Предел вещественной функции вещественного аргумента}
\subsection{Определение предела функции по Коши, примеры}
\subsection{Определение предела функции по Гейне, примеры, эквивалентность определений}
\subsection{Обобщение понятия предела функции на расширенную числовую ось}
...

\section{Свойства пределов функции и функций, имеющих предел}
\subsection{Свойства, связанные с неравенствами}
\subsection{Свойства, связанные с арифметическими  операциями}
...

\section{Односторонние пределы функции}
\subsection{Определение односторонних пределов, связь между существованием предела и односторонних пределов функции}
\subsection{Теорема о существовании односторонних пределов у монотонной функции и её следствия}
...

\section{Критерий Коши, замечательные пределы, бесконечно малые функции}
\subsection{Критерий Коши существования предела функции}
\subsection{Первый замечательный предел}
\subsection{Второй замечательный предел}
\subsection{Бесконечно малые функции и их классификация}
...

\section{Непрерывные функции. Общие свойства}
\subsection{Понятие непрерывности функции в точке}
\subsubsection{Определение непрерывности функиции в точке по Коши.}

\opred

\fXR, $x_0 \in X$.
Функция $f$ непрерывна в точке $x_0$, если
$$
\forall(\epsilon>0)\exists(\delta>0)[|x-x_0|<\delta \Rightarrow |f(x)-f(x_0)|<\epsilon].
$$

Или, что то же самое, но с применением окрестностей:

$$
\forall(\epsilon>0)\exists(\delta>0) [f(U_{\delta}(x_0) \cap X) \subset U_{\epsilon}(f(x_0))]
$$

Или, что то же самое:

$$
\forall(\epsilon>0)\exists(\delta>0) [f(U_{\delta,X}(x_0)) \subset U_{\epsilon}(f(x_0))]
$$

И, наконец, полностью перейдя в термины окрестностей:

$$
\forall(U \in O(f(x_0)))\exists(V \in O_X(x_0)) [f(V) \subset U]
$$

\subsubsection{Замечание 1.}

Вдумчивый читатель легко заметит, что это определение похоже на определение предела в точке, в котором проколотые окрестности заменены на непроколотые. Несколькими строками ниже мы рассмотрим вопрос о связи непрерывности функции, её предела и её значения в данной точке.

\subsubsection{Замечание 2.}

Если $x_0$ - изолированная точка множества $X$, то
$$
 \exists(U \in O(x_0))[U \cap X = \{x_0\}] \Rightarrow f(U)=\{f(x_0)\}],
$$
т. е. найдётся окрестность точки $x_0$, образом которой явялется единственная точка, и функция $f$ в точке $x_0$ непрерывно. Однако никаких содержательных результатов этот случай не даёт, и потому в дальнейшем мы, как правило, будем рассматривать непрерывность функции, заданной на множестве точек, лишь в предельных точках этого множества.

\subsubsection{Критерий непрерывности функции в точке.}

\fXRx.
$f$ непрерывна в $x_0$ тогда и только тогда, когда 

$$
\lim_{x\to x_0}f(x)=f(x_0)
$$

\subsubsection{Следствие 1.}

\fXRx.
$f$ непрерывна в $x_0$ тогда и только тогда, когда знак предела и знак функции коммутируют, т. е. 
$$
\lim_{x \to x_0} f(x) = f(\lim_{x \to x_0}x)
$$

\subsubsection{Следствие 2.}

\fXRx, $f$ непрерывна в $x_0$, $\Delta y = f(x_0+\Delta x)-f(x_0)$.
$\Delta x \to 0$ тогда и только тогда, когда $\Delta y \to 0$

\subsubsection{Определение непрерывности в точке по Гейне.}

\fXRx.
$f$ непрерывна в $x_0$, если 
$$
\forall(\{x_n\}:x_n \in X \cap x_n \to x_0)[f(x_n)\to f(x_0)]
$$

Обозначив $\Delta x = x_n-x_0$, $\Delta x = f(x_n)-f(x_0)$, можем сформулировать:

$$
\Delta x \to 0 \Rightarrow \Delta y \to 0
$$


\subsection{Непрерывность функции на множестве}
\opred

Функция $f:X\to \R$ называется непрерывной на $X$, если она непрерывна во всех точках $x \in X$.

\opred

Если функция $f:x \to \R$ не является непрерывной в точке $x_0 \in X$, то $x_0$ называется точкой разрыва функции $f$.

\subsubsection{Замечание 1.}

Так как все точки множества $\N$ изолированны, то любая функция $f:\N \to \R$ непрерывна.

\subsection{Понятие колебания функции на множестве и в точке. Необходимое и достаточное условие непрерывности функции в точке}
\opred

\fXR, $E \subset X$, $\alpha_E=\inf_E f(x)$, $\beta_E=\sup_E f(x)$.
Тогда разность $\alpha_E-\beta_E$ называется колебанием функции $f$ на множестве $E$:

$$
\omega(f,E)=\alpha_E-\beta_E=\sup_E f(x) - \inf_E f(x)
$$

Или, что то же самое, 

$$
\omega(f,E)=\sup_{a,b \in E}(f(a)-f(b))
$$

\subsubsection{Примеры.}

$\omega(x^2,[-2;4])=16$

$\omega(\sgn x,[0;4])=1$

$\omega(\sgn x,(0;4])=0$

$\omega(\sgn x,[-1;4])=2$

\opred

\fXRx.
Величина $\lim_{\delta \to 0+}\omega(f,U_{\delta}(x_0))$ называется колебанием функции $f$ в точке $x_0$:

$$
\omega(f,x_0)=\lim_{\delta \to 0+}\omega(f,U_{\delta}(x_0))
$$

\subsubsection{Теорема.}

Пусть $f:X\to\R$.
Функция $f$ непрерывна в точке $x_0\in X$ тогда и только тогда, когда $\omega(f,x_0)=0$.

\subsection{Односторонняя непрерывность}
\subsection{Классификация точек разрыва}
\subsection{Локальные свойства непрерывных функций}
...

\section{Функции, непрерывные на отрезке}
\subsection{Теорема Больцано-Коши и следствия из неё}
\subsubsection{Теорема.}
Пусть $f:[a;b]\to \R$ и $f$ непрерывна на $[a;b]$, при этом $f(a) \cdot f(b) <0$,
т. е. на концах отрезка $[a;b]$ непрерывная на нём функция $f$ принимает значения разного знака.
Тогда $\exists(c \in (a;b))[f(c)=0]$,
т. е. хотя бы в одной точке интервала $(a;b)$ функция обращается в нуль.

\subsubsection{Замечание.}
Теорема Больцано-Коши не только утверждает существование точки, в которой функция обращается в нуль, но и фактически даёт способ её найти - методом половинного деления отрезка. Этот факт может быть применён при нахождении корня уравнения численными методами.

\subsubsection{Следствие 1 (теорема о промежуточном значении).}
\fXR, при этом $f$ непрерывна на некотором промежутке $Y \subset X$, $\{a;b\}\subset Y$, $a<b$.
Тогда $\forall(\gamma$ между $f(a)$ и $f(b))\exists(c:c\in[a;b])[f(c)=\gamma]$.

\subsubsection{Следствие 2.}
\fXR, $X$ - промежуток и $f$ непрерывна на нём.
Тогда $f(X)$ - тоже промежуток.


 


\subsection{Первая теорема Вейерштрасса}
\subsubsection{Теорема.}

Функция, непрерывная на отрезке, ограничена на нём.

\opred

Компактом (компактным множеством) называется такое множество $X$, что
$$
\forall(\{x_n\}:x_n \in X)\exists(\{x_{n_k}\})[\{x_{n_k}\}\to x_0 \in X],
$$ 
т. е. в любой последовательности точек этого множества можно выделить подпоследовательность, сходящуюся к точке этого множества.

\subsubsection{Замечание.}
Конечный или бесконечный интервал $(a;b)$, где $\{a;b\}\subset\overline\R$, не является компактом, т. к. любая подпоследовательность любой последовательности его точек, сходящейся к $a$ или $b$, сходится к не принадлежащей интервалу точке $a$ или $b$ соответственно.

Полуинтервал также не является компактом.
Предоставляем читателю доказать это самостоятельно.

\subsubsection{Обобщение первой теоремы Вейерштрасса.}

Функция, непрерывная на компакте, ограничена на нём.

\subsubsection{Замечание.}

Функция, определённая на некомпактном множестве, может быть на нём неограничена. Пример - тождественная функция $f(x)=x$ на некомпактом множестве $(-\infty;+\infty)$.


\subsection{Вторая теорема Вейерштрасса}
\subsubsection{Теорема.}

Функция, непрерывная на компакте, достигает на нём точных верхней и нижней границ множества своих значений.

\mnemo
Эту теорему можно запоминать по начертанию цифры {\LARGE 2}, разделив его на три части: горизонтальная черта снизу - отрезок, средняя часть - непрерывная функция, <<завиток>> сверху - точное верхнее значение.


\subsubsection{Следствие.}

Пусть $f:[a;b]\to\R$ и $f$ непрерывна, $\alpha=\inf(f[a;b])$, $\beta=\sup(f[a;b])$.
Тогда $f([a;b])=[f(a);f(b)]$.




\subsection{Понятие равномерной непрерывности функции. Теорема Кантора, следствия из неё}
Согласно определению непрерывности,
$f:X \to R$ непрерывна, если 
$\forall(x_0 \in X) \forall(\epsilon >0) \exists(\delta>0)[0<|x-x_0|<\delta \Rightarrow |f(x)-f(x_0)|<\epsilon] $

В общем случае $\delta$ зависит от $\epsilon$ и $x_0$, т. е. $\delta=\delta(\epsilon,x_0)$.
Однако иногда $\delta$ зависит только от $\epsilon$ и не зависит от $x_0$, т. е. $\delta=\delta(\epsilon)$.

\opred
$f(x)$ равномерно непрерывна на $X$, если
$$ \forall(\epsilon >0) \exists(\delta>0) \forall(x_0 \in X) [0<|x-x_0|<\delta \Rightarrow |f(x)-f(x_0)|<\epsilon] $$

\subsubsection{Замечание 1.}
Если $f(x)$ равномерно непрерывна на $X$, то $f(x)$ непрерывна на $X$.
(Т.~к. квантор общности $\forall$ можно переносить вправо.)

\subsubsection{Замечание 2.}
Не всякая функция $f$, непрерывная на $X$, равномерно непрерывна на $X$.
(Например: $f(x)=x^2, f:\R \to \R$.)


\subsubsection{Теорема Кантора о равномерной непрерывности.}
\fXR, $X$ - компакт и $f$ непрерывна на $X$.
Тогда $f$ равномерно непрерывна на $X$.

\subsubsection{Следствие 1.}

Если $f:[a;b] \to \R$ непрерывна на отрезке $[a;b]$, то она равномерно непрерывна на этом отрезке.

\subsubsection{Следствие 2.}

Если $f:[a;b] \to \R$ непрерывна на отрезке $[a;b]$, то
$$
\forall(\epsilon > 0) \exists (\delta > 0) \exists(a_1, b_1 : a < a_1 < b_1 < b, b_1 - a_1<\delta)[\omega(f,[a_1,b_1]<\epsilon],
$$
или, что то же самое,
$$
\forall(\epsilon > 0) \exists (отрезок \Delta \subset [a;b])[\omega(f,\Delta)<\epsilon]
$$
т. е. найдётся подотрезок, на котором колебание функции меньше любого наперёд заданного.

\subsubsection{Замечание.}



\subsection{Свойства монотонных функций. Теорема об обратной функции}
\subsubsection{Лемма 1.}

Непрерывная функция, заданная на отрезке, инъективна в том и только том случае, когда она строго монотонна.

\subsubsection{Лемма 2.}

Пусть $X \subset \mathbb{R}$.
Любая строго монотонная функция $f:X \to Y \subset \mathbb{R}$ обладает обратной функцией $f^{-1}:Y \to X$,
причём обратная функция $f^{-1}$ имеет тот же характер монотонности на $Y$, что и функция $f$ на $X$.

\subsubsection{Лемма 3.}

Пусть $X \subset \mathbb{R}$.
Монотонная функция $f:X\to \mathbb{R}$ может иметь разрывы только первого рода.

\subsubsection{Следствие 1.}

Если $a$ - точка разрыва монотонной функции $f$, то по крайней мере один из пределов функции $f$ слева или справа от $a$ определён.

\dokvo

Если $a$ - точка разрыва, то она является предельной точкой множества $X$ и, по лемме 3, точкой разрыва первого рода.
Таким образом, точка $a$ является по крайней мере правосторонней или левосторонней предельной для множества $X$, т. е. выполнено хотя бы одно из следующих условий:
\[
f(a-0)=\lim_{x \to a-0}f(x)
\]
\[
f(a+0)=\lim_{x \to a+0}f(x)
\]
Если $a$ - двусторонняя предельная точка, то существуют и конечны оба односторонних предела.

\subsubsection{Следствие 2.}

Если $a$ - точка разрыва монотонной функции $f$, то по крайней мере в одном из неравенств $f(a-0)\leq f(a)\leq f(a+0)$ - для неубывающей $f$ или $f(a-0)\geq f(a)\geq f(a+0)$ - для невозрастающей $f$, имеет место знак строгого неравенства, т. е. $f(a-0) < f(a+0)$ - для неубывающей $f$ или $f(a-0) > f(a+0)$ - для невозрастающей $f$, и в интервале, определённым этим строгим неравенством, нет ни одного значения функции.
(Также говорят: интервал свободен от значений функции.)

\subsubsection{Следствие 3.}

Интервалы, свободные от значений монотонной функции, соответствующие разным точкам разрыва этой функции, не пересекаются.

\subsubsection{Лемма 4. Критерий непрерывности монотонной функции.}

Пусть даны отрезок $X=[a;b] \subset \mathbb{R}$ и монотонная функция $f:X \to \mathbb{R}$.
$f$ непрерывна в том и только том случае, когда $f(X)$ - отрезок $Y$ с концами $f(a)$ и $а(b)$.
($f(a) \leq f(b)$ для неубывающей $f$, $f(a) \geq f(b)$ для невозрастающей $f$).

\dokvo
\neobh

Т. к. $f$ монотонна, то все её значения лежат между $f(a)$ и $f(b)$. Т. к. $f$ непрерывна, то она принимает и все промежуточные значения. Следовательно, $f(X)$ - отрезок.

\dost

\pp, т. е. что $\exists \left(c \in [a;b]\right)$ - точка разрыва $f$.
Тогда по следствию 2 леммы 3 один из интервалов: $\left(f(c-0);f(c)\right)$ или $\left(f(c);f(c+0)\right)$ - определён и не содержит значений $f$.
С другой стороны, этот интервал содержится в $Y$, т. е. $f$ принимает не все значения из $Y$, $f(X)\neq Y$. Получили противоречие.












\subsubsection{Теорема.}
\fXR и $f$ строго монотонна.
Тогда существует обратная функция $f^{-1}:Y \to X$, где $Y=f(X)$, притом $f^{-1}$ строго монотонна на $Y$ и имеет тот же характер монотонности, что и $f$ на $X$.
Если, кроме того, $X=[a;b]$ и $f$ непрерывна на отрезке $X$, то $f([a; b])$ есть отрезок с концами $f(a)$ и $f(b)$ и $f^{-1}$ непрерывна на нём.


\subsection{Непрерывность элементарных функций}
...

