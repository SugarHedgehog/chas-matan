\begin{opred}
Фигура на плоскости - любое множество точек на плоскости.
\end{opred}

\begin{opred}
Ограниченная фигура - фигура, которая целиком содержится в некотором круге ограниченного радиуса.
\\
Мы будем рассматривать только ограниченные фигуры.
\end{opred}

\begin{opred}
Говорят, что фигура $F_1$ вписана в $F_2$, если все точки $F_1 \in F_2$. Тогда $F_2$ описана вокруг $F_1$.	
\end{opred}
\\
Чтобы определить понятие площади, возьмем точки на $O_x$ и проведем прямые || $O_y$. Аналогично - на $O_y$. Получим плоскость разбитую на квадраты со стороной = 1. S каждого такого квадрата равна 1. Это квадраты ранга 1.
\\
Обозначим через $\sigma_1$ фигуру, составленную из квадратов ранга 1, полностью лежащих в F.
\\
Пусть $s_1$ - площадь $\sigma_1$.
Обозначим через $\Sigma_1$ фигуру, составленную из квадратов, имеющих с F непустое пересечение. $S_1$ - площадь $\Sigma_1$.
\\
Затем делим каждый квадрат 1 ранга на 100 маленьких квадратов со стороной 0,1; S каждого такого квадрата равна 0,01 (это квадраты второго ранга).
\\
Пусть $\sigma_2$ - фигура, состоящая из квадратов ранга 2, полностью лежащих в $F(s_2=S(\sigma_2));$ $\Sigma_2$-фигура, состоящая их квадратов ранга 2, имеющих с F непустое пересечение $(S_2=S(\Sigma_2))$
\\
В итоге, уменьшая площадь квадратов, получим последовательность фигур $\{\Sigma_n\}$ и $\{\sigma_n\}$. Очевидно, что при $n \to \infty$ $S_{k+1} \leqslant S_k$, а $s_{k+1}\geqslant s_k,$ при том 
$$\forall (k,m \in \{1;n\})[s_k \leqslant S_m].$$
Отсюда $\{S_n\}$ и $\{s_n\}$ - ограничены (т.к. $s_1 \leq s_n \leq S_1 \leq S_n): \{S_n\}$ ограничена снизу, $\{s_n\}$ сверху.
\\
Обозначим:
$$
S=\lim_{n\to \infty}S_n, s=\lim_{n \to \infty}s_n
$$

\begin{opred}
Фигура F - квадрируема, если S=s, и
$$
S(F)=\lim_{n\to \infty}\{S_n\},\lim_{n \to \infty}\{s_n\}
$$
\end{opred}

\subsubsection{Упражнение (монотонность площади)}
Доказать, что если $F_1$ и $F_2$ - квадрируемы, причем $F_1 \leq F_2$, то $[S(F_1) \leq S(F_2)]$

\subsubsection{Лемма 1.}
Пусть $F_1,F_2$ - квадрируемые фигуры, $F_1\cap F_2$ - квадрируемо и $S(F_1 \cap F_2)=0,$ тогда $F_1 \cup F_2$ - квадрируемо и $S(F_1+F_2)=S(F_1)+S(F_2).$
\dokvo
Рассмотрим  $F_1\cap F_2$ и $\{\Sigma\}$, где $\{\Sigma\}$ - квадраты ранга n, имеющие непустое пересечение с $F_1\cap F_2$.
\\
По условию: $F_1\cap F_2$ - квадрируема и $S(F_1\cap F_2)=0$ $\Rightarrow$ при достаточно большом n $[S(\Sigma_n)<\epsilon].$
\\
$\Sigma_n$ расширим до $\Sigma_n^1$ - фигуры, состоящей из квадратов ранга n, покрывающих $F^1.$
\\
$\Sigma_n$ расширим до $\Sigma_n^2$ - фигуры, состоящей из квадратов ранга n, покрывающей $F^2$.
\\
Можем считать(при необходимости увеличивая n), что:
$$
S(\Sigma_n^1)-S(F^1)<\frac{\epsilon}{2};
$$

$$
S(\Sigma_n^2)-S(F^2)<\frac{\epsilon}{2};
$$

$$
S(\Sigma_n^1 \cup \Sigma_n^2)=S(\Sigma_n^1)+S(\Sigma_n^2)-S(\Sigma_n^1 \cap \Sigma_n^2) \geq S(F^1)+S(F^2)-\epsilon
$$
\\
т.к.
$$
(\Sigma_n^1 \supset F_1, а F_2\subset\Sigma_n^2, то S(F_1) \leq S(\Sigma_n^1), S(F_2) \leq S(\Sigma_n^2))
$$

$$
S(\Sigma_n^1 \cup \Sigma_n^2) \leq S(\Sigma_n^1) + S(\Sigma_n^2) \leq (S(F^1)-\frac{\epsilon}{2})+(S(F^2)-\frac{\epsilon}{2})=S(F^1)+S(F^2)-\epsilon
$$
\\
Значит, $\lim_{n \to \infty}(\Sigma_n^1 \cup \Sigma_n^2)=S(F^1)+S(F^2)$
\\
Аналогично, и с $\sigma_n^1$, $\sigma_n^2$:
$$
\lim_{n \to \infty}(\sigma_n^1 \cup \sigma_n^2)=S(F^1)+S(F^2)
$$
Получается, фигура $F_1 \cup F_2$ по определению квадрируема и 
$$
S(F_1 \cup F_2) = S(F_1) + S(F_2)
$$
\dokno

\subsubsection{Упражнение}
Обобщить Лемму 1 на случай более двух фигур

\begin{opred}
Граничная точка фигуры - точка на плоскости (которая не может принадлежать фигуре) такая, что в круге любого радиуса с центром в этой точке содержатся как точки фигуры, так и точки дополнения фигуры до плоскости.
\end{opred}

\begin{opred}
Изолированная точка фигуры F - точка А, такая, что $\exists(\sigma > 0)[F \cap U_\sigma = A]$
\end{opred}

\begin{opred}
Внутренняя точка фигуры F - точка B, такая, что $\exists(\sigma > 0)[ U_\sigma \subset F]$
\end{opred}

\subsubsection{Пример}
Возьмем фигуру $F=\{(x;y):x^2+y^2<2\}\cup (4;0).$
Здесь: множество внутренних точек: $\{(x;y):x^2+y^2<2\}$
\\
множество граничных точек:
$\{(x;y):x^2+y^2=2\};$
\\
множество изолированных точек: (4;0).

\subsubsection{Лемма 2.}
Если граница фигуры F квадрируема, и $S(\delta F)=0$, то сама фигура квадрируема.

\begin{opred}
Граница фигуры - совокупность граничных точек ($\delta F$ - граница фигуры F)
\end{opred}

\dokvo
Пусть $\tilde{\Sigma_n}$ - фигура, составленная из квадратов ранга n, имеющих с $\delta F$ непустое пересечение. Тогда т.к. $S(\delta F) = 0,$ то $S(\tilde{\Sigma_n})<\epsilon$.
\\
Пусть $\sigma_n^0$ - фигура, составленная из квадратов ранга n, не вошедших в $\tilde{\Sigma_n}$ каждый из которых содержит внутренние точки из F.
\\
Очевидно $\sigma_n^0 \subseteq \sigma_n \subseteq F \subseteq \Sigma_n\subseteq (\tilde{\Sigma_n}\cup \sigma_n^0).$

$$
\lim_{n\to \infty}(S(\Sigma_n))=S(F), \lim_{n \to \infty}(S(\sigma))=S(F) \Rightarrow
$$
$\Rightarrow$ F-квадрируема.
\dokno

\subsubsection{Упражнение}
Доказать, что кривая, изображаюшая график непр. функции на отрезке, имеет нулевую S.

\begin{teorema}
Пусть f - непрерывная и неограниченная на [a;b] функция. Тогда криволинейная трапеция F, ограниченная сверху кривой y=f(x), снизу осью $O_x$ и с боков x=f,x=b - квадрируема и $S(F)=\int_{a}^{b} f(x) dx$
\end{teorema}

\dokvo
т.к. f(x) - непрерывная $\Rightarrow S(\delta F)=0 \Rightarrow$ (в силу Леммы 2) F-квадрируема.
$S(\alpha,\beta)=S(\beta)-S(\alpha)$ - в силу Леммы об аддитивности промежутка.
\\
Пусть m=min f(x), M=max f(x), то $m(\beta-\alpha) \leq S(\alpha,\beta) \leq M(\beta-\alpha),$ f(x) - непрерывна ($f\in R[a;b]$)
\\
Из теоремы об аддитивности функции ориентированного промежутка следует утверждение теоремы.
\dokno











