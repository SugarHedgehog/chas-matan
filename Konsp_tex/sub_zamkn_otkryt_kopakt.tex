Пусть $\R^n - $нормированное пространство.
\begin{opred}
Открытый шар с центром в $x_0$ и радиусом r - множество $B(x_0,r)$, состоящее из $\{x:x\in\R^n \textasciicircum ||x-x_0||\le r\}$
\end{opred}
\subsubsection{Упражнение:}
Выяснить, что представляют собой открытые шары единичного радиуса в различных пространствах.

\begin{opred}
Пусть $G\subset\R^n. x_0\in G$ - внутренняя точка, если $x_0\in G$ вместе с некоторым открытым шаром, т.е. $\exists(r>0)[B(x_0,r)\subset G]$ или
\\
$\exists(r>0)\forall(x:||x-x_0||<r)[x\in G]$
\end{opred}

\begin{opred}
$G\subset\R^n$ - открытое, если любая его точка внутренняя.
\end{opred}

\subsubsection{Упражнение:}
Доказать, что $B(x_0,r)$ и множество G, состоящее из $\{x\in\R^n:x^i>0,i\{1;n\}\}$ - открытые.

\begin{opred}
$x_0$ - предельная точка множества G, если 
$$
\forall(B(x_0,r))\exists(x\in G: x\ne x_0)[0<||x-x_0||<r]
$$
\end{opred}

\subsubsection{Упражнение:}
Доказать, что если $x_0$ - предельная точка множества G, то $\exists\{x_k\}$ - последовательность элементов множества G, отличных от $x_0$, сходящиеся к $x_0$.

\begin{opred}
Изолированнные точки множества G - точки множества G, которые не являются предельными, или 
$$
\exists(B(x_0,r))\forall(x\in G: x\ne x_0)[||x-x_0||\ge r]
$$
\end{opred}

\begin{opred}
Множество G - замкнуто, если оно содержит в себе все свои предельные точки.
\end{opred}

\subsubsection{Свойства открытых и замкнутых множеств:}
1. Пересечение любого числа и объединение конечного числа замкнутых множеств является замкнутым множеством.
\dokvo
Пусть $G_2,(\alpha\in\Lambda$ - замкнутые множества, $G=\cap G_2)$
\\
Пусть G - незамкнутое множество. Тогда существует (предельная точка $x_0$)$[x_0\notin G] \Rightarrow$
\\
$\Rightarrow x_0$ - предельная точка $G_2\Rightarrow$
\\
$\Rightarrow \exists(\alpha_0\in\Lambda)[x_0\notin G_{\alpha_0}],$
\\
следовательно $G_{\alpha_0}$ - не замкнуто.
Противоречие.
\dokno
2. а) Дополнение замкнутого множества до всего 		пространства - открытое множество;
   б) Дополнение открытого множества до всего пространства - замкнутое множество.
\dokvo
а) Пусть G - замкнутое множество.
\\
Пусть $G_{\R^n}G$ - дополнение - не открытое множество. Значит,
\\
$\exists(x_0 - $ не внутренняя) $\forall(r>0)\exists(x\in B(x_0,r))[x\notin G_{\R^n}G]$
\\
Отсюда: $x_0$ - предельная точка $G\Rightarrow x_0\in G$ (т.к. G - замкнутое) - противоречие, т.к. $x_0$ - не внутренняя
\dokno

б) доказывается аналогично
\\
3. Объединение  любого числа и пересечение конечного числа открытых множеств является открытым множеством.
\\
4. $K\subset\R^n$ - компактное, если из $\forall$ последоват. $\{x_k\}$ элементов этого множества можно выделить подпоследовательность $\{x_{n_k}\}$, сходящуюся к элементу из множества К.
\dokvo
Достаточность:
\\
Дано: К - огр. и замкнуто;
\\
Доказать: К - компакт.
\\
Возьмем $\forall\{x_{n_k}\}$ т.к. $x_k\in K$ -ограничено, то и $\{x_k\}$ - ограничено. Тогда, в силу теоремы Больцано-Вейрштресса из последовательности $\{x_k\}$ можно выделить сходящуюся подпоследовательность $\{x_{k_n}\},$ где $x_{k_n}\to x_0\Rightarrow x_0$ - предельная точка множества K т.к. К замкнуто, то $x_0\in K,$ следовательно К- компактно
\dokno




