Докажем теперь правило Лопиталя для неопределённостей вида $\left[\frac{\infty}{\infty}\right].$

\subsubsection{Лемма об обратном пределе.}

Пусть даны функции $f$ и $g$, такие, что $lim_{x \to x_0}f(x)=lim_{x \to x_0}g(x)=+\infty$ и 
существует предел $lim_{x \to x_0} \frac{f(x)}{g(x)}$, 
тогда существует и предел $lim_{x \to x_0} \frac{g(x)}{f(x)}$.

\dokvo

По определению бесконечно большой в точке $x_0$ функции $\exists(V=\mathring{U}_\delta(x_0))\forall(x \in V$

\subsubsection{Замечание.}

Очевидно, вынос знака "минус" из-под знака предела не составляет сложности и не влияет на применимость правила.

\subsubsection{Теорема.}

Пусть даны функции $f$ и $g$, такие, что:
1)$f$ и $g$ определены на полуинтервале $(a;b]$
2)$f$ и $g$ дифференцируемы на полуинтервале $(a;b]$
