\section{Случайный вектор}

\opred
Пусть случайный опыт \GOFP.
Случайным вектором $\xI$ размерности $n$, наблюдаемым в опыте $G$, называется упорядоченный набор случайных величин, наблюдаемых в данном опыте.

Можно доказать эквивалентность следующего определения:

\opred
Пусть случайный опыт \GOFP.
Случайным вектором $\xI$ размерности $n$, наблюдаемым в опыте $G$, называется функция $\xI:\Omega\to\R^n$, такая,
что $\xI$ $(\calF,\calB_{\R^n})$-измерима, т. е. $\forall(B\in\calB_{\R^n})[\xI^{-1}(B)\in\calF]$.

\opred
Пусть случайный вектор $\xI$ наблюдается в случайном опыте \GOFP.
Распределением случайного вектора $\xI$ называется функция $P_\xI : \calB_{\R^n} \to [0;1]$, определяемая равенством
$$
P_\xI(B)=P(\xI^{-1}(B))
$$

Можно доказать, что $P_\xI$ --- вероятностная мера на $(\R^n,\calB_{\R^n})$.
Этот факт даёт возможность перейти к выборочному вероятностному пространству (аналогично тому, как это было сделано для случайной величины):
$$
\left<\Omega,\calF,P\right> \stackrel{\xI}{\to} \left<\R^n, \calB_{\R^n}, P_\xI\right>
$$
и рассматривать в нём непосредственно заданный случайный вектор $\vec{\eta}(\vec{x})=\vec{x}$.
Легко видеть, что в таком случае $\forall(B\in\calB_{\R^n})\left[P_{\vec{\eta}}(B)=P_\xI(B)\right]$.
%TODO: примеры







\section{Неравенство Маркова}
Пусть $\xi \in L_1(\Omega,\calF,P)$ и $P\{\xi \geq 0\}=1$, $T>0$.
Тогда 
\begin{equation}\label{neravenstvo Markova}
	P\{\xi \geq T\} \leq \frac{M\xi}{T}
\end{equation}

\dokvo
\begin{multline*}
M\xi = \int_{-\infty}^{\infty} x dF_\xi(x) =
\\
\mbox{(т. к. $\xi$ неотрицательна почти наверное)}
\\
=\int_{\{x \geq T\}} x dF_\xi(x) + \int_{\{0 \leq x < T\}} x dF_\xi(x) \geq 
\\
\mbox{(т. к. $F_\xi$ - неубывающая)} \\
\geq \int_{\{x \geq T\}} x dF_\xi(x) \geq \int_{\{x \geq T\}} T dF_\xi(x)
=T\int_{\{x \geq T\}} dF_\xi(x) = \\ =
T \left( \lim_{x\to+\infty} F_\xi(x) - F_\xi(T-)\right) = T P\{\xi\geq T\}
\end{multline*}

\dokno

\section{Неравенство Чебышева}
Пусть $\xi\in l_2(\Omega,\calF,P)$, $\varepsilon>0$.
Тогда
\begin{equation}\label{neravenstvo_Chebysheva}
P\{|\xi-M\xi|\geq\varepsilon\}\leq\frac{D\xi}{\varepsilon^2}
\end{equation}

\dokvo
\begin{multline*}
	P\{|\xi-M\xi|\geq\varepsilon\}=
	P\{|\xi-M\xi|^2\geq\varepsilon^2\}=
	\\\mbox{(положив в неравенстве Маркова (\ref{neravenstvo Markova}) $T=\varepsilon^2$)}\\
	=\frac{M\left((\xi-M\xi)^2\right)}{\varepsilon^2}
	=\frac{D\xi}{\varepsilon^2}
\end{multline*}


\dokno

\section{Закон больших чисел}

\paragraph{Идеология.}
Обычно случайная величина <<размазана>> по числовой оси.
Если случайные величины складывать, то <<размазанность>> будет <<расползаться>>.
Но оказывается, что при определённых условиях среднее арифметическое величин <<расползаться>> не будет.

\begin{teorema}
Пусть $\left\{\xi_k\right\}_{k=1}^\infty$ --- последовательность стохастически независимых интегрируемых с квадратом случайных величин, дисперсия которых ограничена в совокупности, т. е.
$$
\forall(k\in\N)\left[\xi_k\in L_2(\Omega,\calF,P)\right]
$$
$$
\exists(C>0)\forall(k\in\N)\left[ D\xi_k \leq C \right]
$$

Обозначим $\bar{\xi}_n := \frac{1}{n}\suml_{k=1}^n \xi_k$,
$\bar{\mu}_n := \frac{1}{n}\suml_{k=1}^n M\xi_k$


Тогда 
\begin{equation*}
	\forall(\varepsilon>0)\left[P(|\bar{\xi}_n - \bar{\mu}_n| \geq \varepsilon) \xrightarrow[n\to\infty]{} 0\right]
\end{equation*}
\end{teorema}

\dokvo
\begin{multline*}
	P\{|\bar{\xi}_n-\bar{\mu}_n |\geq \varepsilon\} =
	\\ \mbox{(т. к. $\bar{\mu}_n=M\bar{\xi}_n$)} \\
	=P\{|\bar\xi_n - M\bar\xi_n|\geq \varepsilon\}\leq
	\\\mbox{(применяем неравенство Чебышева (\ref{neravenstvo_Chebysheva}))}\\
	\leq \frac{D\bar\xi_n}{\varepsilon^2}
	=\frac{D\left(\frac{1}{n}\suml_{k=1}^n \xi_k\right)}{\varepsilon^2}
	=\frac{\frac{1}{n^2}D\left(\suml_{k=1}^n \xi_k\right)}{\varepsilon^2}
	=\\\mbox{(в силу стохастической независимости дисперсия аддитивна)}\\
	=\frac{\frac{1}{n^2}\left(\suml_{k=1}^n D\xi_k\right)}{\varepsilon^2}
	\leq \frac{\frac{1}{n^2}\left(\suml_{k=1}^n C\right)}{\varepsilon^2}
	= \frac{nC}{n^2 \varepsilon^2}
	= \frac{C}{n \varepsilon^2}
	\xrightarrow[n\to\infty]{} 0
\end{multline*}
\dokno

\opred
Пусть $\{\xi_k\}_{k=1}^\infty$ --- последовательность случайных величин, наблюдаемых в опыте \GOFP,
$\xi$ --- также случайная величина, наблюдаемая в этом опыте.
Говорят, что $\xi_k$ сходится по вероятности к $\xi$ и пишут:
$$
\xi_k \xrightarrow[n\to\infty]{P} \xi
$$
если
$$
\forall(\varepsilon>0)\left[P\{|\xi_n-\xi|\geq \varepsilon\}\xrightarrow[n\to\infty]{}0\right]
$$

Сформулируем теперь следствие из закона больших чисел --- в случае, когда мы имеем дело с последовательностью одинаково распределённых случайных величин.

\begin{sledstvie}
Рассмотрим последовательность одинаково распределённых интегрируемых с квадратом случайных величин $\{\xi_k\}_{k=1}^\infty$, наблюдаемых в случайном опыте \GOFP.
Обозначим $M\xi_k=\mu$, $D\xi_k=\sigma^2$.
Тогда $\bar\xi_n \xrightarrow[n\to\infty]{P}\mu$.
\end{sledstvie}



