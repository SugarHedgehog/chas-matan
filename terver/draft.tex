\section{Случайный вектор}

\opred
Пусть случайный опыт \GOFP.
Случайным вектором $\xI$ размерности $n$, наблюдаемым в опыте $G$, называется упорядоченный набор случайных величин, наблюдаемых в данном опыте.

Можно доказать эквивалентность следующего определения:

\opred
Пусть случайный опыт \GOFP.
Случайным вектором $\xI$ размерности $n$, наблюдаемым в опыте $G$, называется функция $\xI:\Omega\to\R^n$, такая,
что $\xI$ $(\calF,\calB_{\R^n})$-измерима, т. е. $\forall(B\in\calB_{\R^n})[\xI^{-1}(B)\in\calF]$.

\opred
Пусть случайный вектор $\xI$ наблюдается в случайном опыте \GOFP.
Распределением случайного вектора $\xI$ называется функция $P_\xI : \calB_{\R^n} \to [0;1]$, определяемая равенством
$$
P_\xI(B)=P(\xI^{-1}(B))
$$

Можно доказать, что $P_\xI$ --- вероятностная мера на $(\R^n,\calB_{\R^n})$.
Этот факт даёт возможность перейти к выборочному вероятностному пространству (аналогично тому, как это было сделано для случайной величины):
$$
\left<\Omega,\calF,P\right> \stackrel{\xI}{\to} \left<\R^n, \calB_{\R^n}, P_\xI\right>
$$
и рассматривать в нём непосредственно заданный случайный вектор $\vec{\eta}(\vec{x})=\vec{x}$.
Легко видеть, что в таком случае $\forall(B\in\calB_{\R^n})[P_{\vec{\eta}}(B)=P_\xI(B)]$.
%TODO: примеры







\section{Неравенство Маркова}
Пусть $\xi \in L_1(\Omega,\calF,P)$ и $P\{\xi \geq 0\}=1$, $T>0$.
Тогда 
$$
P\{\xi \geq T\} \leq \frac{M\xi}{T}
$$

\dokvo
\begin{multline*}
M\xi = \int_{-\infty}^{\infty} x dF_\xi(x) =
\\
\mbox{(т. к. $\xi$ неотрицательна почти наверное)}
\\
=\int_{\{x \geq T\}} x dF_\xi(x) + \int_{\{0 \leq x < T\}} x dF_\xi(x) \geq 
\\
\mbox{(т. к. $F_\xi$ - неубывающая)} \\
\geq \int_{\{x \geq T\}} x dF_\xi(x) \geq \int_{\{x \geq T\}} T dF_\xi(x)
=T\int_{\{x \geq T\}} dF_\xi(x) = \\ =
T \left( \lim_{x\to+\infty} F_\xi(x) - F_\xi(T-)\right) = T P\{\xi\geq T\}
\end{multline*}

\dokno
%TODO: доказательство

\section{Неравенство Чебышева}
Пусть $\xi\in l_2(\Omega,\calF,P)$, $\varepsilon>0$.
Тогда
$$
P\{|\xi-M\xi|\geq\varepsilon)\leq\frac{D\xi}{\varepsilon^2}
$$

\section{Закон больших чисел}

