\section{Распределение Парето}
\subsection{Определение}
\opred
Говорят, что случайная величина $\xi$ имеет распределение Парето с параметрами $x_0>0$ и $\alpha>0$ и пишут $\xi \sim Par(x_0,\alpha)$, если
$$
F_\xi (x) = P\{\xi < x\} = \left(1 - \left(\frac{x_0}{x}\right)^\alpha\right) \cdot \one{[x_0; +\infty)}(x)
$$

Легко видеть, что функция распределения непрерывна.
\subsection{Плотность}
$$
f_\xi (x) = \alpha\frac{x_0^\alpha}{x^{\alpha+1}} \cdot \one{[x_0; +\infty)}(x)
$$

\subsection{Математическое ожидание}
Математическое ожидание, а, следовательно, и другие моменты, могут существовать или не существовать в зависимости от значения $\alpha$.
Попробуем найти матожидание:
\begin{multline}\label{matozhidanie_Pareto}
M\xi=
\intl_{-\infty}^{+\infty}x f_\xi(x) dx =
\intl_{-\infty}^{+\infty}x \alpha \frac{x_0^\alpha}{x^{\alpha+1}} \cdot \one{[x_0; +\infty)}(x) dx =
\intl_{x_0}^{+\infty}x \alpha\frac{x_0^\alpha}{x^{\alpha+1}} dx =
\\=
\intl_{x_0}^{+\infty} \alpha\frac{x_0^\alpha}{x^\alpha} dx = 
\alpha x_0^\alpha \intl_{x_0}^{+\infty} \frac{1}{x^\alpha} dx
\end{multline}
Последний интеграл, как мы знаем из курса математического анализа, сходится при $\alpha>1$.
Следовательно, при $\alpha>1$ из формулы (\ref{matozhidanie_Pareto}) имеем
\begin{multline}\label{matozhidanie_Pareto_itog}
M\xi = 
\alpha x_0^\alpha \intl_{x_0}^{+\infty} \frac{1}{x^\alpha} dx =
\alpha x_0^\alpha \left.\left(\frac{1}{1-\alpha}\frac{1}{x^{\alpha-1}}\right)\right|_{x=x_0}^{x=+\infty} dx =
\\=
\alpha x_0^\alpha\left(0-\frac{1}{1-\alpha}\cdot\frac{1}{x_0^{\alpha-1}}\right)=
\alpha x_0^\alpha\frac{1}{\alpha-1}\cdot\frac{1}{x_0^{\alpha-1}}=
\\=
\frac{\alpha x_0^\alpha}{(\alpha-1)x_0^{\alpha-1}}=
\frac{\alpha x_0}{\alpha-1}
\end{multline}

\subsection{Прочие моменты}
Как известно, начальный момент существует или не существует одновременно с центральным.
Для начального момента порядка $k$ рассуждениями, аналогичными (\ref{matozhidanie_Pareto}), имеем
\begin{equation}\label{momenty_Pareto}
M(\xi^k)= \alpha x_0^\alpha \intl_{x_0}^{+\infty} \frac{1}{x^{\alpha-k+1}} dx
\end{equation}

А такой интеграл сходится при $\alpha-k+1 > 1$, т.е. при $\alpha > k$.
Следовательно, у распределения Парето с параметрами $x_0$ и $\alpha$ существуют $k$-ые центральный и начальный моменты тогда и только тогда, когда $\alpha > k$.

\subsection{Дисперсия}
Вооружившись формулами (\ref{matozhidanie_Pareto_itog}) и (\ref{momenty_Pareto}), посчитаем дисперсию этого распределения при $\alpha>2$:
\begin{multline}
D\xi=
M(\xi^2)-(M\xi)^2 =
\alpha x_0^\alpha \intl_{x_0}^{+\infty} \frac{1}{x^{\alpha-1}} dx - \left(\frac{\alpha x_0}{\alpha-1}\right)^2 =
\\=
\alpha x_0^\alpha \frac{1}{\alpha-2}\cdot\frac{1}{x_0^{\alpha-2}} - \frac{\alpha^2 x_0^2}{(\alpha-1)^2} = 
\frac{\alpha x_0^2}{\alpha-2} - \frac{\alpha^2 x_0^2}{(\alpha-1)^2} = 
\\=
\alpha x_0^2\left(\frac{1}{\alpha-2} - \frac{\alpha}{(\alpha-1)^2}\right) = 
\\=
\alpha x_0^2\left(\frac{(\alpha-1)^2}{(\alpha-2)(\alpha-1)^2} - \frac{\alpha^2-2\alpha}{(\alpha-2)(\alpha-1)^2}\right) =
\\= 
\alpha x_0^2\left(\frac{\alpha^2-2\alpha + 1}{(\alpha-2)(\alpha-1)^2} - \frac{\alpha^2-2\alpha}{(\alpha-2)(\alpha-1)^2}\right) = 
\frac{\alpha x_0^2}{(\alpha-2)(\alpha-1)^2}
\end{multline}

