\section{Геометрическое распределение}
\subsection{Определение}
$P\{\xi=k\} = (1-p)^{k} p$, $k\in\N_0$.
\subsection{Математическое ожидание}
\begin{multline}\label{matozhidanie_geom}
M\xi =
\suml_{k=1}^\infty k (1-p)^{k} p =
p(1-p)\suml_{k=1}^\infty k (1-p)^{k-1} =
\\=
p(1-p)\suml_{k=1}^\infty \frac{d}{dp}(-(1-p)^{k}) =
p(1-p)\frac{d}{dp} \suml_{k=1}^\infty (-(1-p)^{k}) =
\\=
-p(1-p)\frac{d}{dp} \frac{1-p}{1-(1-p)} = 
-p(1-p)\frac{d}{dp} \frac{1-p}{p} =
\\=
-p(1-p)\frac{d}{dp} \left(\frac{1}{p}-1\right) = 
-p(1-p)\left(-\frac{1}{p^2}\right) =
\frac{1-p}{p}
\end{multline}

\subsection{Дисперсия}
И снова будем применять почленное дифференцирование рядов.
\begin{multline}\label{dispersia_geom}
D\xi = 
M(\xi^2) - (M\xi)^2 =
M(\xi^2) - \frac{(1-p)^2}{p^2} =
\\=
\suml_{k=1}^\infty k^2 (1-p)^{k} p  - \frac{(1-p)^2}{p^2}=
\\=
p(1-p)\suml_{k=1}^\infty (k^2 + k - k) (1-p)^{k-1}  - \frac{(1-p)^2}{p^2}=
\\=
p(1-p)\suml_{k=1}^\infty k(k+1)(1-p)^{k-1} - p(1-p)\suml_{k=1}^\infty k (1-p)^{k-1}  - \frac{(1-p)^2}{p^2}=
\\ \mbox{(значение второй суммы мы уже находили в (\ref{matozhidanie_geom}))} \\=
(1-p)p\suml_{k=1}^\infty k(k+1)(1-p)^{k-1} - \frac{1-p}{p}  - \frac{(1-p)^2}{p^2}=
\\=
(1-p)p\suml_{k=1}^\infty \frac{d^2}{dp^2}(1-p)^{k+1} - \frac{1-p}{p}  - \frac{(1-p)^2}{p^2}=
\\=
(1-p)p\frac{d^2}{dp^2}\suml_{k=1}^\infty(1-p)^{k+1} - \frac{1-p}{p}  - \frac{(1-p)^2}{p^2}=
\\=
(1-p)p\frac{d^2}{dp^2}\left((1-p)\suml_{k=1}^\infty(1-p)^{k}\right) - \frac{1-p}{p}  - \frac{(1-p)^2}{p^2}=
\\=
p(1-p)\frac{d^2}{dp^2}\left((1-p)\frac{1-p}{p}\right) - \frac{1-p}{p}  - \frac{(1-p)^2}{p^2}=
\\=
p(1-p)\frac{d^2}{dp^2}\left(\frac{1-2p+p^2}{p}\right) - \frac{1-p}{p}  - \frac{(1-p)^2}{p^2}=
\\=
p(1-p)\frac{d^2}{dp^2}\left(\frac{1}{p}-2+p\right) - \frac{1-p}{p}  - \frac{(1-p)^2}{p^2}=
\\=
p(1-p)\frac{2}{p^3} - \frac{1-p}{p}  - \frac{(1-p)^2}{p^2}=
\\=
\frac{2(1-p)}{p^2} - \frac{p(1-p)}{p^2}  - \frac{(1-p)^2}{p^2}=
\frac{1-p}{p^2}(2 - p  - (1-p))=
\frac{1-p}{p^2}
\end{multline}

