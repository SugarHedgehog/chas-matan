\section{Пунктуация}

\subsection{Пренебрежение пунктуацией}

Как уже упоминалось выше, интернет позволяет обмениваться информацией достаточно быстро (по сравнению с традиционными методами коммуникации); соответственно, ценность производимой информации падает, что побуждает многих, отправляющих как личные сообщения, так и публичные комментарии, в целях экономии времени пренебрегать пунктуацией, и в первую очередь:

\begin{itemize}

\item Кавычки в именах собственных. Опускаются почти повсеместно. Де-факто -- это новый стандарт. Обусловлено сложностью отделения имён собственных от нарицательных в условиях технического прогресса.

\item Заглавные буквы в начале предложения и в начале имён собственных.

\item Точки и вопросительные знаки в конце предложения.

\item Запятые при обособленных членах предложения и (реже) перечислении.

\item Пробелы рядом со знаками препинания.

\end{itemize}

Однако необходимо заметить, что пренебрежение пунктуацией считается элементом неформального общения и в некотором роде пренебрежением к собеседнику.

\subsection{Смайлики}

\opred
Смайлик - это автономная комбинация знаков препинания и/или спецсимволов либо графическая миниатюра, изображающая эмоции.

Примеры смайликов:

\begin{itemize}

\item :-)

\item :-(

\item :ъ

\item О\_о

\end{itemize}

Впервые в истории использовать скобку в качестве улыбки догадался проживавший в США русский писатель Владимир Набоков, произнёсший в 1969 году в интервью журналу New York Times: ``Я часто думаю, что должен существовать специальный типографский знак, обозначающий улыбку, -- нечто вроде выгнутой линии, лежащей навзничь скобки; именно этот значок я поставил бы вместо ответа на ваш вопрос.''

19 сентября 1982 года профессор Университета Карнеги-Меллона в городе Питтсбург (штат Пенсильвания) Скотт Фалман предложил использовать последовательность символов :-) в качестве смайлика.

Существует пунктуационное явление, являющееся своеобразным сокращением смайлика, а именно -- использование непарных открывающих и закрывающих скобок в качестве знаков препинания, подобно восклицательному знаку. Есть даже афоризм: ``После смайлика запятая не ставится''.

Примеры использования сокращённых смайликов:

\begin{itemize}

\item Привет)

\item Как дела?)

\item Сессия(((

\end{itemize}


\section{Лексика, орфография, фонетика.}

\subsection{Олбанске, или падокаффский.}

Луркоморье (свободная интернет-энциклопедия, подобие Википедии) приводит следующее определение.

\opred

Язык падонков (он же олбанскей) — метафорический аллегорический язык выражения мыслей в неудобочитаемой форме. Фактически, основной концепцией этого т. н. езыка является альтернативное правописание, при котором слово пишется максимально непохожим на словарное написание при сохранении фонетического образа.

Иначе говоря, олбанске нельзя назвать полноценным языком, поскольку он представляет из себя альтернативную графофонетическую систему для классического русского языка, дополненную характерными фразеологизмами.
Ещё одно определение, данное анонимусом на Луркоморье: ``Важным отребутам труЪолдфага олбанскава изыка ивляицця токое и толька токое езкоженнее слоф, пре каторам в кашдам слови дапусчины фсе васмошныи — но! — толька васмошныи ашыпке.''

Примеры фразеологизмов:

\begin{itemize}

\item Аффтар, выпей йаду!

\item Убей сибя апстенку!

\item Ф Бабруйзг, жывотнае!

\item Аффтар жжот нипадецки.

\end{itemize}

Возник, по одной из версий, как пародия за распространённые ошибки во время деэлитаризации интернета. По другой -- был предназначен для усложнения индексации страниц поисковыми системами. Сейчас падонкаффский используется на некоторых ресурсах, и зачастую сложно отличить его от обыкновенных ошибок, малограмотно пишуших иностранцев и т. д.
Заметим, что предпринимались даже попытки составить свод правил олбанске.

\subsection{ОНОТОЛИЦА}

ОНОТОЛИЦА - полноценный язык, разработанной Упячкой -- интернет-движением, которое осветить в рамках данной работы не представляется возможным -- для деморализующего воздействия на представителей субкультур, которые упячкинцы считают ``унылыми'': эмо, рэперов и т. д. 

Характерными чертами онотолицы являются:

\begin{itemize}

\item Использование исключительно ЗАГЛАВНЫХ БУКВ -- для блочного восприятия текста.

\item Своя графофонетическая система: буква Ъ, например, обозначает звук, средний между А, О и Ы.

\item Своя падежная система: большинство слов имеют окончание -Е, предлог ``В'' почти не используется.

\item Ограниченный, но очень своеобразный лексический запас: ПЯНИ, ПЕПЯКА, ЖЫВТОНЕ, ЩАЧЛО, ПСТО и т. д.

\end{itemize}

\subsection{Луркояз}

Луркояз - публицистический диалект русского языка, зародившийся в свободной энциклопедии Луркоморье и затем заимствовавший многие особенности других сленгов интернета.

Характеризуется:

\begin{itemize}

\item Наличием фразеологизмов (чуть более, чем полностью; люто, бешено доставляет; эпик фейл и т. д.)

\item Графофонетической системой, полностью совпадающей с русским языком, за исключением использования твёрдого знака произвольным образом (``труъ'' -- истинный).

\item Переосмыслением этимологии слов: поц -- поцреот, моск -- Москва и т. д. Подобное встречалось и в русском языке.

\item Значительным собственным лексическим запасом, который, хотя и содержит не столь значительное количество эндемичных слов, покрывает (совместно с фразеологизмами) потребность выражения мыслей кратко, чётко и оригинально.

\item Большим количеством аббревиатур: ЧСВ - чувство собственной важности и т. д.

\item Наличием слов и выражений, сохраняющих английское написание (over 9000).

\end{itemize}

В значительной степени исследован собственно на Луркоморье, используется анонимусом и пиратами, а также рядовыми пользователями.

Необходимо, как это ни печально, отметить, что во всех трёх упомянутых искажениях весьма часто используется обсценная лексика, которая, впрочем, почти никогда не искажается и потому интереса не представляет.

