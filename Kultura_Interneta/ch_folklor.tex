\section{Мемы.}

\opred

Мем есть естественное обобщение фразеологизма на произвольный тип информации, т. е. устойчивый, узнаваемый образ, несущий иносказательное значение.

Примеры мемов:

\begin{itemize}

\item Advice Dog -- собака, раздающая вредные или шутливые советы.

\item Троллфейс -- нагло ухмыляющаяся морда, как правило, сопровождающая насмешки.

\item Сферический конь в вакууме -- чересчур абстрагированный объект.

\item Британские учёные выяснили... -- о результатах сомнительной научной работы.

\end{itemize}

Некоторые мемы возникают в результате форсирования - многократного повторения (вспомним про низкую персонифицируемость в интернете).

\section{Пасты.}

Пастой называется относительно длинный текст (несколько абзацев), распространяемый путём прямой копипасты (отсюда и название), т. е. копирования и вставки практически без внесения изменений. Как правило, пасты - это анонимные выдуманные или реальные истории.


\section{Пикчи.}

Пикчей -- от англ. picture -- называют в интернете стабильную картинку, не важно, изображена ли на ней карикатура или кошка с котятами. Возможности интернета позволяют распространять изображения быстро и без искажений, поэтому особенно остроумые или имеющие иную художественную ценность пикчи переходят в фольклор.

Отдельной строкой следует выделить демотиватор -- изображение, состоящее из фотографии или рисунка, помещённых в чёрную рамку, в нижней части которого нанесена как правило саркастическая подпись.


