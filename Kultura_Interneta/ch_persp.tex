Задача прогнозирования развития интернета, очевидно. является достаточно сложной.
Однако уже сейчас с уверенностью можно сказать, что русский язык становится реальным языком международного киберобщения как минимум на постсоветскои просторанстве; разумно ожидать в ближайшее время свода правил русского языка, отличных от академического русского и составленных лидерами Интернета.

Набирает силу OpenSource; не дремлют и пираты. В ближайшее время развернётся серьёзнейшее противостоятние между сторонниками и противниками авторского права.
