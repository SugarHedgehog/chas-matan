\documentclass[a4paper]{article}
\usepackage[T2A]{fontenc}
\usepackage[utf8]{inputenc} % любая желаемая кодировка
\usepackage[russian,english]{babel}
\usepackage[pdftex,unicode]{hyperref}
\usepackage{indentfirst} % включить отступ у первого абзаца
\usepackage{amssymb}

\begin{document} % начало документа

Докажем, что функция $f(x)=x^a$ непрерывна на области определения. Доказательство проведём в две части: сначала докажем утверждение для неотрицательных $x$, затем - для рациональных $a$. Так как при отрицательных $x$ и иррациональных $a$ функция $f$ не определена, то мы докажем требуемую непрерывность.

Часть 1.

Пусть $x>0$. Тогда
$$
x^a=\left(e^{\ln x}\right)^a=e^{a \ln x}
$$
Суперпозиция непрерывных функций (в нашем случае - показательной и обратной к ней логарифмической, а также умножения на константу) непрерывна. Пусть теперь $x=0$, тогда $x^a=0$ и функция $f$ определена только для $a>0$. Докажем критерий непрерывности функции, пользуясь тем, что для непрерывной функции знак предела и знак функции можно менять местами:
\[
\lim_{x \to 0}x^a=\lim_{x \to 0}e^{a \ln x}=0=0^a
\]
Таким образом, для $x>0$ непрерывность $f(x)=x^a$ доказана.

Часть 2.

Пусть $a=\frac{p}{q}, p \in \mathbb{Z}, q \in \mathbb{N}$, т. е. $a \in \mathbb{Q}$.
Рассмотрим сначала $g(x)=x^q$. Она непрерывна на области определения, т. к. является произведением конечного числа тождественных функций $y(x)=x$. $h(x)=x^{-q}=\frac{1}{x^q}$ также непрерывна на области определения, т. к. является частным непрерывных функций. Значит, непрерывна и $\psi(x)=x^p$. $\phi(x)=x^\frac{1}{q}$ непрерывна, т. к. является (по определению) обратной к непрерывной функции $g(x)$. Однако $f(x)=\psi(\phi(x))$.
Т. к. суперпозиция непрерывных функций является непрерывной функцией, то $f(x)$ непрерывна, ЧИТД.

\end{document} % конец документа
