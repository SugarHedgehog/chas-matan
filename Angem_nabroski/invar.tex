\documentclass[a4paper,11pt]{article}
\usepackage[T2A]{fontenc}
\usepackage[utf8]{inputenc} % любая желаемая кодировка
\usepackage[russian,english]{babel}
\usepackage[pdftex,unicode]{hyperref}
\usepackage{indentfirst} % включить отступ у первого абзаца

\begin{document} % начало документа

$ 2a' \cdot 2c' - 4b' = $

(подставляя значения $2a'$ и $2c'$)

$ ((a+c)-(2b \sin 2 \varphi + (c-a) \cos 2 \varphi))
\cdot
((a+c)+(2b \sin 2 \varphi + (c-a) \cos 2 \varphi))-$

$-(2b')^2 = $

(первое слагаемое - разность квадратов)

$ (a+c)^2-(2b \sin 2 \varphi + (c-a) \cos 2 \varphi)^2 - (2b')^2 = $

(немного преобразуем знаки во второй скобке)

$(a+c)^2-(2b \sin 2 \varphi - (a-c) \cos 2 \varphi)^2 - (2b')^2 =
$

(и ещё чуть-чуть)

$(a+c)^2-((a-c) \cos 2 \varphi - 2b \sin 2 \varphi)^2 -(2b')^2 =
$

(теперь подставляем $2b'$)

$(a+c)^2-((a-c) \cos 2 \varphi - 2b \sin 2 \varphi)^2 - $

$-((a-c) \sin 2 \varphi + 2b \cos 2 \varphi)^2 =
$

(деваться некуда, раскрываем второе и третье слагаемое по формулам квадрата суммы и разности соответственно)

$
(a+c)^2-
$

$
-(
(a-c)^2 \cos ^2 2 \varphi
- 4 b (a-c) \sin 2 \varphi \cos 2 \varphi
+ 4b^2 \sin ^2 2 \varphi
)-
$

$
-(
(a-c)^2 \sin ^2 2 \varphi
+ 4 b (a-c) \sin 2 \varphi \cos 2 \varphi
+ 4b^2 \cos ^2 2 \varphi
)=
$

(частично раскрываем скобки)

$(a+c)^2
-(a-c)^2 \cos ^2 2 \varphi
+ 4 b (a-c) \sin 2 \varphi \cos 2 \varphi
- 4b^2 \sin ^2 2 \varphi -$

$-(a-c)^2 \sin ^2 2 \varphi
- 4 b (a-c) \sin 2 \varphi \cos 2 \varphi
- 4b^2 \cos ^2 2 \varphi
=$

(приводим подобные слагаемые)

$(a+c)^2
-(a-c)^2 (\cos ^2 2 \varphi + \sin ^2 2 \varphi)
- 4b^2 (\sin ^2 2 \varphi + \cos ^2 2 \varphi)
=$

(используем основное тригонометрическое тождество)

$(a+c)^2 - (a-c)^2 - 4b^2 =$

(раскрываем оставшиеся скобки по формулам квадрата суммы и разности соответственно)

$a^2 + 2ac + c^2 - (a^2 - 2ac + c^2) - 4b^2 =$

(и последние скобки)

$a^2 + 2ac + c^2 - a^2 + 2ac - c^2 - 4b^2 =$

(приводим подобные)

$4ac - 4b^2 $

Вуаля!



\end{document} % конец документа
