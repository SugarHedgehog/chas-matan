\documentclass[a4paper,14pt]{report} %размер бумаги устанавливаем А4, шрифт 12пунктов
\usepackage[T2A]{fontenc}
\usepackage[utf8]{inputenc}
\usepackage[english,russian]{babel} %используем русский и английский языки с переносами
\usepackage{amssymb,amsfonts,amsmath,mathtext,cite,enumerate,float} %подключаем нужные пакеты расширений
\usepackage[pdftex,unicode,colorlinks=true,linkcolor=blue]{hyperref}
\usepackage{indentfirst} % включить отступ у первого абзаца
\usepackage[dvips]{graphicx} %хотим вставлять рисунки?
\graphicspath{{illustr/}}%путь к рисункам

\makeatletter
\renewcommand{\@biblabel}[1]{#1.} % Заменяем библиографию с квадратных скобок на точку:
\makeatother %Смысл этих трёх строчек мне непонятен, но поверим "Запискам дебианщика"

\usepackage{geometry} % Меняем поля страницы. 
\geometry{left=1cm}% левое поле
\geometry{right=1cm}% правое поле
\geometry{top=1cm}% верхнее поле
\geometry{bottom=2cm}% нижнее поле

\renewcommand{\theenumi}{\arabic{enumi}}% Меняем везде перечисления на цифра.цифра
\renewcommand{\labelenumi}{\arabic{enumi}}% Меняем везде перечисления на цифра.цифра
\renewcommand{\theenumii}{.\arabic{enumii}}% Меняем везде перечисления на цифра.цифра
\renewcommand{\labelenumii}{\arabic{enumi}.\arabic{enumii}.}% Меняем везде перечисления на цифра.цифра
\renewcommand{\theenumiii}{.\arabic{enumiii}}% Меняем везде перечисления на цифра.цифра
\renewcommand{\labelenumiii}{\arabic{enumi}.\arabic{enumii}.\arabic{enumiii}.}% Меняем везде перечисления на цифра.цифра


\LARGE
\begin{document}
\newcommand{\pp}{Предположим противное}
%\newcommand{\pp}{{\LARП\!\!\!\!п~}}
\newcommand{\dokvo}{\paragraph{Доказательство.}}
\newcommand{\dokno}{\textbf {Доказано.}}
\newcommand{\neobh}{\paragraph{Необходимость.}}
\newcommand{\dost }{\paragraph{Достаточность.}}
\newcommand{\opred}{\paragraph{Определение.}}
\newcommand{\mnemo}{\paragraph{Мнемоника.}}
\newcommand{\N}{\mathbb{N}}
\newcommand{\Z}{\mathbb{Z}}
\newcommand{\Q}{\mathbb{Q}}
\newcommand{\R}{\mathbb{R}}
\renewcommand{\C}{\mathbb{C}}
\newcommand{\Beta}{B}%Костыль, а что поделать?
\newcommand{\Rn}{$\mathbb{R}^n~$}
\newcommand{\Rm}{$\mathbb{R}^m~$}
\renewcommand{\epsilon}{\varepsilon}
\renewcommand{\geq}{\geqslant}
\renewcommand{\leq}{\leqslant}
\newcommand{\fXR}{Пусть $X \subset \R, f:X \to \R$ }
\newcommand{\fXRx}{\fXR, $x_0$ - предельная точка $X$ }
\newcommand{\sgn}{\mathrm{sgn}~}
\newcommand{\nid}{\Leftrightarrow}
\newcommand{\intl}{\int\limits}
\newcommand{\Models}{|\!\!\!=\!\!\!|}
\newcommand{\Rightleftarrow}{\Leftrightarrow}

\newcommand{\xI}{{\vec{\xi}}}%Костыль для тервера, очень уж там часто встречается
\newcommand{\calF}{\mathcal{F}}
\newcommand{\calB}{\mathcal{B}}
\newcommand{\GOFP}{$G \sim \left<\Omega,\calF,P\right>$}

\newenvironment{zamena}[1][c]{=\left<\begin{array}{#1}}{\end{array}\right>=}

\newtheorem{theorem}{Теорема}[section]
\newenvironment{teorema}[1][{}]{\begin{theorem}{#1}\upshape}{\end{theorem}}

\theoremstyle{definition}

\newtheorem{zamech}{Замечание}[section]
\newtheorem{primer}{Пример}[section]
\newtheorem{opr}{Определение}[section]

\newtheorem{sledstvie}{Следствие}[theorem]
\newtheorem{utverzhd}[theorem]{Утверждение}
\newtheorem{lemma}[theorem]{Лемма}

\long\def\comment{}

\LARGE

\opred
Высказывание -- это предложение, относительно которого имеет смысл утверждать, истинно оно или ложно.

\opred
Предикат -- это предложение, относящееся к одному или не\-с\-коль\-ким неопределённым объектам и обращающееся в высказывание всякий раз, когда все входящие в него неопределённые объекты заменены конкретными представителями.

\opred
Имена неопределённых объектов предиката называются переменными данного предиката.
Каждая переменная имеет область определения, которая должна указываться или подразумеваться.

\opred
Пусть $A_1, ..., A_n, B$ -- предикаты.
Высказывание вида:\\ ``Из $A_1, ..., A_n$ следует $B$'', независимо от того, на чём оно основано, называется умозаключением.
Предикаты $A_1, ..., A_n$ называются посылками, $B$ -- заключением.

\opred
Логическая форма предложения определяется следующими словами и словосочетаниями:

1) ``не'' -- отрицание $\lnot A$;

2) ``и'' -- конъюнкция $A \cap B$;

3) ``или'' -- дизъюнкция $A \cup B$;

4) ``если ..., то ...'' -- импликация  $A \to B$;

5) ``если и только если'' -- двойная импликация  $A \leftrightarrow B$;

6) ``для любого'' -- общность $\forall$;

7) ``существует'' -- существование $\exists$;

\opred

Пусть дан некоторый список предикатов.
Интерпретацией этого списка предикатов называется такой набор произвольных смысловых значений для всех его переменных, что при придании данным предикатам данных смысловых значений форма предикатов не меняется, а сами предикаты становятся высказываниями.

\opred

Контрпримером к умозаключению называется такая его интерпретация, при которой все посылки истинны, а заключение -- ложно.

\opred
Говорят, что заключение $B$ логически следует из посылок $A_1, ..., A_n$, если к умозаключению ``Из $A_1, ..., A_n$ следует $B$'' не существует контрпримера.
Пишут: $$A_1, ..., A_n \models B$$

\opred

Пусть $T$ - некоторая теория.
Говорят, что в этой теории из посылок $A_1, ..., A_n$ следует $B$, если список посылок можно дополнить истинными в $T$ утверждениями $T_1, ..., T_m$ так, что из расширенного списка посылок предложение $B$ следует логически, т. е. $A_1, ..., A_n, T_1, ..., T_m \models B$.
Пишут: $$A_1, ..., A_n \models_T B$$

\opred
Стандартной интерпретацией предиката или списка предикатов называется придание все элементарным предикатам истинностных значений И (1, истина) или Л (0, ложь).
При стандартной интерпретации различным вхождениям одного и того же элементарного предиката придаются одинаковые истинностные значения.

\opred
Списки предикатов $A_1, ..., A_n$ и $B_1, ..., B_n$ называются логически эквивалентными, если каждый из них является логическим следствием другого.
Пишут:
$$A_1, ..., A_n \Models B_1, ..., B_n$$

\opred
Предикат $B$ называется тавтологией, если он истинен в любой интерпретации.
Пишут:
$$\models B$$

\opred

Список предикатов $A_1, ..., A_n$ называется логически противоречивым (или противоречием), если входящие в него предикаты не могут быть одновременной истинными ни при какой интерпретации.

\opred

Конъюнкция нескольких простых предикатов или их отрицаний называется элементарной конъюнкцией.

\opred

Дизъюнкция нескольких различных элементарных конъюнкций называется дизъюнктивной нормальной формой (днф).

\opred

Днф называется совершенной (сднф), если во всех её элементарных конъюнкциях участвуют все предикаты, входящие в данную форму.

\opred

Дизъюнкция нескольких простых предикатов или их отрицаний называется элементарной дизъюнкцией.

\opred

Конъюнкция нескольких различных элементарных дизъюнкций называется конъюнктивной нормальной формой (кнф).

\opred

Кнф называется совершенной (скнф), если во всех её элементарных дизъюнкциях участвуют все предикаты, входящие в данную форму.

\opred

Система булевых функций $\{f_1,...,f_n\}$ называется полной, если любую булеву функцию (в том числе константы 0 и 1) моэно записать через функции $f_1,...,f_n$.

%Вторая часть

\opred

Константы --- имена индивидуальныхх предметов.

\opred

Переменные --- имена неопределённых предметов, конкретные значения которых выбираются в единой для всех переменных предметной области.

\opred

Функциональные знаки --- это знаки, с помощью которых из простых выражений (констант и переменных) образуются сложные.

\opred

Знак отношения --- функциональный знак, действующий в двухэлементое множество булевых констант.

\opred

Простой предикат --- отношение от $n$ переменных, применённое к $n$ выражениям.

\opred

Сложный предикат --- предикат, построенный их простых предикатов с помощью логических операций.

\opred

Связанная переменная --- это переменная, к которой применён какой-либо квантор.

\opred

Свободная переменная --- это переменная, к которой не применён ни один квантор.

\opred

Предметная область набора формул --- это множество всех участвующих в данном наборе формул констант и определений, дополненное бесконечным множеством стандартных констант $\{c_0, c_1, c_2 ...\}$.


\end{document}
