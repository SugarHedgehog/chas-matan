%\documentclass[a4paper,14pt]{article}
\documentclass[a4paper,14pt]{report} %размер бумаги устанавливаем А4, шрифт 12пунктов
\usepackage[T2A]{fontenc}
\usepackage[utf8]{inputenc}
\usepackage[english,russian]{babel} %используем русский и английский языки с переносами
\usepackage{amssymb,amsfonts,amsmath,mathtext,cite,enumerate,float} %подключаем нужные пакеты расширений
\usepackage[pdftex,unicode,colorlinks=true,linkcolor=blue]{hyperref}
\usepackage{indentfirst} % включить отступ у первого абзаца
\usepackage[dvips]{graphicx} %хотим вставлять рисунки?
\graphicspath{{illustr/}}%путь к рисункам

\makeatletter
\renewcommand{\@biblabel}[1]{#1.} % Заменяем библиографию с квадратных скобок на точку:
\makeatother %Смысл этих трёх строчек мне непонятен, но поверим "Запискам дебианщика"

\usepackage{geometry} % Меняем поля страницы. 
\geometry{left=1cm}% левое поле
\geometry{right=1cm}% правое поле
\geometry{top=1cm}% верхнее поле
\geometry{bottom=2cm}% нижнее поле

\renewcommand{\theenumi}{\arabic{enumi}}% Меняем везде перечисления на цифра.цифра
\renewcommand{\labelenumi}{\arabic{enumi}}% Меняем везде перечисления на цифра.цифра
\renewcommand{\theenumii}{.\arabic{enumii}}% Меняем везде перечисления на цифра.цифра
\renewcommand{\labelenumii}{\arabic{enumi}.\arabic{enumii}.}% Меняем везде перечисления на цифра.цифра
\renewcommand{\theenumiii}{.\arabic{enumiii}}% Меняем везде перечисления на цифра.цифра
\renewcommand{\labelenumiii}{\arabic{enumi}.\arabic{enumii}.\arabic{enumiii}.}% Меняем везде перечисления на цифра.цифра


\newcommand{\pp}{Предположим противное}
%\newcommand{\pp}{{\LARП\!\!\!\!п~}}
\newcommand{\dokvo}{\paragraph{Доказательство.}}
\newcommand{\dokno}{\textbf {Доказано.}}
\newcommand{\neobh}{\paragraph{Необходимость.}}
\newcommand{\dost }{\paragraph{Достаточность.}}
\newcommand{\opred}{\paragraph{Определение.}}
\newcommand{\mnemo}{\paragraph{Мнемоника.}}
\newcommand{\N}{\mathbb{N}}
\newcommand{\Z}{\mathbb{Z}}
\newcommand{\Q}{\mathbb{Q}}
\newcommand{\R}{\mathbb{R}}
\renewcommand{\C}{\mathbb{C}}
\newcommand{\Beta}{B}%Костыль, а что поделать?
\newcommand{\Rn}{$\mathbb{R}^n~$}
\newcommand{\Rm}{$\mathbb{R}^m~$}
\renewcommand{\epsilon}{\varepsilon}
\renewcommand{\geq}{\geqslant}
\renewcommand{\leq}{\leqslant}
\newcommand{\fXR}{Пусть $X \subset \R, f:X \to \R$ }
\newcommand{\fXRx}{\fXR, $x_0$ - предельная точка $X$ }
\newcommand{\sgn}{\mathrm{sgn}~}
\newcommand{\nid}{\Leftrightarrow}
\newcommand{\intl}{\int\limits}
\newcommand{\Models}{|\!\!\!=\!\!\!|}
\newcommand{\Rightleftarrow}{\Leftrightarrow}

\newcommand{\xI}{{\vec{\xi}}}%Костыль для тервера, очень уж там часто встречается
\newcommand{\calF}{\mathcal{F}}
\newcommand{\calB}{\mathcal{B}}
\newcommand{\GOFP}{$G \sim \left<\Omega,\calF,P\right>$}

\newenvironment{zamena}[1][c]{=\left<\begin{array}{#1}}{\end{array}\right>=}

\newtheorem{theorem}{Теорема}[section]
\newenvironment{teorema}[1][{}]{\begin{theorem}{#1}\upshape}{\end{theorem}}

\theoremstyle{definition}

\newtheorem{zamech}{Замечание}[section]
\newtheorem{primer}{Пример}[section]
\newtheorem{opr}{Определение}[section]

\newtheorem{sledstvie}{Следствие}[theorem]
\newtheorem{utverzhd}[theorem]{Утверждение}
\newtheorem{lemma}[theorem]{Лемма}

\long\def\comment{}

\linespread{1.468}
\begin{document} % начало документа
\large

\begin{opr}
	Множество точек $M$ называется системой целочисленно удалённых точек (СЦТ), если
	$$
		\forall(M_1 \in M, M_2\in M)[|M_1 M_2|\in\Z],
	$$
	и при этом $M$ не содержится целиком ни в какой прямой.
\end{opr}

\begin{zamech}
	Мы рассматриваем случай точек на плоскости, т. е. в $\R^2$.
\end{zamech}

\begin{zamech}
	Прямую, разбитую точками на целочисленные отрезки, мы, как видно из определения, не рассматриваем.
	Аналогично не представляют для нас интереса множества, состоящие из одной или двух точек.
\end{zamech}

\begin{opr}
	Количество точек в СЦТ $S$ называется её мощностью $P(S)$.
\end{opr}

\begin{lemma}[доказана в теме <<Применение фокального свойства гиперболы>> \cite{angem1kurs}]\label{lemma_Semenova}
	Если в СЦТ $S$ три точки $M_1$, $M_2$ и $M_3$ не лежат на одной прямой и 
	$a=|M_1 M_2| \in \mathbb{N}$,
	$b=|M_1 M_3| \in \mathbb{N}$,
	$c=|M_2 M_3| \in \mathbb{N}$,
	то 
	$P(S) \leq 4\cdot\min\{ab,ac,bc\}$.
\end{lemma}

\begin{sledstvie}
	Любая СЦТ конечна.
\end{sledstvie}

\begin{lemma}\label{lemma_ocenka_bestriad_1}
	Если в СЦТ $S$ найдётся $\beta = 2m^2 +1$ точек, никакие три из которых не лежат на одной прямой,
	и $S$ лежит в пределах квадрата со стороной $n$,
	то $n > \frac{\beta - 1}{4}$ (иначе говоря, $ \beta < 4n +1$).
\end{lemma}

\dokvo
	СЦТ лежит в пределах квадрата со стороной $n$.
	Разобьём этот квадрат на $m^2$ меньших равных между собой квадратов со стороной $\frac{n}{m}$.
	Тогда по принципу Дирихле найдётся хотя бы один квадрат со стороной $\frac{n}{m}$,
	внутри которого (возможно, включая границы) найдутся три точки, принадлежащие рассматриваемой СЦТ.
	Обозначим их через $M_1$, $M_2$ и $M_3$.
	Ни одно из расстояний $|M_1 M_2|$, $|M_1 M_3|$ и $|M_2 M_3|$, очевидно, не превышает диагонали квадрата со стороной $\frac{n}{m}$,
	т. е. $\frac{n}{m}\sqrt{2}$.
	Тогда по лемме 1 количество точек в СЦТ $\beta \le \frac{8n^2}{m^2}$.
	Имеем:
	$$ 2m^2+1 \le \frac{8n^2}{m^2}$$
	$$ 2m^2 < 2m^2+1 \le \frac{8n^2}{m^2}$$
	$$ 2m^2 < \frac{8n^2}{m^2}$$
	$$ m^2 < \frac{4n^2}{m^2}$$
	$$ m^4 < 4n^2$$
	Т. к. $n$ положительно, извлекаем корень:
	$$ m^2 < 2n$$
	$$ 2m^2 +1 < 4n +1$$
	$$ \beta < 4n +1$$
	$$n > \frac{\beta - 1}{4}$$
\\ Лемма доказана.

\begin{opr}
	Пусть даны точки $M_1$ и $M_2$ такие, что $|M_1 M_2|\in\Z$.
	Семейством целоразностных линий (СЦРЛ) для точек $M_1$ и $M_2$ назовём объединение семейства софокуных гипербол,
	имеющих фокусы в $M_1$ и $M_2$ и являющихся геометрическими местами точек, разность расстояний от которых до $M_1$ и $M_2$ есть число целое,
	а также вырожденной гиперболы,
	образованной объединением серединного перпендикуляра к отрезку $M_1 M_2$ и прямой $M_1 M_2$.
	Гиперболу $Q$ будем называть вырожденной частью данного СЦРЛ,
	объединение остальных гипербол --- собственной частью СЦРЛ,
	прямую $M_1 M_2$ --- несобственной частью.
\end{opr}

\begin{utverzhd}
	Пусть $|M_1 M_2|= n \in\Z$.
	Тогда СЦРЛ $R$ для $M_1$ и $M_2$ есть объединение не более $\frac{n+1}{2}$ гипербол (возможно, вырожденных).
\end{utverzhd}

\dokvo
	Собственная часть $R$ состоит из гипербол, являющихся геометрическими местами точек,
	разность расстояний от которых до $M_1$ и $M_2$ есть цисло целое.
	Таких гипербол не более $\frac{n-1}{2}$.
	Кроме того, есть ещё вырожденная часть $R$, являющаяся вырожденной гиперболой.
\\ Утверждение доказано.

Теперь сформулируем лемму, аналогичную лемме \ref{lemma_Semenova}.

\begin{lemma}
	Пусть в СЦТ $S$ три точки $M_1$, $M_2$ и $M_3$ лежат на одной прямой $l$ и 
	$a=|M_1 M_3| \in \mathbb{N}$,
	$b=|M_1 M_2| \in \mathbb{N}$,
	$c=|M_2 M_3| \in \mathbb{N}$,
	при этом $a=b+c$, т. е. точка $M_2$ лежит между $M_1$ и $M_3$.
	Обозначим через $S^*$ СЦТ, получаемую из $S$ исключением всех точек, лежащих на $l$:
	$$
		S^* = S \setminus l
	$$
	Тогда
	$P(S^*) \leq (b-1)(c-1)$.
\end{lemma}

\dokvo
	Пусть точка $M \in S^*$.
	Построим СЦРЛ $P$ для $M_1$ и $M_2$ и СЦРЛ $R$ для $M_2$ и $M_3$.
	Так как $\{|M M_1|,|M M_2|,|M M_3|\}\subset\Z$, то $M\in A = (P \cup R)\setminus l$.
	Выясним теперь мощность множества $A$.
	Невырожденные части СЦРЛ $P$ и $Q$ имеют не более $4\cdot{\frac{b-1}{2}}\cdot{\frac{c-1}{2}}=(b-1)(c-1)$ точек пересечения;
	Вырожденные же части пересекаются только по прямой $l$, на которой, по условию, нет точек из $S^*$.
\\ Лемма доказана.


\begin{utverzhd}[вспомогательное]
	$\forall \left(\beta \in \mathbb N\right)\left[  2 \sqrt{\beta - 1} \leq \beta \right]$
\end{utverzhd}

\dokvo
	Т. к. $\beta >0$, возводим обе части неравенства в квадрат:

	$$4 (\beta - 1) \leq \beta^2$$
	$$ (\beta-2)^2 \geq 0$$
\\ Утверждение доказано.

\begin{opr}
	Назовём СЦТ $S$ бестриадной, если никакие три точки из $S$ не лежат на одной прямой.
\end{opr}

\begin{opr}
	Большим диаметром $D(S)$ СЦТ $S$ называется диаметр наименьшего круга, целиком покрывающего СЦТ $S$.
\end{opr}

\begin{opr}
	Малым диаметром $d(S)$ СЦТ $S$ называется максимум из попарных расстояний между её точками.
\end{opr}

Заметим, что оба диаметра, во-первых, определены корректно, во-вторых, конечны, в-третьих, связаны соотношением $d(S)\leq D(s) \leq 2d(s)$ (последнее неравенство вытекает из того, что, выбрав точку $A$ из произвольной пары максимально удалённых точек, можно построить круг радиуса $d(S)$ с центром в точке $A$, который, очевидно, покроет всю СЦТ).


\begin{lemma}
	Пусть СЦТ $S$ --- бестриадна и $S$ лежит внутри квадрата со стороной $n$, $P(S)=\gamma$.
	Тогда $\gamma<4(1+\sqrt{2})n+2+\sqrt{2}$.
\end{lemma}

\dokvo
	Возьмём $m \in \mathbb{N}$ такое, что $2m^2+1 \le \gamma \le 2(m+1)^2$ (это можно сделать единственным образом).
	Обозначим $2m^2+1=\beta$, откуда $m=\sqrt{\frac{\beta-1}{2}}$. Тогда по лемме 2 имеем $ \beta < 4n +1$. Оценим $\gamma$:

	\begin{multline}
		\gamma \le 2(m+1)^2 = 2m^2+4m +2 \leq
		\beta + 1 + 2 \cdot 2 \sqrt{\frac{\beta-1}{2}} =
		\beta + 1 +2\sqrt{2}\sqrt{\beta-1} \leq
		\\ \leq
		(1+\sqrt{2})\beta+1 <
		(4n+1)(1+\sqrt{2})+1 =
		4(1+\sqrt{2})n+2+\sqrt{2}
	\end{multline}
\\ Лемма доказана.

\begin{sledstvie}
	Если СЦТ $S$ бестриадна, то
	$P(S) < 10 D(S)+4$.
	Заметим, что это очень грубое ограничение, указывающее, однако,
	на не более чем линейный характер зависимости максимальной возможной мощности  бестриадной СЦТ от её диаметра.
\end{sledstvie}

В \cite{angem1kurs} Е.М. Семёнов даёт способ построения СЦТ произвольной наперёд заданной мощности, не приводя, однако, зависимость диаметра СЦТ, получаемой при таком построении, от её мощности.
Приведём здесь оригинальный способ построения СЦТ, конструктивно доказав следующую лемму:

\begin{lemma}
	Для любого натурального $k$ найдётся СЦТ $S$ такая, что $P(S)=2k+3$, $D(S)\leq 2^{4k+1}$.
\end{lemma}

\dokvo
	Известно, что $(2mn)^2+(m^2-n^2)^2=(m^2+n^2)^2$.
	Заметим, что $2^{2k+1} = 2 \cdot 2^{2k-p} \cdot 2^p$, где $p$ --- целое число от $0$ до $k-1$.
	Таким образом, мы получили представление числа $2^{2k+1}$ в виде $2mn$ $k$ способами.
	Рассмотрим теперь множество точек
	$$
		S=\{O=(0;0),B_\pm=(0;\pm 2^{2k+1}),A_{\pm p}(\pm 2^{2(2k-p)}-2^{2p};0)\}
	$$
	Покажем, что $S$ --- СЦТ.
	Понятно, что $|O-B_\pm|\in\Z$, $|O-A_{\pm p}|\in\Z$.
	Убедимся, что $|B_\pm-A_{\pm p}|\in\Z$.
	Действительно, $|B_\pm-A_{\pm p}|=|(0;\pm 2^{2k+1}) - (\pm 2^{2(2k-p)}-2^{2p};0)| = \sqrt{(2^{2k+1})^2+(2^{2(2k-p)}-2^{2p})^2}=
	\sqrt{2^{4k+2}+2^{4(2k-p)}-2\cdot 2^{4k}+2^{4p}}=\sqrt{2^{4(2k-p)}+2\cdot 2^{4k}+2^{4p}}=2^{2(2k-p)}+2^{2p}\in\Z$
	Мощность СЦТ $S$ равна в точности $2k+3$.
	Построим круг с центром в $O$ радиуса $2^{4k}$.
	Заметим, что $|O-B_\pm|=2^{2k+1}<2^{4k}$, $|O-A_{\pm p}| = 2^{2(2k-p)}-2^{2p} < 2^{4k}$.
	Значит, построенный круг диаметра $2^{4k+1}$ покрывает СЦТ $S$.
\\ Лемма доказана.


\addcontentsline{toc}{chapter}{Литература}
\begin{thebibliography}{99}

\bibitem{angem1kurs} Аналитическая геометрия на плоскости / Е.М. Семенов, С.Н. Уксусов. – Воронеж : Воронежский государственный университет, 2013. – 100с.

\end{thebibliography}
\end{document} % конец документа
