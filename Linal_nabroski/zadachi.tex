\documentclass[a4paper,14pt]{report} %размер бумаги устанавливаем А4, шрифт 12пунктов
\usepackage[T2A]{fontenc}
\usepackage[utf8]{inputenc}
\usepackage[english,russian]{babel} %используем русский и английский языки с переносами
\usepackage{amssymb,amsfonts,amsmath,mathtext,cite,enumerate,float} %подключаем нужные пакеты расширений
\usepackage[pdftex,unicode,colorlinks=true,linkcolor=blue]{hyperref}
\usepackage{indentfirst} % включить отступ у первого абзаца
\usepackage[dvips]{graphicx} %хотим вставлять рисунки?
\graphicspath{{illustr/}}%путь к рисункам

\makeatletter
\renewcommand{\@biblabel}[1]{#1.} % Заменяем библиографию с квадратных скобок на точку:
\makeatother %Смысл этих трёх строчек мне непонятен, но поверим "Запискам дебианщика"

\usepackage{geometry} % Меняем поля страницы. 
\geometry{left=1cm}% левое поле
\geometry{right=1cm}% правое поле
\geometry{top=1cm}% верхнее поле
\geometry{bottom=2cm}% нижнее поле

\renewcommand{\theenumi}{\arabic{enumi}}% Меняем везде перечисления на цифра.цифра
\renewcommand{\labelenumi}{\arabic{enumi}}% Меняем везде перечисления на цифра.цифра
\renewcommand{\theenumii}{.\arabic{enumii}}% Меняем везде перечисления на цифра.цифра
\renewcommand{\labelenumii}{\arabic{enumi}.\arabic{enumii}.}% Меняем везде перечисления на цифра.цифра
\renewcommand{\theenumiii}{.\arabic{enumiii}}% Меняем везде перечисления на цифра.цифра
\renewcommand{\labelenumiii}{\arabic{enumi}.\arabic{enumii}.\arabic{enumiii}.}% Меняем везде перечисления на цифра.цифра


\newcommand{\pp}{Предположим противное}
%\newcommand{\pp}{{\LARП\!\!\!\!п~}}
\newcommand{\dokvo}{\paragraph{Доказательство.}}
\newcommand{\dokno}{\textbf {Доказано.}}
\newcommand{\neobh}{\paragraph{Необходимость.}}
\newcommand{\dost }{\paragraph{Достаточность.}}
\newcommand{\opred}{\paragraph{Определение.}}
\newcommand{\mnemo}{\paragraph{Мнемоника.}}
\newcommand{\N}{\mathbb{N}}
\newcommand{\Z}{\mathbb{Z}}
\newcommand{\Q}{\mathbb{Q}}
\newcommand{\R}{\mathbb{R}}
\renewcommand{\C}{\mathbb{C}}
\newcommand{\Beta}{B}%Костыль, а что поделать?
\newcommand{\Rn}{$\mathbb{R}^n~$}
\newcommand{\Rm}{$\mathbb{R}^m~$}
\renewcommand{\epsilon}{\varepsilon}
\renewcommand{\geq}{\geqslant}
\renewcommand{\leq}{\leqslant}
\newcommand{\fXR}{Пусть $X \subset \R, f:X \to \R$ }
\newcommand{\fXRx}{\fXR, $x_0$ - предельная точка $X$ }
\newcommand{\sgn}{\mathrm{sgn}~}
\newcommand{\nid}{\Leftrightarrow}
\newcommand{\intl}{\int\limits}
\newcommand{\Models}{|\!\!\!=\!\!\!|}
\newcommand{\Rightleftarrow}{\Leftrightarrow}

\newcommand{\xI}{{\vec{\xi}}}%Костыль для тервера, очень уж там часто встречается
\newcommand{\calF}{\mathcal{F}}
\newcommand{\calB}{\mathcal{B}}
\newcommand{\GOFP}{$G \sim \left<\Omega,\calF,P\right>$}

\newenvironment{zamena}[1][c]{=\left<\begin{array}{#1}}{\end{array}\right>=}

\newtheorem{theorem}{Теорема}[section]
\newenvironment{teorema}[1][{}]{\begin{theorem}{#1}\upshape}{\end{theorem}}

\theoremstyle{definition}

\newtheorem{zamech}{Замечание}[section]
\newtheorem{primer}{Пример}[section]
\newtheorem{opr}{Определение}[section]

\newtheorem{sledstvie}{Следствие}[theorem]
\newtheorem{utverzhd}[theorem]{Утверждение}
\newtheorem{lemma}[theorem]{Лемма}

\long\def\comment{}

\begin{document}

\newcounter{N} % для создания списков, маркированных со стилями, нужен счётчик
\begin{list}{\arabic{N}.}{\usecounter{N}}

\item Покажите, что если система векторов $u_1, ..., u_k$ линейно независима, то система векторов $u_1, u_1+u_2, u_2+u_3, ..., u_{k-1}+u_k$ также линейно независима.

\item Докажите линейную независимость системы функций $\sin x, \cos x$.

\item Докажите линейную зависимость системы функций $1, \sin x, \cos x, \sin^2 x, \cos^2 x$.

\item Покажите, что пространство $M_n((\R))$ есть прямая сумма $M_n((\R))=R_1 \oplus R_2$ подпространства $R_1$ -- симметрических и $R_2$ -- кососимметрических матриц. Найдите проекции $A_1$ и $A_2$ матрицы
$$A=
\begin{pmatrix}
1 & 1 & \cdots & 1\\
0 & 1 & \cdots & 1\\
\vdots & \vdots & \ddots & \vdots\\
0 & 0 & \cdots & 1\\
\end{pmatrix}
$$

на $R_1$ параллельно $R_2$ и на $R_2$ параллельно $R_1$.

\item Докажите, что всякий линейный оператор любую линейно зависимую систему векторов переводит в линейно зависимую систему.

\item Докажите, что всякий линейный оператор $A:R^1 \to R^1$, действующий в одномерном пространстве, имеет вид $A=\lambda I$, т. е. является гомотетией с коэффициентом гомотетии $\lambda$.

\item Пусть $A:P_n \to P_n$ - оператор, определённый равенством $Af(t)=f(t+1)$ (оператор сдвига по аргументу). Покажите, что $A$ - линейный оператор и найдите его матрицу в базисе $1, t, t^2,...,t^n$.

\item Пусть оператор $A:P_n \to P_n$ задан формулой $A f(t)=\frac{f(t)-f(0)}{t}$. Покажите, что $A$ - линейный оператор, найдите его ранг и дефект.

\item Оператор $A_h:P_n \to P_n$ задан формулой $A_h f(t)=\frac{f(t)-f(h)}{h}$. Покажите, что оператор $A_h$ - линейный и найдите его ядро и образ.

\item Найдите общий вид матрицы линейного оператора $A:R^n \to R^n$, первые $k$ векторов которого составляют:\\
а) базис ядра оператора $A$;\\
б) базис образа оператора $A$.

\item Докажите, что для всякого линейного оператора $A:R^n \to R^m$ существуют базисы $e,f$ пространств $R^n$ и $R^m$ соответственно такие, что

$$A_{ef}=\left(
\begin{array}{c|c}
E_r & 0 \\
\hline
0 & 0
\end{array}
\right), r=\mathrm{rank}A
$$

\item Покажите, что всякое подпространство линейного пространства $R^n$ является образом некоторого линейного оператора $A:R^n \to R^n$.

\item Покажите, что всякое подпространство линейного пространства $R^n$ является ядром некоторого линейного оператора $A:R^n \to R^n$.

\item Покажите, что оператор дифференцирования $D:P_n \to P_n$ является вырожденным.

\item Покажите, что линейный оператор $A:R^n \to R^n$ обратим тогда и только тогда, когда $0 \notin \mathrm{Spec}A$.

\item Найдите собственные векторы и собственные значения оператора дифференцирования $D:P_n \to P_n$.

\item Покажите, что все ненулевые векторы пространства являются собственными векторами линейного оператора $A$ тогда и только тогда, когда $A$ - оператор гомотетии, т. е. $A=\lambda I$.

\item Докажите, что геометрическая кратность собственного значения линейного оператора не превосходит его алгебраической кратности.

\item Оператор $A:P_n \to P_n$ зададим формулой $Af(t)=f(t+1)-f(t)$. Покажите линейность оператора $A$ и найдите его спектр.

\item Докажите, что для любых линейных операторов $A,B:r^n \to R^n$ характеристические многочлены операторов $AB$ и $BA$ совпадают.

\item Докажите, что линейная оболочка любой системы собственных векторов линейного оператора инвариантна относительно этого оператора.

\item Пусть $A:R^n \to R^n$ -- линейный оператор. Докажите, что любое подпространство $R_1 \subset R^n$, содержащее $\mathrm{Im}A$, инвариантно относительно оператора $A$.

\item Докажите, что сумма двух и пересечение любого числа инвариантных подпространств линейного оператора -- инвариантные подпространства.

\item Покажите, что если линейные операторы $A,B:R^n \to R^n$ перестановочны, то всякое собственное подпространство оператора $B$ инвариантно относительно оператора $A$.

\item Диагонализируем ли оператор дифференцирования $D:P_n \to P_n$?

\item Покажите, что если линейный оператор $A:R^n \to R^n$ итмеет $n$ различных собственных значений, то любой оператор $B$, перестановочный с $A$, обладает базисом из собственных векторов.

\item Докажите, что равенство $|(x,y)=||x||\cdot||y||$ имеет место тогда и только тогда, когда векторы $x,y$ линейно независимы.

\item Пусть $R_1, R_2$ -- подпространства евклидова пространства и $\mathrm{dim} R_1 < \mathrm{dim} R_2$. Покажите, что в $R_2$ найдётся ненулевой вектор, ортогональный подпространству $R_1$.

\item Докажите, что для любых подпространств  $R_1,R_2$ евклидова или унитарного пространства справедиливо равенство $R_1 + R_2)^\perp =R_1^\perp \cap R_2^\perp$

\item Докажите, что определитель матрицы Грана любой конечной линейно независимой системы векторов евклидова пространства положителен.

\end{list}

\end{document}
