\subsubsection{Поле}
\begin{opr}
Полем называется множество $F$, в котором определены две алгебраические бинарные операции $+$ (сложение) и $\cdot$ (умножение) и выполнены аксиомы:
\end{opr}

\newcounter{axipol} 
\begin{list}{\arabic{axipol}.}{\usecounter{axipol}}

\item $(a+b)+c=a+(b+c)$ -- ассоциативность сложения;

\item $\exists(0\in F)\forall(a\in F)[a+0=a]$ -- наличие нулевого элемента, т. е. нейтрального по сложению;

\item $\forall(a\in F)\exists((-a)\in F)[a+(-a)=0]$ -- обратимость любого элемента по сложению;

\item $a+b=b+a$ -- коммутативность сложения;

\item $(a\cdot b)\cdot c = a\cdot (b\cdot c)$ -- ассоциативность умножения;

\item $\exists(1\in F)\forall(a\in F)[1 \cdot a=a]$ -- наличие единичного элемента, т. е. нейтрального по умножению;

\item $\forall(a\in F, a\neq 0)\exists(a^{-1}\in F)[a\cdot a^{-1} =1]$ -- обратимость по умножению всех элементов, кроме нулевого;
\item $a\cdot b = b\cdot a$ -- коомутативность умножения;
\item $(a+b)\cdot c= a\cdot c + b\cdot c$
\end{list}

\begin{zamech}
Из аксиомы 6 определения поля следует, что поле содержит не менее двух элементов.
\end{zamech}

\begin{zamech}
Фактически аксиомы 1-4 утверждают, что $(F,+)$ -- абелева группа, аксиомы 5-8 -- что $(F \setminus 0,\cdot)$ -- абелева группа, а аксиома 9 связывает операции $+$ и $\cdot$.
\end{zamech}

\begin{primer}
Каждое из множеств $\Q$, $\R$ и $\C$ с обычными операциями сложения и умножения является полем.
\end{primer}

\begin{primer}
Множество $\Q(\sqrt)=\{x|x=p+q\sqrt{2},p\in \Q, q\in\Q\}$ с обычными операциями сложения и умножения является полем. Операцию образования такого поля называют расширением поля.
\end{primer}

\begin{primer}
Пусть $p$ - простое число. На множестве $\Z_p=\{0,1,...,p-1\}$ определим операции сложения $\oplus$ и умножения $\odot$ следующим образом: $m\oplus n$ и $m\odot n$ равны остаткам от деления обычной суммы и обычного произведения $m$ и $n$ соответственно. $(\Z_p,\oplus,\odot)$ - поле.
\end{primer}

\begin{primer}
Множества целых чисел $\Z$ и натуральных чисел $\N$ с обычными операциями сложения и умножения полями не является, т. к. не содержат обратного элемента по умножению, например, для $a=2$.
\end{primer}

\begin{primer}
Множество всевозможных рациональных дробей вида $\frac{P(x)}{Q(x)}$, где $P(x)$ и $Q(x)$ - многочлены с вещественными коэффициентами, притом $Q(x)$ - ненулевой многочлен, с обычными операциями сложения и умножения дробей является полем.
\end{primer}

Элементы полей мы будем называть скалярами.

\subsubsection{Линейные пространства}

\begin{opr}
Множество $R$ называется линейным (векторным) пространством над полем $F$ и обозначается $R((F))$, если для $\forall(x,y,z\in R)\forall(\alpha, \beta \in F)$ определены сумма $x+y\in R$ и внешнее умножение $\alpha x \in R$ и выполнены аксиомы:
\end{opr}

\newcounter{axilp} 
\begin{list}{\arabic{axilp}.}{\usecounter{axilp}}
\item
$(x+y)+z=x+(y+z)$ -- ассоциативность сложения;
\item
$\exists(\theta\in R)\forall(x\in R)[x+\theta=x]$ -- существованание нулевого, т. е. нейтрального по сложению, элемента;
\item
$\forall(x\in R)\exists((-x)\in R)[x+(-x)=\theta]$ -- обратимость любого элемента по сложению;
\item
$x+y=y+x$ -- коммутативность сложения;
\item
$\alpha\cdot(\beta\cdot x)=(\alpha\cdot\beta)\cdot x$ -- ассоциативность внешнего умножения;
\item
$1\cdot x=x$
\item
$\alpha\cdot(x+y)=\alpha\cdot x + \alpha\cdot y$
\item
$(\alpha+\beta)\cdot x = \alpha\cdot x + \beta \cdot x$ -- так как операнды внешнего умножения неравноправны, аксиом дистрибутивности две, а не одна, как в определении поля.
\end{list}

Элементы линейного пространства называют векторами, элемент $\theta$ -- нулевым вектором, а вектор $(-x)$ -- вектором, проивоположным вектору $x$. Пространства над полем вещественных чисел $\R$ называют вещественными, над полем комплексных чисел $\C$ - комплексными.

Для удобства восприятия векторы мы будем обозначать малыми латинскими буквами, скаляры - малыми греческими. 

\begin{primer}
Множества $V_1$
\end{primer}
